% Options for packages loaded elsewhere
\PassOptionsToPackage{unicode}{hyperref}
\PassOptionsToPackage{hyphens}{url}
\PassOptionsToPackage{dvipsnames,svgnames,x11names}{xcolor}
%
\documentclass[
  letterpaper,
  DIV=11,
  numbers=noendperiod]{scrreprt}

\usepackage{amsmath,amssymb}
\usepackage{lmodern}
\usepackage{iftex}
\ifPDFTeX
  \usepackage[T1]{fontenc}
  \usepackage[utf8]{inputenc}
  \usepackage{textcomp} % provide euro and other symbols
\else % if luatex or xetex
  \usepackage{unicode-math}
  \defaultfontfeatures{Scale=MatchLowercase}
  \defaultfontfeatures[\rmfamily]{Ligatures=TeX,Scale=1}
\fi
% Use upquote if available, for straight quotes in verbatim environments
\IfFileExists{upquote.sty}{\usepackage{upquote}}{}
\IfFileExists{microtype.sty}{% use microtype if available
  \usepackage[]{microtype}
  \UseMicrotypeSet[protrusion]{basicmath} % disable protrusion for tt fonts
}{}
\makeatletter
\@ifundefined{KOMAClassName}{% if non-KOMA class
  \IfFileExists{parskip.sty}{%
    \usepackage{parskip}
  }{% else
    \setlength{\parindent}{0pt}
    \setlength{\parskip}{6pt plus 2pt minus 1pt}}
}{% if KOMA class
  \KOMAoptions{parskip=half}}
\makeatother
\usepackage{xcolor}
\setlength{\emergencystretch}{3em} % prevent overfull lines
\setcounter{secnumdepth}{5}
% Make \paragraph and \subparagraph free-standing
\ifx\paragraph\undefined\else
  \let\oldparagraph\paragraph
  \renewcommand{\paragraph}[1]{\oldparagraph{#1}\mbox{}}
\fi
\ifx\subparagraph\undefined\else
  \let\oldsubparagraph\subparagraph
  \renewcommand{\subparagraph}[1]{\oldsubparagraph{#1}\mbox{}}
\fi

\usepackage{color}
\usepackage{fancyvrb}
\newcommand{\VerbBar}{|}
\newcommand{\VERB}{\Verb[commandchars=\\\{\}]}
\DefineVerbatimEnvironment{Highlighting}{Verbatim}{commandchars=\\\{\}}
% Add ',fontsize=\small' for more characters per line
\usepackage{framed}
\definecolor{shadecolor}{RGB}{241,243,245}
\newenvironment{Shaded}{\begin{snugshade}}{\end{snugshade}}
\newcommand{\AlertTok}[1]{\textcolor[rgb]{0.68,0.00,0.00}{#1}}
\newcommand{\AnnotationTok}[1]{\textcolor[rgb]{0.37,0.37,0.37}{#1}}
\newcommand{\AttributeTok}[1]{\textcolor[rgb]{0.40,0.45,0.13}{#1}}
\newcommand{\BaseNTok}[1]{\textcolor[rgb]{0.68,0.00,0.00}{#1}}
\newcommand{\BuiltInTok}[1]{\textcolor[rgb]{0.00,0.23,0.31}{#1}}
\newcommand{\CharTok}[1]{\textcolor[rgb]{0.13,0.47,0.30}{#1}}
\newcommand{\CommentTok}[1]{\textcolor[rgb]{0.37,0.37,0.37}{#1}}
\newcommand{\CommentVarTok}[1]{\textcolor[rgb]{0.37,0.37,0.37}{\textit{#1}}}
\newcommand{\ConstantTok}[1]{\textcolor[rgb]{0.56,0.35,0.01}{#1}}
\newcommand{\ControlFlowTok}[1]{\textcolor[rgb]{0.00,0.23,0.31}{#1}}
\newcommand{\DataTypeTok}[1]{\textcolor[rgb]{0.68,0.00,0.00}{#1}}
\newcommand{\DecValTok}[1]{\textcolor[rgb]{0.68,0.00,0.00}{#1}}
\newcommand{\DocumentationTok}[1]{\textcolor[rgb]{0.37,0.37,0.37}{\textit{#1}}}
\newcommand{\ErrorTok}[1]{\textcolor[rgb]{0.68,0.00,0.00}{#1}}
\newcommand{\ExtensionTok}[1]{\textcolor[rgb]{0.00,0.23,0.31}{#1}}
\newcommand{\FloatTok}[1]{\textcolor[rgb]{0.68,0.00,0.00}{#1}}
\newcommand{\FunctionTok}[1]{\textcolor[rgb]{0.28,0.35,0.67}{#1}}
\newcommand{\ImportTok}[1]{\textcolor[rgb]{0.00,0.46,0.62}{#1}}
\newcommand{\InformationTok}[1]{\textcolor[rgb]{0.37,0.37,0.37}{#1}}
\newcommand{\KeywordTok}[1]{\textcolor[rgb]{0.00,0.23,0.31}{#1}}
\newcommand{\NormalTok}[1]{\textcolor[rgb]{0.00,0.23,0.31}{#1}}
\newcommand{\OperatorTok}[1]{\textcolor[rgb]{0.37,0.37,0.37}{#1}}
\newcommand{\OtherTok}[1]{\textcolor[rgb]{0.00,0.23,0.31}{#1}}
\newcommand{\PreprocessorTok}[1]{\textcolor[rgb]{0.68,0.00,0.00}{#1}}
\newcommand{\RegionMarkerTok}[1]{\textcolor[rgb]{0.00,0.23,0.31}{#1}}
\newcommand{\SpecialCharTok}[1]{\textcolor[rgb]{0.37,0.37,0.37}{#1}}
\newcommand{\SpecialStringTok}[1]{\textcolor[rgb]{0.13,0.47,0.30}{#1}}
\newcommand{\StringTok}[1]{\textcolor[rgb]{0.13,0.47,0.30}{#1}}
\newcommand{\VariableTok}[1]{\textcolor[rgb]{0.07,0.07,0.07}{#1}}
\newcommand{\VerbatimStringTok}[1]{\textcolor[rgb]{0.13,0.47,0.30}{#1}}
\newcommand{\WarningTok}[1]{\textcolor[rgb]{0.37,0.37,0.37}{\textit{#1}}}

\providecommand{\tightlist}{%
  \setlength{\itemsep}{0pt}\setlength{\parskip}{0pt}}\usepackage{longtable,booktabs,array}
\usepackage{calc} % for calculating minipage widths
% Correct order of tables after \paragraph or \subparagraph
\usepackage{etoolbox}
\makeatletter
\patchcmd\longtable{\par}{\if@noskipsec\mbox{}\fi\par}{}{}
\makeatother
% Allow footnotes in longtable head/foot
\IfFileExists{footnotehyper.sty}{\usepackage{footnotehyper}}{\usepackage{footnote}}
\makesavenoteenv{longtable}
\usepackage{graphicx}
\makeatletter
\def\maxwidth{\ifdim\Gin@nat@width>\linewidth\linewidth\else\Gin@nat@width\fi}
\def\maxheight{\ifdim\Gin@nat@height>\textheight\textheight\else\Gin@nat@height\fi}
\makeatother
% Scale images if necessary, so that they will not overflow the page
% margins by default, and it is still possible to overwrite the defaults
% using explicit options in \includegraphics[width, height, ...]{}
\setkeys{Gin}{width=\maxwidth,height=\maxheight,keepaspectratio}
% Set default figure placement to htbp
\makeatletter
\def\fps@figure{htbp}
\makeatother
\newlength{\cslhangindent}
\setlength{\cslhangindent}{1.5em}
\newlength{\csllabelwidth}
\setlength{\csllabelwidth}{3em}
\newlength{\cslentryspacingunit} % times entry-spacing
\setlength{\cslentryspacingunit}{\parskip}
\newenvironment{CSLReferences}[2] % #1 hanging-ident, #2 entry spacing
 {% don't indent paragraphs
  \setlength{\parindent}{0pt}
  % turn on hanging indent if param 1 is 1
  \ifodd #1
  \let\oldpar\par
  \def\par{\hangindent=\cslhangindent\oldpar}
  \fi
  % set entry spacing
  \setlength{\parskip}{#2\cslentryspacingunit}
 }%
 {}
\usepackage{calc}
\newcommand{\CSLBlock}[1]{#1\hfill\break}
\newcommand{\CSLLeftMargin}[1]{\parbox[t]{\csllabelwidth}{#1}}
\newcommand{\CSLRightInline}[1]{\parbox[t]{\linewidth - \csllabelwidth}{#1}\break}
\newcommand{\CSLIndent}[1]{\hspace{\cslhangindent}#1}


\KOMAoption{captions}{tableheading}
\makeatletter
\@ifpackageloaded{tcolorbox}{}{\usepackage[many]{tcolorbox}}
\@ifpackageloaded{fontawesome5}{}{\usepackage{fontawesome5}}
\definecolor{quarto-callout-color}{HTML}{909090}
\definecolor{quarto-callout-note-color}{HTML}{0758E5}
\definecolor{quarto-callout-important-color}{HTML}{CC1914}
\definecolor{quarto-callout-warning-color}{HTML}{EB9113}
\definecolor{quarto-callout-tip-color}{HTML}{00A047}
\definecolor{quarto-callout-caution-color}{HTML}{FC5300}
\definecolor{quarto-callout-color-frame}{HTML}{acacac}
\definecolor{quarto-callout-note-color-frame}{HTML}{4582ec}
\definecolor{quarto-callout-important-color-frame}{HTML}{d9534f}
\definecolor{quarto-callout-warning-color-frame}{HTML}{f0ad4e}
\definecolor{quarto-callout-tip-color-frame}{HTML}{02b875}
\definecolor{quarto-callout-caution-color-frame}{HTML}{fd7e14}
\makeatother
\makeatletter
\makeatother
\makeatletter
\@ifpackageloaded{bookmark}{}{\usepackage{bookmark}}
\makeatother
\makeatletter
\@ifpackageloaded{caption}{}{\usepackage{caption}}
\AtBeginDocument{%
\ifdefined\contentsname
  \renewcommand*\contentsname{İçindekiler}
\else
  \newcommand\contentsname{İçindekiler}
\fi
\ifdefined\listfigurename
  \renewcommand*\listfigurename{Şekil Listesi}
\else
  \newcommand\listfigurename{Şekil Listesi}
\fi
\ifdefined\listtablename
  \renewcommand*\listtablename{Tablo Listesi}
\else
  \newcommand\listtablename{Tablo Listesi}
\fi
\ifdefined\figurename
  \renewcommand*\figurename{Şekil}
\else
  \newcommand\figurename{Şekil}
\fi
\ifdefined\tablename
  \renewcommand*\tablename{Tablo}
\else
  \newcommand\tablename{Tablo}
\fi
}
\@ifpackageloaded{float}{}{\usepackage{float}}
\floatstyle{ruled}
\@ifundefined{c@chapter}{\newfloat{codelisting}{h}{lop}}{\newfloat{codelisting}{h}{lop}[chapter]}
\floatname{codelisting}{Liste}
\newcommand*\listoflistings{\listof{codelisting}{Kod Listesi}}
\makeatother
\makeatletter
\@ifpackageloaded{caption}{}{\usepackage{caption}}
\@ifpackageloaded{subcaption}{}{\usepackage{subcaption}}
\makeatother
\makeatletter
\@ifpackageloaded{tcolorbox}{}{\usepackage[many]{tcolorbox}}
\makeatother
\makeatletter
\@ifundefined{shadecolor}{\definecolor{shadecolor}{rgb}{.97, .97, .97}}
\makeatother
\makeatletter
\makeatother
\ifLuaTeX
  \usepackage{selnolig}  % disable illegal ligatures
\fi
\IfFileExists{bookmark.sty}{\usepackage{bookmark}}{\usepackage{hyperref}}
\IfFileExists{xurl.sty}{\usepackage{xurl}}{} % add URL line breaks if available
\urlstyle{same} % disable monospaced font for URLs
\hypersetup{
  pdftitle={ Dersleri},
  pdfauthor={Zafer ACAR},
  colorlinks=true,
  linkcolor={blue},
  filecolor={Maroon},
  citecolor={Blue},
  urlcolor={Blue},
  pdfcreator={LaTeX via pandoc}}

\title{{\LaTeX} Dersleri}
\author{Zafer ACAR}
\date{06/03/2022}

\begin{document}
\maketitle
\ifdefined\Shaded\renewenvironment{Shaded}{\begin{tcolorbox}[interior hidden, sharp corners, boxrule=0pt, borderline west={3pt}{0pt}{shadecolor}, breakable, enhanced, frame hidden]}{\end{tcolorbox}}\fi

\renewcommand*\contentsname{İçindekiler}
{
\hypersetup{linkcolor=}
\setcounter{tocdepth}{2}
\tableofcontents
}
\bookmarksetup{startatroot}

\hypertarget{sunuux15f}{%
\chapter*{Sunuş}\label{sunuux15f}}
\addcontentsline{toc}{chapter}{Sunuş}

Bu kitap, artık aktif olmayan \url{https://texdizgi.com/} adresinde
yazmış olduğum yazıların derlenmesiyle ortaya çıktı. Aslında oradaki
yazılar da {\LaTeX}'le ilgili yazmaya başladığım ve bir türlü sonunun
getiremediğim kitabın tamamlanmış bölümlerinin ürünüydü.

Kitabı doğal olarak {\LaTeX}'le yazıyordum.
\href{https://rmarkdown.rstudio.com/}{RMarkdown} ve
\href{https://quarto.org/}{Quarto} ile tanıştıktan sonra kitabı yalnızca
PDF olarak değil, etkileşime izin vermesi ve daha bir çok avantajından
dolayı HTML olarak da dağıtmaya karar verdim.

Kitap etkileşimli hale \href{https://texlive.net/}{TeXLive.net}
sayesinde geldi. Okurun internet bağlantısı olduğunda etkileşimli HTML
sürümünü kullanması önerilir.

\begin{Shaded}
\begin{Highlighting}[]
\BuiltInTok{\textbackslash{}documentclass}\NormalTok{\{}\ExtensionTok{article}\NormalTok{\}}
\KeywordTok{\textbackslash{}begin}\NormalTok{\{}\ExtensionTok{document}\NormalTok{\}}
\NormalTok{  Merhaba }\FunctionTok{\textbackslash{}LaTeX}\NormalTok{!}
\KeywordTok{\textbackslash{}end}\NormalTok{\{}\ExtensionTok{document}\NormalTok{\}}
\end{Highlighting}
\end{Shaded}

Yukarıdaki ``Derle'' düğmesine bastığınızda {\LaTeX} çıktısı ile
``Çıktıyı Temizle'' düğmesi görünecek.\footnote{Bu düğmelerin PDF
  sürümde olmayacağını unutmayın.} Dilerseniz bu düğmeye basarak çıktıyı
temizleyebilirsiniz. Ayrıca içeriği değiştirerek kendi denemelerinizi
yapabilirsiniz.

Keyifli okumalar.

\bookmarksetup{startatroot}

\hypertarget{temeller}{%
\chapter{Temeller}\label{temeller}}

\hypertarget{tex}{%
\section{\texorpdfstring{{\TeX}}{}}\label{tex}}

{\TeX}, birçok matematiksel ve teknik ifade içeren belgeleri yüksek
kaliteli çıktı ile üretmek için Stanford Üniversitesi'nden Profesör
\href{https://www-cs-faculty.stanford.edu/~knuth/}{Donald E.Knuth}
tarafından, başlangıçta
``\href{https://www-cs-faculty.stanford.edu/~knuth/taocp.html}{Bilgisayar
Programlama Sanatı}'' adlı kitap serisinin hazırlanması için tasarlanmış
bir dizgi programıdır. Teknik kitaplar ve makaleler üreten birçok yazar
ve yayıncı tarafından benimsenmiştir. {\TeX}, Knuth tarafından ücretsiz
olarak kullanıma sunulmuştur.

\hypertarget{latex}{%
\section{\texorpdfstring{{\LaTeX}}{}}\label{latex}}

{\LaTeX} (La(mport){\TeX}), \href{http://www.lamport.org/}{Leslie
Lamport} tarafından {\TeX} üzerine kurulmuştur. Amacı {\TeX}'i daha
işlevli ve kolay hale getirmektir.

Günümüzde Frank Mittelbach liderliğinde bir grup programcı tarafından
\href{https://www.latex-project.org/latex3/}{geliştirilmektedir}.

\hypertarget{temel-farklux131lux131klar}{%
\section{Temel Farklılıklar}\label{temel-farklux131lux131klar}}

{\LaTeX}, genelde WYSIWYG (\emph{Ne Görüyorsan Onu Alırsın})
editörleriyle karşılaştırılır. WYSIWYG, Microsoft Word, Libreoffice
Writer gibi kelime işlemcilere ya da Adobe Indesign gibi programlara
verilen genel bir isimdir. Hepsinin ortak özelliği, girdi ile çıktının
aynı anda ve birlikte görünmesidir.

{\LaTeX}'de girdi ve çıktı ekranı farklıdır ve çıktıyı görmek için
girdinin derleme işleminden geçmesi gerekir. {\LaTeX}'in zor gibi
görünmesinin bir nedeni de komutlarla çalışmasıdır. Gerçekten de WYSIWYG
programlarına alışkın bir kullanıcı için bu süreç başlangıçta biraz
sıkıcı ve zor olabilir. Ancak belli bir aşamaya geldiğinizde komutlarla
çalışmanın aslında işleri kolaylaştırdığını göreceksinizdir. Örneğin
{\LaTeX}'le binlerce sayfadan oluşan bir kitabın içindekiler tablosunu
oluşturmak için sadece \texttt{\textbackslash{}tableofcontents} komutunu
girersiniz ve {\LaTeX} bu komutu girdiğiniz yere içindekiler tablosunu
hatasız bir şekilde yazdırır. Ayrıca örnekte görüldüğü gibi {\LaTeX}
komutları doğal dile\footnote{Tabii ki İngilizce.} oldukça yakındır.

Bir metnin genel görünümü ve okunabilirliği, metnin nasıl
hizalandığından ve kesildiğinden büyük ölçüde etkilenir. {\LaTeX}, tüm
paragraf için hizalamayı ve kesmeleri optimize eden son derece gelişmiş
{\TeX} algoritmalarını kullanır. Kelime işlemciler ve diğer programlar,
satır başına çalıştıkları için oldukça yetersiz kalırlar. Bu, diğer
şeylerin yanı sıra düzensiz aralıklara ve birçok kısa çizgiye sebep
olur. Sonuçları görmeniz için Microsoft Word 2008 (Mac), Adobe InDesign
CS4 ve {\LaTeX}'le dizilmiş bir metni
\href{http://www.rtznet.nl/zink/comparison.pdf}{şuradan}
inceleyebilirsiniz.

Sonuç, {\LaTeX}'in diğer programların her ikisinden de üstün olduğunu
açıkça gösterir: iki kat daha az tireleme kullanır ve yine de sözcük
aralığındaki varyasyon, Word veya InDesign'dan belirgin şekilde daha
azdır. {\LaTeX}'te çok büyük sözcük aralığı içeren satırlar oluşmaz.

\hypertarget{neden-latex}{%
\section{\texorpdfstring{Neden {\LaTeX}}{Neden }}\label{neden-latex}}

{\LaTeX}'i kullanmaya karar vermeniz için birkaç neden:

\begin{itemize}
\item
  Ücretsizdir ve tüm işletim sistemlerinde düşük donanım
  gereksinimleriyle çalışır.
\item
  Basit bir metin editörüyle bile kaynak dosyanızı düzenleyebilirsiniz.
\item
  Her sürüm bir önceki sürümü içerdiğinden şimdi hazırladığınız bir
  belgeyi on yıl sonra da kullanabilirsiniz, hiçbir zaman ``Dosya bu
  sürümle uyumlu değil'\,' biçiminde bir uyarıyla karşılaşmazsınız.
\item
  Bir belgedeki biçim veya metni diğer bir belgeye kolaylıkla
  taşıyabilir ve düzenleyebilirsiniz. Bu, internetten bulabileceğiniz
  binlerce şablonu kullanabileceğiniz anlamına gelir.
\item
  Belgenizi {\LaTeX} biçimlendirir, siz sadece içeriğe odaklanırsınız.
\item
  Dipnot koymak, atıf yapmak, kaynakça, dizin ve içindekiler tablosu
  oluşturmak işten bile sayılmaz.
\item
  Matematiksel formüller kullanacaksanız, bu {\LaTeX}'in en güçlü olduğu
  konudur. Formülleri belgenize adeta \emph{inci dizer gibi} dizer.
\item
  Çok yaygındır. Uluslararası bir çok yayınevi (örneğin Springer) ve
  dergi yazarlardan {\LaTeX} dosyalarını talep eder.
\end{itemize}

\hypertarget{uxf6nemli-yapux131lar}{%
\section{Önemli Yapılar}\label{uxf6nemli-yapux131lar}}

\hypertarget{komutlar}{%
\subsection{Komutlar}\label{komutlar}}

{\LaTeX} komutları bir geribölü (\texttt{\textbackslash{}}) işaretiyle
başlar ve ya sadece harflerden ya da bir tane harf olmayan karakterden
oluşurlar. Komut yazıldıktan sonra ya boşluk, ya bir sayı ya da harf
olmayan bir karakter gelebilir.

Çoğu komut, zorunlu değişken alır. Bu zorunlu değişken komut adından
sonra çengelli parantezler içine yazılır. Zorunlu değişken alan
komutlar, zorunlu olmayan (isteğe bağlı) değişkenler de alabilir, bunlar
da komut adından sonra gelen köşeli parantezler içine yazılırlar. Eğer
değişkenler birden fazlaysa aralarına virgül koyularak ayrılır.

\begin{Shaded}
\begin{Highlighting}[]
\FunctionTok{\textbackslash{}:}
\FunctionTok{\textbackslash{}LaTeX}
\FunctionTok{\textbackslash{}item}\NormalTok{[...]}
\FunctionTok{\textbackslash{}emph}\NormalTok{\{...\}}
\FunctionTok{\textbackslash{}subfloat}\NormalTok{[...][...]\{...\}}
\FunctionTok{\textbackslash{}raisebox}\NormalTok{\{...\}[...][...]\{...\}}
\FunctionTok{\textbackslash{}multicolumn}\NormalTok{\{...\}\{...\}\{...\}}
\NormalTok{\{}\FunctionTok{\textbackslash{}bfseries}\NormalTok{ ...\}}
\end{Highlighting}
\end{Shaded}

Fikir vermesi açısından yukarıda dokuz adet komut örneği verilmiştir.
Birinci komut bir tane harf olmayan karakterden oluşan bir komuttur.
İkincisi, değişkeni olmayan bir komuttur. Bazı harflerin büyük
bazılarınınsa küçük olması komutların büyük-küçük harfe duyarlı olduğunu
gösterir. Dokuzuncu komut ise bildirim şeklinde verilmiştir.

\hypertarget{paketler}{%
\subsection{Paketler}\label{paketler}}

{\LaTeX}'de bazı özelliklerin (renkli yazmak, şekil eklemek vb.)
kullanılabilmesi için kaynak dosyaya bazı paketlerin eklenmesi gerekir.
Bu, \texttt{\textbackslash{}usepackage} komutuyla yapılır. Bu komutun
zorunlu değişkenine paket adı, zorunlu olmayan kısmına ise paket
seçenekleri yazılır:

\begin{Shaded}
\begin{Highlighting}[]
\BuiltInTok{\textbackslash{}usepackage}\NormalTok{[\textless{}seçenekler\textgreater{}]\{}\ExtensionTok{\textless{}paket adı\textgreater{}}\NormalTok{\}}
\end{Highlighting}
\end{Shaded}

Bu komutla paketin kaynak dosyaya eklenmesi {\TeX} dağıtımıyla
sisteminize kurulmuş olan paketin belgeye çağrılarak işe koşulması
demektir.

\hypertarget{ortamlar}{%
\subsection{Ortamlar}\label{ortamlar}}

{\LaTeX}'de ortamlar önemli bir yer tutar. Örneğin \texttt{document} bir
ortamdır. Ortamları birden fazla ögeye uygulanan komutlar olarak
düşünebiliriz.

Bir ortam \texttt{\textbackslash{}begin} komutuyla başlayıp
\texttt{\textbackslash{}end} komutuyla biter. Her iki komutun zorunlu
değişkeni ortamın adıdır:

\begin{Shaded}
\begin{Highlighting}[]
\KeywordTok{\textbackslash{}begin}\NormalTok{\{}\ErrorTok{\textless{}}\NormalTok{ortam adı\textgreater{}\}}
\NormalTok{ ...}
\KeywordTok{\textbackslash{}end}\NormalTok{\{}\ErrorTok{\textless{}}\NormalTok{ortam adı\textgreater{}\}}
\end{Highlighting}
\end{Shaded}

\hypertarget{gruplar}{%
\subsection{Gruplar}\label{gruplar}}

Gruplar, ortam benzeri yapılardır. Grup
\texttt{\textbackslash{}begingroup} komutuyla başlar ve
\texttt{\textbackslash{}endgroup} komutuyla biter. Grubun içinde
kullanılan bir bildirim sadece gruba uygulanır.

\hypertarget{boux15fluklar}{%
\subsection{Boşluklar}\label{boux15fluklar}}

{\LaTeX}'de belgenizin metnini oluştururken ister klavyedeki Space,
ister Tab tuşu ile boşluk bırakın, bu boşluklar {\LaTeX} tarafından bir
karakter boşluk olarak algılanır. Arka arkaya çok sayıda boşluk
bırakılsa da {\LaTeX} bunu tek bir boşluk olarak algılar.

Bütün bir satırın boş bırakılması {\LaTeX} tarafından paragraf başı
olarak algılanır. Arka arkaya boş bırakılan çok sayıda boş satır
{\LaTeX} tarafından tek bir boş satır yani paragraf başı olarak
algılanır.

\begin{Shaded}
\begin{Highlighting}[]
\BuiltInTok{\textbackslash{}documentclass}\NormalTok{\{}\ExtensionTok{article}\NormalTok{\}}
 \KeywordTok{\textbackslash{}begin}\NormalTok{\{}\ExtensionTok{document}\NormalTok{\}}
\NormalTok{    İster bir boşluk, isterseniz de çok         sayıda boşluk bırakın.}
\NormalTok{    İkisi de bir boşluk gibi işlem görür. Değişen bir şey yok.}

\NormalTok{    Ayrıca boş bir satır yeni paragraf demektir, burada olduğu gibi.}
 \KeywordTok{\textbackslash{}end}\NormalTok{\{}\ExtensionTok{document}\NormalTok{\}}
\end{Highlighting}
\end{Shaded}

Komutlardan sonra gelen boşlukları {\LaTeX} dikkate almaz. Komuttan
sonra gerçekten bir boşluk bırakmak için, ya \texttt{\{\}} ve ardından
boşluk girilir ya da komut adından sonra özel bir boşluk komutu
kullanılır.

\begin{Shaded}
\begin{Highlighting}[]
\BuiltInTok{\textbackslash{}documentclass}\NormalTok{\{}\ExtensionTok{article}\NormalTok{\}}
 \KeywordTok{\textbackslash{}begin}\NormalTok{\{}\ExtensionTok{document}\NormalTok{\}}
    \FunctionTok{\textbackslash{}LaTeX}\NormalTok{  boşluk yok.}\FunctionTok{\textbackslash{}\textbackslash{}}
    \FunctionTok{\textbackslash{}LaTeX}\NormalTok{\{\} boşluk var.}\FunctionTok{\textbackslash{}\textbackslash{}}
    \FunctionTok{\textbackslash{}LaTeX\textbackslash{} }\NormalTok{boşluk komutuyla  boşluk.}
 \KeywordTok{\textbackslash{}end}\NormalTok{\{}\ExtensionTok{document}\NormalTok{\}}
\end{Highlighting}
\end{Shaded}

\hypertarget{yorum-satux131rlarux131} sembolü
kullanılır. {\LaTeX}, bu sembolü gördüğü anda, o satırın geri
kalanındaki her şeyi işlemeden bırakır ve bir sonraki satırın başına
geçip devam eder.

\begin{Shaded}
\begin{Highlighting}[]
\BuiltInTok{\textbackslash{}documentclass}\NormalTok{\{}\ExtensionTok{article}\NormalTok{\}}
 \KeywordTok{\textbackslash{}begin}\NormalTok{\{}\ExtensionTok{document}\NormalTok{\}}
\NormalTok{  Yaz, }\CommentTok{\% sonrası yok.}
\NormalTok{  devamı burada.}
 \KeywordTok{\textbackslash{}end}\NormalTok{\{}\ExtensionTok{document}\NormalTok{\}}
\end{Highlighting}
\end{Shaded}

Eğer daha uzun yorumlar eklemek istenirse
\texttt{\textbackslash{}usepackage\{verbatim\}} komutuyla
\textbf{verbatim} paketini ekledikten sonra yorum comment ortamında
yazılır.

\begin{Shaded}
\begin{Highlighting}[]
\BuiltInTok{\textbackslash{}documentclass}\NormalTok{\{}\ExtensionTok{article}\NormalTok{\}}
\BuiltInTok{\textbackslash{}usepackage}\NormalTok{\{}\ExtensionTok{verbatim}\NormalTok{\}}
 \KeywordTok{\textbackslash{}begin}\NormalTok{\{}\ExtensionTok{document}\NormalTok{\}}
\NormalTok{  Yorum eklemenin başka bir yolu da}
    \KeywordTok{\textbackslash{}begin}\NormalTok{\{}\ExtensionTok{comment}\NormalTok{\}}
\CommentTok{        Buraya uzun yorumlarınızı}
\CommentTok{        ekleyebilirsiniz. Burası}
\CommentTok{        baskıda gözükmeyecektir.}
\CommentTok{    }\KeywordTok{\textbackslash{}end}\NormalTok{\{}\ExtensionTok{comment}\NormalTok{\}}
\NormalTok{ budur.}
 \KeywordTok{\textbackslash{}end}\NormalTok{\{}\ExtensionTok{document}\NormalTok{\}}
\end{Highlighting}
\end{Shaded}

\hypertarget{uxf6zel-amauxe7lux131-karakterler}{%
\subsection{Özel amaçlı
karakterler}\label{uxf6zel-amauxe7lux131-karakterler}}

Aşağıdaki karakterlerin herbiri {\LaTeX}'de özel bir amaç için
kullanılır. Dolayısıyla bu karakterleri doğrudan kullanmak istenmeyen
sonuçlara yol açabilir.

\begin{Shaded}
\begin{Highlighting}[]
\NormalTok{\# }\SpecialStringTok{$ }\CommentTok{\%   \&   \{   \}   \textasciitilde{}  \^{}  \_ \textbackslash{}}
\end{Highlighting}
\end{Shaded}

Bu karakterleri çıktıda elde etmek isterseniz, sondaki hariç, başına bir
geribölü koymanız gerekir. Sondaki için, yani bir geribölü sembolü elde
etmek içinse \texttt{\textbackslash{}textbackslash} komutunu
kullanabilirsiniz. Eğer \texttt{\textbackslash{}\textbackslash{}}
komutunu verirseniz yeni bir satır başlatmış olursunuz.

Örneğin (\texttt{\$}) karakteri matematik kipini açma ve kapatmaya
yarar. (\texttt{\&}) karekteri tablo ve benzeri yapılarda dikey hizalama
yapmak için veya sütun ayracı olarak kullanılır. Çengelli parantezlerden
ve yüzde sembolünden bahsettik. (\texttt{\#}) karakteri yeni komutlar
tanımlamakta kullanılır. Tilda (\texttt{\textasciitilde{}}) ise
genişlemeyen bir boşluk yaratmak için kullanılır. (\texttt{\^{}}) ve
(\texttt{\_}) karakterleri de matematikte üst ve alt indis yazmak için
kullanılır. Her birinin kullanımlarından yeri geldiğinde tekrar
bahsedeceğiz.

\hypertarget{kurulum}{%
\section{Kurulum}\label{kurulum}}

{\LaTeX}'i kurmak için ilk olarak bir {\TeX} dağıtımı edinmeniz gerekir.
Dağıtımlar, dizgi sistemini ve {\LaTeX}'de belge oluşturabilmek için
gereken paketleri içerir.

İkinci ihtiyaç duyacağınız şey bir {\LaTeX} editörüdür. Edindiğiniz
{\TeX} dağıtımları genelde bir {\LaTeX} editörüyle birlikte gelir. Tabi
editör kişisel bir tercihtir ve bir {\LaTeX} editörü yerine basit bir
metin editörü kullanabilirsiniz. Ancak farklı işletim sistemleri için
birçok iyi {\LaTeX} editörü vardır ve bunların kod vurgulama, otomatik
tamamlama, otomatik belge oluşturma gibi {\LaTeX}'e özgü işlevleri
vardır. Dolayısıyla {\LaTeX}'de yeniyseniz bir editör kullanmanızı
tavsiye ederiz.

\hypertarget{gnulinux}{%
\subsection{GNU/Linux}\label{gnulinux}}

Linux sistemlere \href{https://miktex.org/download}{MiKTeX} ya da
\href{http://www.tug.org/texlive/}{TeX Live} kurulabilir. MiKTeX'in
indirme sayfasında Ubuntu, Mint, Debian, Fedora, CentOS ve openSUSE gibi
Linux dağıtımlarında nasıl kurulacağı anlatılmıştır. TeX Live ise tüm
popüler Linux dağıtımlarının depolarında mevcut olup, paket yöneticisi
ya da komut satırı yardımıyla kurulabilir. Örneğin Ubuntu, Debian, Mint,
Pardus gibi \texttt{.deb} uzantılı paketlerin kullanıldığı dağıtımlarda

\begin{Shaded}
\begin{Highlighting}[]
\FunctionTok{sudo}\NormalTok{ apt{-}get install texlive{-}base}
\end{Highlighting}
\end{Shaded}

komutuyla temel kurulum,

\begin{Shaded}
\begin{Highlighting}[]
\FunctionTok{sudo}\NormalTok{ apt{-}get install texlive{-}full}
\end{Highlighting}
\end{Shaded}

komutuyla da tam kurulum yapılır.

\hypertarget{mac-os}{%
\subsection{Mac OS}\label{mac-os}}

Mac OS kullanıcıları için iki seçenek mevcuttur:
\href{https://miktex.org/download}{MiKTeX} ya da
\href{http://www.tug.org/mactex/}{MacTeX}. MiKTeX kurulumu için
\texttt{.dmg} uzantılı, MacTeX içinse \texttt{.pkg} uzantılı dosya
indirilir ve standart kurulum yapılır.

\hypertarget{windows}{%
\subsection{Windows}\label{windows}}

Windows için aşağıdaki dağıtımlardan birini kurabilirsiniz.

\begin{itemize}
\tightlist
\item
  \href{https://miktex.org/download}{MiKTeX}
\item
  \href{http://www.tug.org/texlive/}{TeX Live}
\item
  \href{https://tug.org/protext/}{proTeXt}
\end{itemize}

MiKTeX veya TeX Live dağıtımını kurarsanız sisteminize
\href{https://www.tug.org/texworks/}{TeXworks} editörü de kurulur.
proTeXt dağıtımı MiKTeX tabanlı bir dağıtım olup, tüm paketleri içerir
ve beraberinde \href{https://texstudio.org/}{TeXstudio} editörüyle
gelir.

\hypertarget{latex-edituxf6rleri}{%
\subsection{\texorpdfstring{{\LaTeX}
editörleri}{ editörleri}}\label{latex-edituxf6rleri}}

Hangi editörü kullanacağınıza birkaç deneme yaptıktan sonra karar
verebilirsiniz. \href{https://beebom.com/best-latex-editors/}{Burada} en
çok beğenilen editörler listelenmiş.

Her {\LaTeX} editöründe olan özelliklerin (otomatik kod tamamlama vb.)
yanı sıra kullanıcı dostu arayüzü, yüzde yüze yakın Türkçe desteği,
ücretsiz oluşu ve her üç sistemde de çalışabilmesinden dolayı
\href{https://texstudio.org/}{TeXstudio}'yu tavsiye ediyoruz. Karar
sizin.

\hypertarget{uxe7evrimiuxe7i-edituxf6rler}{%
\subsection{Çevrimiçi editörler}\label{uxe7evrimiuxe7i-edituxf6rler}}

{\LaTeX}'i hiçbir kurulum yapmadan çevrimiçi de kullanabilirsiniz.
Aşağıda üç tanesi listelenmiştir.

\begin{itemize}
\tightlist
\item
  \href{https://www.overleaf.com/}{Overleaf}
\item
  \href{https://papeeria.com/}{Papeeria}
\item
  \href{https://latexbase.com/}{{\LaTeX} Base}
\end{itemize}

En popüler olanı Overleaf olup, sayfasında beğenebileceğiniz binlerce
\href{https://www.overleaf.com/latex/templates}{şablon} ve {\LaTeX}
kullanımına yönelik \href{https://www.overleaf.com/learn}{anlatımlar}
bulunur.

\bookmarksetup{startatroot}

\hypertarget{ilk-adux131mlar}{%
\chapter{İlk Adımlar}\label{ilk-adux131mlar}}

\hypertarget{tipik-bir-belge-yazux131mux131}{%
\section{Tipik Bir Belge Yazımı}\label{tipik-bir-belge-yazux131mux131}}

{\LaTeX}'in varsayılan dosya uzantısı \texttt{.tex}'tir. Bu basit bir
metin dosyası olup, {\LaTeX} editörleriyle oluşturulup düzenlenebileceği
gibi basit bir metin editörüyle de düzenlenebilir.

Bir belge hazırlamaya başlamak için verilecek ilk komut
\texttt{\textbackslash{}documentclass{[}...{]}\{...\}} olup, çengelli
parantezler arasına oluşturmak istediğiniz belgenin sınıfı yazılır.
Köşeli parantezlerin içine de isteğe bağlı bazı değişkenler yazılabilir.
Eğer bu kısım boş bırakılırsa {\LaTeX} varsayılan değerleri alacaktır.
Bu komutun ardından sırasıyla \texttt{\textbackslash{}begin\{document\}}
ve \texttt{\textbackslash{}end\{document\}} komutları verilerek belge
ortamı oluşturulur. \texttt{\textbackslash{}end\{document\}} komutuyla
{\LaTeX}'e belgenin bittiği söylenmiş olur ve {\LaTeX} bu komuttan sonra
girilenleri dikkate almaz.

\texttt{\textbackslash{}documentclass} komutuyla
\texttt{\textbackslash{}begin\{document\}} komutu arasına
\emph{sahanlık} denir. Sahanlık, belgenin ayarlarının yapıldığı kısımdır
ve bu kısım çıktıda görünmez. \texttt{\textbackslash{}begin\{document\}}
ile \texttt{\textbackslash{}end\{document\}} arasına da \emph{gövde}
denir. İçerik burada oluşturulur.

Aşağıda asgari bir {\LaTeX} kaynak dosyası gösterilmiştir.
\texttt{\textbackslash{}documentclass} komutunun değişkeni olan
\texttt{article}, belgenin makale olacağını belirtir.

\begin{Shaded}
\begin{Highlighting}[]
\BuiltInTok{\textbackslash{}documentclass}\NormalTok{\{}\ExtensionTok{article}\NormalTok{\}}
\KeywordTok{\textbackslash{}begin}\NormalTok{\{}\ExtensionTok{document}\NormalTok{\}}
\NormalTok{  İşte ilk belgem.}
\KeywordTok{\textbackslash{}end}\NormalTok{\{}\ExtensionTok{document}\NormalTok{\}}
\end{Highlighting}
\end{Shaded}

Bu noktadan sonra örnek kaynak dosyayı {\LaTeX} editörünüzünde oluşturup
önceden oluşturduğunuz bir dizine kaydedin. Kaydederken dosya adında
boşluk ve Türkçe karakter kullanmayın. Örneğin kaynak dosyanız
\texttt{belge1.tex} olsun.

İkinci aşama kaynak dosyanın derlenmesidir. Derleme işlemi için {\LaTeX}
editörlerinde genelde araç çubuğunda oklar bulunur. Oka tıklandığında
dosya derlenir ve sonuç, çıktı ekranında görünür.

Eğer metin editörü kullanıyorsanız derlemeyi uçbirimde (terminal,
konsol,\ldots) yapmanız gerekir. Derleme için uçbirim kaynak dosyanın
olduğu dizinde açılıp

\begin{Shaded}
\begin{Highlighting}[]
\ExtensionTok{pdflatex}\NormalTok{ belge1}
\end{Highlighting}
\end{Shaded}

komutu verilmelidir.

Derleme işleminden sonra kaynak dosyanızın olduğu dizinde
\texttt{belge1.tex} ve \texttt{belge1.pdf} dosyalarının yanında yine
\texttt{belge1} ile başlayan farklı uzantılara sahip dosyalar olacaktır.
Bu dosyaların ne olduklarına ilerleyen yazılarda değinilecektir ancak
dileyen okur Oetiker et al. (2006)'e bakabilir.

\hypertarget{ch-belgesinifi}{%
\section{Belge Sınıfları ve Seçenekleri}\label{ch-belgesinifi}}

Başka sınıflar olmakla birlikte {\LaTeX}'de varsayılan olarak kullanılan
beş belge sınıfı vardır (Tablo~\ref{tbl-belgesin}).

\hypertarget{tbl-belgesin}{}
\begin{longtable}[]{@{}
  >{\raggedright\arraybackslash}p{(\columnwidth - 2\tabcolsep) * \real{0.5000}}
  >{\raggedright\arraybackslash}p{(\columnwidth - 2\tabcolsep) * \real{0.5000}}@{}}
\caption{\label{tbl-belgesin}{\LaTeX}'de Belge Sınıfları}\tabularnewline
\toprule()
\begin{minipage}[b]{\linewidth}\raggedright
\textbf{Sınıf}
\end{minipage} & \begin{minipage}[b]{\linewidth}\raggedright
\textbf{Açıklama}
\end{minipage} \\
\midrule()
\endfirsthead
\toprule()
\begin{minipage}[b]{\linewidth}\raggedright
\textbf{Sınıf}
\end{minipage} & \begin{minipage}[b]{\linewidth}\raggedright
\textbf{Açıklama}
\end{minipage} \\
\midrule()
\endhead
\texttt{article} & Makale \\
\texttt{report} & Makaleden daha hacimli belgeler için kullanılır.
Rapor, tez gibi \\
\texttt{book} & Kitap \\
\texttt{letter} & Mektup \\
\texttt{beamer} & Sunu \\
\bottomrule()
\end{longtable}

Bu beş sınıftan \texttt{article}, \texttt{report} ve \texttt{book} için
kullanılabilecek seçenekler Tablo~\ref{tbl-belgesec}'de gösterilmiştir.

\hypertarget{tbl-belgesec}{}
\begin{longtable}[]{@{}
  >{\raggedright\arraybackslash}p{(\columnwidth - 2\tabcolsep) * \real{0.5000}}
  >{\raggedright\arraybackslash}p{(\columnwidth - 2\tabcolsep) * \real{0.5000}}@{}}
\caption{\label{tbl-belgesec}{\LaTeX}'de Belge
Seçenekleri}\tabularnewline
\toprule()
\begin{minipage}[b]{\linewidth}\raggedright
\textbf{Seçenek}
\end{minipage} & \begin{minipage}[b]{\linewidth}\raggedright
\textbf{Açıklama}
\end{minipage} \\
\midrule()
\endfirsthead
\toprule()
\begin{minipage}[b]{\linewidth}\raggedright
\textbf{Seçenek}
\end{minipage} & \begin{minipage}[b]{\linewidth}\raggedright
\textbf{Açıklama}
\end{minipage} \\
\midrule()
\endhead
\textbf{10pt, 11pt, 12pt} & Belge ana yazı büyüklüğü. \\
\textbf{a4paper, a5paper, letterpaper,\ldots{}} & Kağıt boyutu. \\
\textbf{fleqn} & Formülleri ortada yazmak yerine, sola bitişik yazar. \\
\textbf{leqno} & Formül numaralarını sağ yerine sol tarafa koyar. \\
\textbf{titlepage, notitlepage} & Belge başlığını attıktan sonra yeni
bir sayfa açıp açmayacağını belirler. \\
\textbf{onecolumn, twocolumn} & Belgenin tek sütun veya çift sütun
dizileceğini belirtir. \\
\textbf{twoside, oneside} & Belgenin kağıdın hep tek tarafına mı yoksa
iki tarafına mı basılacağını belirtir. \\
\textbf{landscape} & Belgeyi enine tutulmuş kağıda basılmak üzere
hazırlar. \\
\textbf{openright, openany} & Belgede bölümleri hep sağ sayfalardan veya
ilk gelen boş sayfadan başlatır. \\
\textbf{draft, final} & Belgeyi sırasıyla \emph{taslak} ve \emph{son}
şeklinde hazırlar. \textbf{draft} seçilirse, sağ taraftan fırlamış olan
satırlar kalın siyah bir çizgiyle işaretlenir. \\
\bottomrule()
\end{longtable}

Bu seçeneklerin her birinin kullanılabilirliği belge sınıfına göre
farklılık gösterir. Tablo~\ref{tbl-seckul}'de hangi seçeneğin hangi
sınıf için varsayılan olduğu ve kullanılabilir olup olmadığı
gösterilmiştir.

\hypertarget{tbl-seckul}{}
\begin{longtable}[]{@{}
  >{\raggedright\arraybackslash}p{(\columnwidth - 6\tabcolsep) * \real{0.2500}}
  >{\raggedright\arraybackslash}p{(\columnwidth - 6\tabcolsep) * \real{0.2500}}
  >{\raggedright\arraybackslash}p{(\columnwidth - 6\tabcolsep) * \real{0.2500}}
  >{\raggedright\arraybackslash}p{(\columnwidth - 6\tabcolsep) * \real{0.2500}}@{}}
\caption{\label{tbl-seckul}Seçeneklerin belge sınıflarına göre
kullanılabilirliği (1: varsayılan 1/2: kullanılabilir 0:
kullanılamaz)}\tabularnewline
\toprule()
\begin{minipage}[b]{\linewidth}\raggedright
\textbf{Seçenek}
\end{minipage} & \begin{minipage}[b]{\linewidth}\raggedright
\texttt{book}
\end{minipage} & \begin{minipage}[b]{\linewidth}\raggedright
\texttt{report}
\end{minipage} & \begin{minipage}[b]{\linewidth}\raggedright
\texttt{article}
\end{minipage} \\
\midrule()
\endfirsthead
\toprule()
\begin{minipage}[b]{\linewidth}\raggedright
\textbf{Seçenek}
\end{minipage} & \begin{minipage}[b]{\linewidth}\raggedright
\texttt{book}
\end{minipage} & \begin{minipage}[b]{\linewidth}\raggedright
\texttt{report}
\end{minipage} & \begin{minipage}[b]{\linewidth}\raggedright
\texttt{article}
\end{minipage} \\
\midrule()
\endhead
\textbf{10pt} & 1 & 1 & 1 \\
\textbf{letterpaper} & 1 & 1 & 1 \\
\textbf{oneside} & 1/2 & 1 & 1 \\
\textbf{twoside} & 1 & 1/2 & 1/2 \\
\textbf{openany} & 1/2 & 1 & 0 \\
\textbf{openright} & 1 & 1/2 & 0 \\
\textbf{titlepage} & 1 & 1 & 1/2 \\
\textbf{final} & 1 & 1 & 1 \\
\bottomrule()
\end{longtable}

Örneğin belgeye
\texttt{\textbackslash{}documentclass{[}a4paper,12pt{]}\{article\}}
komutuyla başlarsak {\LaTeX}'e kağıt boyutu A4, ana yazı büyüklüğü 12
punto olan bir makale yazacağımızı bildirmiş oluruz. Başka bir örnek
\texttt{\textbackslash{}documentclass{[}a5paper,11pt,twocolumn{]}\{book\}}
olsun. Bu örnekte kağıt boyutu A5, ana yazı büyüklüğü 11 punto olan bir
kitap yazacağımızı ve kitabın iki sütun olarak dizilmesini söyledik.

\hypertarget{tuxfcrkuxe7e-dil-ayarlarux131-ve-uxe7oklu-dil-kullanux131mux131}{%
\section{Türkçe Dil Ayarları ve Çoklu Dil
Kullanımı}\label{tuxfcrkuxe7e-dil-ayarlarux131-ve-uxe7oklu-dil-kullanux131mux131}}

{\LaTeX}'de Türkçe belgeler oluşturmak için öncelikle sahanlığa

\begin{Shaded}
\begin{Highlighting}[]
\BuiltInTok{\textbackslash{}usepackage}\NormalTok{[T1]\{}\ExtensionTok{fontenc}\NormalTok{\}}
\BuiltInTok{\textbackslash{}usepackage}\NormalTok{[turkish]\{}\ExtensionTok{babel}\NormalTok{\}}
\end{Highlighting}
\end{Shaded}

komutlarının verilmesi gerekir.

\texttt{T1} seçenekli \textbf{fontenc} paketi yazıtipi kodlamasıyla
ilgili bir paket olup, hecelemenin doğru şekilde yapılmasını sağlar. Bir
çok Avrupa dilinde de \texttt{T1} seçeneğiyle kullanılır.
\texttt{turkish} seçenekli \textbf{babel} paketi de Chapter, Table,
Contents,\ldots{} gibi isimlerin Türkçeleşmesi (Bölüm, Tablo,
İçindekiler,\ldots) içindir.

\begin{tcolorbox}[enhanced jigsaw, opacitybacktitle=0.6, coltitle=black, leftrule=.75mm, rightrule=.15mm, toprule=.15mm, bottomtitle=1mm, titlerule=0mm, colbacktitle=quarto-callout-note-color!10!white, breakable, arc=.35mm, opacityback=0, colframe=quarto-callout-note-color-frame, toptitle=1mm, title=\textcolor{quarto-callout-note-color}{\faInfo}\hspace{0.5em}{Not}, bottomrule=.15mm, left=2mm, colback=white]
Yakın zamana kadar ö, ş, ç,\dots{} gibi Türkçe karakterlerin
kullanılabilmesi için sahanlığa
\texttt{\textbackslash{}usepackage{[}utf8{]}\{inputenc\}} ya da
\texttt{\textbackslash{}usepackage{[}latin5{]}\{inputenc\}}
komutlarından birinin verilmesi gerekiyordu. Bu paket
(\textbf{inputenc}) girdi kodlamasını yöneten bir pakettir. Son
güncellemelerle birlikte bu paketin kullanılma zorunluluğu ortadan
kalkmıştır.
\end{tcolorbox}

Aşağıda Türkçe asgari bir {\LaTeX} kaynak dosyası örneği verilmiştir.

\begin{Shaded}
\begin{Highlighting}[]
\BuiltInTok{\textbackslash{}documentclass}\NormalTok{\{}\ExtensionTok{article}\NormalTok{\}}
\BuiltInTok{\textbackslash{}usepackage}\NormalTok{[T1]\{}\ExtensionTok{fontenc}\NormalTok{\}}
\BuiltInTok{\textbackslash{}usepackage}\NormalTok{[turkish]\{}\ExtensionTok{babel}\NormalTok{\}}
\KeywordTok{\textbackslash{}begin}\NormalTok{\{}\ExtensionTok{document}\NormalTok{\}}
\NormalTok{  İşte  Türkçe ilk belgem.}
\KeywordTok{\textbackslash{}end}\NormalTok{\{}\ExtensionTok{document}\NormalTok{\}}
\end{Highlighting}
\end{Shaded}

Türkçe dışında ikinci bir dil kullanmak isterseniz, örneğin İngilizce,
\textbf{babel} paketinin seçeneğini

\begin{Shaded}
\begin{Highlighting}[]
\BuiltInTok{\textbackslash{}usepackage}\NormalTok{[english,turkish]\{}\ExtensionTok{babel}\NormalTok{\}}
\end{Highlighting}
\end{Shaded}

şeklinde değiştirmeniz gerekir. Burada etkin olan dil Türkçedir.
İngilizceyi etkin hale getirmek için
\texttt{\textbackslash{}selectlanguage\{english\}} komutu kullanılır.
Tekrar Türkçeye geçmek için de benzer şekilde
\texttt{\textbackslash{}selectlanguage\{turkish\}} komutu kullanılır.

Bir kelime ya da cümle gibi kısa metinler kullanılacaksa
\texttt{\textbackslash{}foreignlanguage} komutu kullanılabilir:

\begin{Shaded}
\begin{Highlighting}[]
\FunctionTok{\textbackslash{}foreignlanguage}\NormalTok{\{\textless{}dil\textgreater{}\}\{\textless{}metin\textgreater{}\}}
\end{Highlighting}
\end{Shaded}

Uzun metinler içinse diğer bir seçenek \texttt{otherlanguage} ortamıdır.

\begin{Shaded}
\begin{Highlighting}[]
\KeywordTok{\textbackslash{}begin}\NormalTok{\{}\ExtensionTok{otherlanguage}\NormalTok{\}\{\textless{}dil\textgreater{}\}}
\NormalTok{  ...}
\KeywordTok{\textbackslash{}end}\NormalTok{\{}\ExtensionTok{otherlanguage}\NormalTok{\}}
\end{Highlighting}
\end{Shaded}

Bu ortamın isimleri değiştirmeyen, örneğin, dil seçeneği İngilizce
olmasına rağmen belgeye bir tablo eklediğinizde ``Table'' yerine yine
``Tablo'' adını yazan yıldızlı sürümü de (\texttt{otherlanguage*})
vardır.

\hypertarget{belgeye-baux15flux131k-oluux15fturma}{%
\section{Belgeye Başlık
Oluşturma}\label{belgeye-baux15flux131k-oluux15fturma}}

{\LaTeX}'de belgeye başlık oluşturmak için
\texttt{\textbackslash{}title} komutu kullanılır. Yazar adı
\texttt{\textbackslash{}author} komutuyla girilir. Birden fazla yazar
varsa yazar adları arasına \texttt{\textbackslash{}and} komutu girilir.

İsteğe bağlı olarak tarih için \texttt{\textbackslash{}date} komutu
kullanılır. Eğer \texttt{\textbackslash{}date} komutu kullanılmazsa
{\LaTeX} belgenizi derlediğiniz günün tarihini basar. Tarihin
basılmasını istemiyorsanız, bu komutu tarih yazılmadan
\texttt{\textbackslash{}date\{\}} şeklinde kullanmanız gerekir.

Son olarak, başlığın belgenize yazılması için
\texttt{\textbackslash{}begin\{document\}} komutundan sonra başlığı
oluşturmak istediğiniz yere \texttt{\textbackslash{}maketitle} komutunu
girersiniz. Belge başlığını attıktan sonra yeni bir sayfanın açılıp
açılmayacağı belgenin sınıfına bağlı olarak belirlenir.

Ayrıca \texttt{\textbackslash{}title}, \texttt{\textbackslash{}author}
ve \texttt{\textbackslash{}date} komutları
\texttt{\textbackslash{}thanks} komutunu içerebilir. Bu komutun
değişkeni bir e-posta adresi, iş adresi veya bir teşekkür metni
olabilir.

\begin{Shaded}
\begin{Highlighting}[]
\BuiltInTok{\textbackslash{}documentclass}\NormalTok{[a4paper,12pt]\{}\ExtensionTok{article}\NormalTok{\}}
\BuiltInTok{\textbackslash{}usepackage}\NormalTok{[T1]\{}\ExtensionTok{fontenc}\NormalTok{\}}
\BuiltInTok{\textbackslash{}usepackage}\NormalTok{[turkish]\{}\ExtensionTok{babel}\NormalTok{\}}
\FunctionTok{\textbackslash{}title}\NormalTok{\{Belge Başlığı\}}
\FunctionTok{\textbackslash{}author}\NormalTok{\{Yazar 1}\FunctionTok{\textbackslash{}thanks}\NormalTok{\{A Üniversitesi\} }\FunctionTok{\textbackslash{}and}\NormalTok{ Yazar 2}\FunctionTok{\textbackslash{}thanks}\NormalTok{\{B Üniversitesi\}\}}
\FunctionTok{\textbackslash{}date}\NormalTok{\{XX.XX.XXXX\}}
\KeywordTok{\textbackslash{}begin}\NormalTok{\{}\ExtensionTok{document}\NormalTok{\}}
\FunctionTok{\textbackslash{}maketitle}
\NormalTok{  İçerik...}
\KeywordTok{\textbackslash{}end}\NormalTok{\{}\ExtensionTok{document}\NormalTok{\}}
\end{Highlighting}
\end{Shaded}

\hypertarget{buxf6luxfcmleme-ve-iuxe7indekiler-tablosu}{%
\section{Bölümleme ve İçindekiler
Tablosu}\label{buxf6luxfcmleme-ve-iuxe7indekiler-tablosu}}

{\LaTeX}'de belgenizi bölümlere ayırmak için 7 seviye bulunmaktadır.

\hypertarget{tbl-bolumseviyeleri}{}
\begin{longtable}[]{@{}lll@{}}
\caption{\label{tbl-bolumseviyeleri}{\LaTeX}'de Bölüm
Seviyeleri}\tabularnewline
\toprule()
\textbf{Komut} & \textbf{Seviye} & \textbf{Açıklama} \\
\midrule()
\endfirsthead
\toprule()
\textbf{Komut} & \textbf{Seviye} & \textbf{Açıklama} \\
\midrule()
\endhead
\texttt{\textbackslash{}part\{...\}} & -1( book ve report ) 0 ( article)
& letter hariç \\
\texttt{\textbackslash{}chapter\{...\}} & 0 & sadece book ve report \\
\texttt{\textbackslash{}section\{...\}} & 1 & letter hariç \\
\texttt{\textbackslash{}subsection\{...\}} & 2 & letter hariç \\
\texttt{\textbackslash{}subsubsection\{...\}} & 3 & letter hariç \\
\texttt{\textbackslash{}paragraph\{...\}} & 4 & letter hariç \\
\texttt{\textbackslash{}subparagraph\{...\}} & 5 & letter hariç \\
\bottomrule()
\end{longtable}

Türkçe dil paketi ekli belgelerde \texttt{\textbackslash{}part} komutu
``Kısım'', \texttt{\textbackslash{}chapter} komutu ``Bölüm'' olarak
yazılır. Kısımlar I, II, III,\ldots{} şeklinde bölümler ise 1, 2,
3,\ldots{} şeklinde numaralandırılır. \texttt{\textbackslash{}section}
komutu \texttt{book} ve \texttt{report} sınıflarında
\texttt{\textbackslash{}chapter} komutunu takip ederek 1.1, 1.2,\ldots{}
diğer sınıflarda 1, 2, 3,\ldots{} şeklinde numaralandırılır.
\texttt{\textbackslash{}subsection} komutu da
\texttt{\textbackslash{}section} komutunu takip ederek numaralandırılır.

Türkçe dil paketi ekli belgelerde \texttt{\textbackslash{}part} komutu
``Kısım'', \texttt{\textbackslash{}chapter} komutu ``Bölüm'' olarak
yazılır. Kısımlar I, II, III,\ldots{} şeklinde bölümler ise 1, 2,
3,\ldots{} şeklinde numaralandırılır. \texttt{\textbackslash{}section}
komutu \texttt{book} ve \texttt{report} sınıflarında
\texttt{\textbackslash{}chapter} komutunu takip ederek 1.1, 1.2,\ldots{}
diğer sınıflarda 1, 2, 3,\ldots{} şeklinde numaralandırılır.
\texttt{\textbackslash{}subsection} komutu da
\texttt{\textbackslash{}section} komutunu takip ederek numaralandırılır.

İçindekiler tablosu için {\LaTeX}'e
\texttt{\textbackslash{}tableofcontents} komutu verilir. Bu komutun
yazıldığı yerde İçindekiler tablosu oluşturulur. İçindekiler tablosunun
doğru dizilmesi için kaynak dosyanızı en az iki kere derlemeniz gerekir.

{\LaTeX}'de \texttt{article} sınıfında 4 ve 5'inci seviye başlıklara,
\texttt{book} ve \texttt{report} sınıflarında ise bunlara ek 3'üncü
seviye başlıklara numara verilmez ve numara verilmeyen başlıklar
İçindekiler tablosuna yazılmaz. Bu seviyelerdeki başlıklara numara
verilmesini ve İçindekiler tablosuna yazılması için iki adet
\texttt{\textbackslash{}setcounter} komutu

\begin{Shaded}
\begin{Highlighting}[]
\FunctionTok{\textbackslash{}setcounter}\NormalTok{\{secnumdepth\}\{\textless{}seviye\textgreater{}\}}
\FunctionTok{\textbackslash{}setcounter}\NormalTok{\{tocdepth\}\{\textless{}seviye\textgreater{}\}}
\end{Highlighting}
\end{Shaded}

şeklinde kullanılır. Birinci komuttaki
\texttt{\textless{}seviye\textgreater{}} değişkeninde kaçıncı seviyeye
kadar olan başlıkların numaralandırılacağını, ikinci komuttaki
\texttt{\textless{}seviye\textgreater{}} değişkeninde de kaçıncı
seviyeye kadar olan başlıkların İçindekiler tablosuna yazılacağını
sayıyla belirtirsiniz. Örneğin \texttt{book} ve \texttt{report}
sınıflarında

\begin{Shaded}
\begin{Highlighting}[]
\FunctionTok{\textbackslash{}setcounter}\NormalTok{\{secnumdepth\}\{3\}}
\FunctionTok{\textbackslash{}setcounter}\NormalTok{\{tocdepth\}\{3\}}
\end{Highlighting}
\end{Shaded}

komutlarıyla \texttt{\textbackslash{}subsubsection} komutuna kadar olan
başlıklara hem numara verir hem de İçindekiler tablosuna yazdırırsınız.
Komutların çalışması için ya sahanlıkta ya da
\texttt{\textbackslash{}tableofcontents} komutundan önce verilmelidir.

Uzun başlıkların İçindekiler tablosunda daha kısa yazılması istenirse
bölüm komutlarının zorunlu olmayan değişkenine başlıkların kısa şekli
yazılır:

\begin{Shaded}
\begin{Highlighting}[]
\KeywordTok{\textbackslash{}section}\NormalTok{[Kısa Başlık]\{Uzuuuuuuuuuuuuuuuuun Başlık\}}
\end{Highlighting}
\end{Shaded}

Bölüm komutlarının birde yıldızlı sürümleri vardır:

\begin{Shaded}
\begin{Highlighting}[]
\KeywordTok{\textbackslash{}part*}\NormalTok{\{...\}}
\KeywordTok{\textbackslash{}chapter*}\NormalTok{\{...\}}
\KeywordTok{\textbackslash{}section*}\NormalTok{\{...\}}
\KeywordTok{\textbackslash{}subsection*}\NormalTok{\{...\}}
\KeywordTok{\textbackslash{}subsubsection*}\NormalTok{\{...\}}
\KeywordTok{\textbackslash{}paragraph*}\NormalTok{\{...\}}
\KeywordTok{\textbackslash{}subparagraph*}\NormalTok{\{...\}}
\end{Highlighting}
\end{Shaded}

Komutlar bu şekilde verildiğinde başlığa numara verilmez ve İçindekiler
tablosuna yazılmaz.

İçindekiler tablosunu {\LaTeX} otomatik oluştursa da elle eklemeler
yapılabilir, hatta Kaynakça gibi özel sayfalarda bu eklemeler
gereklidir. Bunun için \texttt{\textbackslash{}addcontentsline} komutu
kullanılır.

\begin{Shaded}
\begin{Highlighting}[]
\FunctionTok{\textbackslash{}addcontentsline}\NormalTok{\{toc\}\{\textless{}giriş formatı\textgreater{}\}\{\textless{}giriş metni\textgreater{}\}}
\end{Highlighting}
\end{Shaded}

Burada \texttt{toc}, bilginin yazılacağı İçindekiler tablosunun dosya
uzantısıdır. Bütünleşik olarak oluşturduğunuz kaynakçanın İçindekiler
tablosuna yazılması için
\texttt{\textbackslash{}begin\{thebibliography\}} komutunun peşine
\texttt{book} ve \texttt{report} sınıflarında

\begin{Shaded}
\begin{Highlighting}[]
\FunctionTok{\textbackslash{}addcontentsline}\NormalTok{\{toc\}\{chapter\}\{Kaynakça\}}
\end{Highlighting}
\end{Shaded}

\texttt{article} sınıfında ise

\begin{Shaded}
\begin{Highlighting}[]
\FunctionTok{\textbackslash{}addcontentsline}\NormalTok{\{toc\}\{section\}\{Kaynaklar\}}
\end{Highlighting}
\end{Shaded}

komutunun verilmesi gerekir.

\begin{tcolorbox}[enhanced jigsaw, opacitybacktitle=0.6, coltitle=black, leftrule=.75mm, rightrule=.15mm, toprule=.15mm, bottomtitle=1mm, titlerule=0mm, colbacktitle=quarto-callout-note-color!10!white, breakable, arc=.35mm, opacityback=0, colframe=quarto-callout-note-color-frame, toptitle=1mm, title=\textcolor{quarto-callout-note-color}{\faInfo}\hspace{0.5em}{Not}, bottomrule=.15mm, left=2mm, colback=white]
``Kaynakça'' ya da ``Kaynaklar'' isimleri yerine farklı isimler
kullanılabilir elbette. Ancak \texttt{thebibliography} ortamının
oluşturulduğu yerlerde {\LaTeX} bu isimleri yazdıracağından tutarlı
olması açısından bu isimler önerilmiştir.
\end{tcolorbox}

\begin{Shaded}
\begin{Highlighting}[]
\BuiltInTok{\textbackslash{}documentclass}\NormalTok{[a4paper,12pt]\{}\ExtensionTok{article}\NormalTok{\}}
\BuiltInTok{\textbackslash{}usepackage}\NormalTok{[T1]\{}\ExtensionTok{fontenc}\NormalTok{\}}
\BuiltInTok{\textbackslash{}usepackage}\NormalTok{[turkish]\{}\ExtensionTok{babel}\NormalTok{\}}
\FunctionTok{\textbackslash{}title}\NormalTok{\{}\FunctionTok{\textbackslash{}LaTeX}\NormalTok{\textquotesingle{}de Bölümlendirme  ve İçindekiler Tablosu Oluşturma\}}
\FunctionTok{\textbackslash{}author}\NormalTok{\{Zafer Acar\}}
\KeywordTok{\textbackslash{}begin}\NormalTok{\{}\ExtensionTok{document}\NormalTok{\}}
\FunctionTok{\textbackslash{}maketitle}
\FunctionTok{\textbackslash{}tableofcontents}
\KeywordTok{\textbackslash{}section}\NormalTok{\{Birinci Seviye Başlık\}}
\NormalTok{  İçerik...}
\KeywordTok{\textbackslash{}subsection}\NormalTok{\{İkinci Seviye Başlık\}}
\NormalTok{  İçerik...}
\KeywordTok{\textbackslash{}subsubsection}\NormalTok{\{Üçüncü Seviye Başlık\}}
\NormalTok{  İçerik...}
\KeywordTok{\textbackslash{}section}\NormalTok{[Kısa Başlık]\{Uzuuuuuuuuun Başlık\}}
\NormalTok{  İçerik...}
\KeywordTok{\textbackslash{}section*}\NormalTok{\{Numarasız Başlık\}}
\FunctionTok{\textbackslash{}addcontentsline}\NormalTok{\{toc\}\{section\}\{Numarasız Başlık\}}
\NormalTok{  İçerik...}
\KeywordTok{\textbackslash{}end}\NormalTok{\{}\ExtensionTok{document}\NormalTok{\}}
\end{Highlighting}
\end{Shaded}

\hypertarget{buxfcyuxfck-projeler}{%
\section{Büyük Projeler}\label{buxfcyuxfck-projeler}}

{\LaTeX}'de kitap yazmaya başlamak için belge sınıfı \texttt{book}
seçilir. Bunun dışında kitapların \emph{Baş}, \emph{Gövde} ve \emph{Son}
kısımları olur. Bu kısımların nerede başlayıp nerede bittikleri
aşağıdaki komutlarla {\LaTeX}'e bildirilir:

\begin{itemize}
\tightlist
\item
  \texttt{\textbackslash{}frontmatter} (baş)
  komutu\texttt{\textbackslash{}begin\{document\}} komutundan hemen
  sonra verilir. Bu komut, baş taraftaki İçindekiler, Önsöz gibi
  kısımların sayfa numaralandırmasını Roma rakamıyla yapar. Ayrıca bu
  kısımda bölüm komutları (\texttt{*}) işareti olmadan verildiğinde
  (örneğin \texttt{\textbackslash{}chapter\{Önsöz\}}) bunlara numara
  verilmez ancak İçindekiler tablosuna yazılırlar.
\item
  \texttt{\textbackslash{}mainmatter} (gövde) komutu kitabın ilk bölüm
  başlığından hemen önce verilmelidir. Buradan itibaren sayfa
  numaralandırmasını yeniden başlatıp Arap rakamlarına geçer.
\item
  \texttt{\textbackslash{}appendix} (ekler) komutu kitabınızın
  eklerindeki bölümleri harflerle numaralandırır (Ek A, Ek B, \ldots{}
  ).
\item
  \texttt{\textbackslash{}backmatter} (son) komutu kitabınızda her şey
  bittikten sonra verilir fakat bilinen belge sınıflarında görünürde
  hiçbir etkisi yoktur.
\end{itemize}

\begin{Shaded}
\begin{Highlighting}[]
\BuiltInTok{\textbackslash{}documentclass}\NormalTok{[a4paper,12pt]\{}\ExtensionTok{book}\NormalTok{\}}
\BuiltInTok{\textbackslash{}usepackage}\NormalTok{[T1]\{}\ExtensionTok{fontenc}\NormalTok{\}}
\BuiltInTok{\textbackslash{}usepackage}\NormalTok{[turkish]\{}\ExtensionTok{babel}\NormalTok{\}}
\FunctionTok{\textbackslash{}title}\NormalTok{\{Başlık\}}\CommentTok{\%Başlık}
\FunctionTok{\textbackslash{}author}\NormalTok{\{Yazar Adı\}}\CommentTok{\%yazar}
\FunctionTok{\textbackslash{}date}\NormalTok{\{Eylül 2022\}}\CommentTok{\%Tarih}

\KeywordTok{\textbackslash{}begin}\NormalTok{\{}\ExtensionTok{document}\NormalTok{\}}
\FunctionTok{\textbackslash{}frontmatter}\CommentTok{\%sayfa numaraları I,II,...}
\FunctionTok{\textbackslash{}maketitle}\CommentTok{\%başlığı oluştur}
\FunctionTok{\textbackslash{}tableofcontents}\CommentTok{\%içindekiler tablosu}

\KeywordTok{\textbackslash{}chapter}\NormalTok{\{Önsöz\}}
\NormalTok{  İçerik...}

\FunctionTok{\textbackslash{}mainmatter} \CommentTok{\% sayfa numaraları 1,2,3,...}

\KeywordTok{\textbackslash{}part}\NormalTok{\{Birinci Kısım\}}

\KeywordTok{\textbackslash{}chapter}\NormalTok{\{Birinci Bölüm\}}
\KeywordTok{\textbackslash{}section}\NormalTok{\{Altbölüm\}}
\NormalTok{  İçerik...}

\FunctionTok{\textbackslash{}appendix} \CommentTok{\%eklerin başlangıcı}

\KeywordTok{\textbackslash{}part}\NormalTok{\{Ekler\}}

\KeywordTok{\textbackslash{}chapter}\NormalTok{\{Birinci Ek\}}

\NormalTok{  İçerik...}

\FunctionTok{\textbackslash{}backmatter}

\KeywordTok{\textbackslash{}end}\NormalTok{\{}\ExtensionTok{document}\NormalTok{\}}
\end{Highlighting}
\end{Shaded}

Kitap gibi büyük hacimli belgelerle çalışırken kaynak dosyanızı
parçalara ayırmak gerekebilir. {\LaTeX} bunun için size iki komutla
yardımcı olur: \texttt{\textbackslash{}input} ve
\texttt{\textbackslash{}include}. İkisi arasındaki fark
\texttt{\textbackslash{}include} komutuyla eklediğiniz metin yeni bir
sayfadan başlayarak dizilir.

Bu komutların zorunlu değişkeni eklemek istediğiniz dosyanın adıdır.
Örneğin kaynak dosyanızla aynı dizinde yer alan \texttt{dosya1.tex}
dosyasını eklemek için

\begin{Shaded}
\begin{Highlighting}[]
\BuiltInTok{\textbackslash{}include}\NormalTok{\{}\ExtensionTok{dosya1}\NormalTok{\}}
\end{Highlighting}
\end{Shaded}

komutunu kullanırsınız. Eğer dosya uzantısı \texttt{.tex} değilse
(örneğin \texttt{.txt} olsun) o zaman dosya adını uzantısıyla yazmanız
gerekir:

\begin{Shaded}
\begin{Highlighting}[]
\BuiltInTok{\textbackslash{}include}\NormalTok{\{}\ExtensionTok{dosya1.txt}\NormalTok{\}}
\end{Highlighting}
\end{Shaded}

Ayrıca, hangi dosyaların eklenebileceğini {\LaTeX}'e bildiren bir komut
vardır: \texttt{\textbackslash{}includeonly}. Bu komut, sadece sahanlığa
yazılabilir. Komutun zorunlu değişkeninde eklenebilecek dosyalar
aralarına virgül koyularak (ve boşluk bırakılmadan) listelenir:

\begin{Shaded}
\begin{Highlighting}[]
\FunctionTok{\textbackslash{}includeonly}\NormalTok{\{dosya1,dosya2,dosya3,...\}}
\end{Highlighting}
\end{Shaded}

Böyle bir liste oluşturulduktan sonra bu listede olmayan bir dosya artık
\texttt{\textbackslash{}include} komutuyla kaynak dosyaya eklenemez.

\texttt{\textbackslash{}input} komutu sahanlıkta da kullanılabilir.
Örneğin, sahanlığınızı tek bir dosyaya yazıp, bu dosyayı bu komutla
sahanlığa ekleyebilirsiniz.

Daha düzenli çalışmak adına kaynak dosyanızın olduğu dizini de
düzenleyebilirsiniz.

\begin{figure}

{\centering \includegraphics[width=0.6\textwidth,height=\textheight]{./images/dizin.png}

}

\caption{\label{fig-dizin}Kaynak dosyanın olduğu dizinin düzenlenmesi}

\end{figure}

Bu şekilde bir düzenleme yaptığınızda \texttt{\textbackslash{}input} ya
da \texttt{\textbackslash{}include} komutlarıyla dosya eklemek
istediğinizde dosyanın bulunduğu dizini de göstermeniz gerekir.

Burada kaynak dosya \texttt{kitap.tex}'dir. Bu kaynak dosyaya
\texttt{bolum1.tex} dosyasını eklemek istediğinizde komutu

\begin{Shaded}
\begin{Highlighting}[]
\FunctionTok{\textbackslash{}input}\NormalTok{\{Bolumler/bolum1\}}
\end{Highlighting}
\end{Shaded}

şeklinde verirsiniz. Bu sayede kaynak dosyanızın olduğu dizinde (Örnek
Kitap) sadece \texttt{kitap} ile başlayan dosyalar olur. Diğer dosyalar
alt dizinde (Bolumler) yer alır.

Dikkat edilirse ``Örnek Kitap'' dışında, ``Bolumler'' alt dizini ve tüm
dosya adları Türkçe karakter ya da boşluk içermez.

\bookmarksetup{startatroot}

\hypertarget{dizgi}{%
\chapter{Dizgi}\label{dizgi}}

\hypertarget{satux131r-ve-sayfa-kesme}{%
\section{Satır ve Sayfa Kesme}\label{satux131r-ve-sayfa-kesme}}

{\LaTeX}, kelimeler arası boşlukları otomatik ayarlayarak satırları iki
yana yaslayarak dizer. Bir satırı kesip yeni bir satıra geçmek için
\texttt{\textbackslash{}\textbackslash{}} veya
\texttt{\textbackslash{}newline} komutları kullanılır.

Birinci komut \texttt{\textbackslash{}\textbackslash{}*} şeklinde
verildiğinde satırdan sonra sayfa kesilmesini önler.

Benzer şeyi \texttt{\textbackslash{}linebreak} komutu da yapar. Fakat bu
komut ile satır kesilirse {\LaTeX} kalan yarım satırı iki yana yaslar.
\texttt{\textbackslash{}nolinebreak} komutu ise satırın kesilmesini
önler.

\begin{Shaded}
\begin{Highlighting}[]
\BuiltInTok{\textbackslash{}documentclass}\NormalTok{\{}\ExtensionTok{article}\NormalTok{\}}
\BuiltInTok{\textbackslash{}usepackage}\NormalTok{[T1]\{}\ExtensionTok{fontenc}\NormalTok{\}}
\BuiltInTok{\textbackslash{}usepackage}\NormalTok{[turkish]\{}\ExtensionTok{babel}\NormalTok{\}}
\KeywordTok{\textbackslash{}begin}\NormalTok{\{}\ExtensionTok{document}\NormalTok{\}}

\FunctionTok{\textbackslash{}LaTeX}\NormalTok{, kelimeler arası boşlukları otomatik ayarlayarak satırları iki}
\NormalTok{yana yaslayarak dizer. Bir satırı kesip yeni bir satıra geçmek için }\FunctionTok{\textbackslash{}\textbackslash{}}
\NormalTok{veya }\FunctionTok{\textbackslash{}newline}\NormalTok{ komutları kullanılır.}

\KeywordTok{\textbackslash{}end}\NormalTok{\{}\ExtensionTok{document}\NormalTok{\}}
\end{Highlighting}
\end{Shaded}

Birçok kelimeyi birlikte aynı satırda tutmak gerekirse
\texttt{\textbackslash{}mbox} komutu kullanılır:

\begin{Shaded}
\begin{Highlighting}[]
\FunctionTok{\textbackslash{}mbox}\NormalTok{\{\textless{}metin\textgreater{}\}}
\end{Highlighting}
\end{Shaded}

Buradaki \texttt{\textless{}metin\textgreater{}} içindeki kelimeler her
durumda birleşik kalırlar. Benzer şeyi \texttt{\textbackslash{}fbox}
komutu da metin etrafına çizgi çizerek yapar.

Sayfayı kesip yeni bir sayfaya geçmek için
\texttt{\textbackslash{}newpage} ya da
\texttt{\textbackslash{}pagebreak} komutları kullanılır.
\texttt{\textbackslash{}nopagebreak} komutu sayfa kesilmesini önler.
\texttt{\textbackslash{}newpage} ile \texttt{\textbackslash{}pagebreak}
komutları arasında da \texttt{\textbackslash{}newline} ile
\texttt{\textbackslash{}linebreak} komutlarındakine benzer bir fark
vardır.

\hypertarget{heceleme}{%
\section{Heceleme}\label{heceleme}}

Bazen tüm bu ayarlamalara rağmen {\LaTeX} bazı kelimeleri doğru
heceleyemeyebilir. Böyle durumlarda hecelemeyi elle yapmak gerekir.
Yanlış hecelenen kelimenin bölünebileceği yerler
\texttt{\textbackslash{}-} komutuyla gösterilir:

\begin{Shaded}
\begin{Highlighting}[]
\NormalTok{He}\FunctionTok{\textbackslash{}{-}}\NormalTok{ce}\FunctionTok{\textbackslash{}{-}}\NormalTok{le}\FunctionTok{\textbackslash{}{-}}\NormalTok{me}
\end{Highlighting}
\end{Shaded}

Bu sadece ilgili kelimenin tireyle ayrıldığı yerde doğru hecelenmesini
sağlar. Aynı kelime belgenin başka bir yerinde yine yalnış
hecelenebilir. Bunun yerine \texttt{\textbackslash{}begin\{document\}}
komutundan sonra \texttt{\textbackslash{}hyphenation} komutuyla hece
yerleri tire (\texttt{-}) işaretiyle gösterilmiş olan kelime listesi
oluşturulursa belgenin tamamına bu kural uygulanmış olur. Örneğin

\begin{Shaded}
\begin{Highlighting}[]
\FunctionTok{\textbackslash{}hyphenation}\NormalTok{\{He{-}ce{-}le{-}me FORTRAN\}}
\end{Highlighting}
\end{Shaded}

komutuyla ``Heceleme'' kelimesinin nereden bölüneceği, ``FORTRAN'',
``Fortran'' ya da ``fortran'' kelimelerinin bölünmeyeceği {\LaTeX}'e
söylenmiş olur.

\hypertarget{paragraflar-ve-cuxfcmle-sonlarux131}{%
\section{Paragraflar ve Cümle
Sonları}\label{paragraflar-ve-cuxfcmle-sonlarux131}}

Boş bir satırın yeni bir paragraf açtığını daha önce belirtmiştik. Aynı
şey, \texttt{\textbackslash{}par} komutuyla da yapılabilir. Ancak bu
komut yeni bir paragraf açmaktan ziyade farklı amaçlar için kullanılır
(yeri geldiğinde değinilecektir). Nitekim kaynak dosyanızın
okunabilirliği açısından paragrafları ayırmak için boş bir satır
bırakmak daha kullanışlıdır.

{\LaTeX}'de varsayılan olarak \texttt{\textbackslash{}chapter} ve
\texttt{\textbackslash{}section} gibi bölümleme komutlarından sonra
oluşturulan ilk paragraf girintisiz, sonrakiler girintili olur. Bu,
paragraf başlarında \texttt{\textbackslash{}indent} ya da
\texttt{\textbackslash{}noindent} komutlarıyla tek seferliğine
değiştirilebilir. Birinci komut girinti oluşturur, ikincisi ise
girintiyi kaldırır.

{\LaTeX}, okumayı kolaylaştırmak için cümle sonlarında fazladan
boşluklar bırakır. Bunu yaparken de her cümlenin nokta, soru işareti
veya ünlem işaretiyle bittiğini varsayar. Kısaltmalarda büyük harflerden
sonra nokta gel(ebil)diğinden, büyük harften sonra nokta koyulursa
{\LaTeX} bunu cümle sonu saymaz. Eğer bir büyük harften sonra nokta
koyuyorsanız ve burası cümlenin sonuysa {\LaTeX}'in burayı cümle sonu
sayması için büyük harften sonraki noktanın önüne
\texttt{\textbackslash{}@} koymanız gerekir.

{\LaTeX}'in noktadan sonra fazladan boşluk \emph{koymamasını} isterseniz
\texttt{\textbackslash{}frenchspacing} komutunu kullanırsınız. Bu komutu
kullandıysanız, artık noktadan önce \texttt{\textbackslash{}@} koymanıza
gerek yoktur. Daha sonra tekrar cümle sonlarında fazladan boşluk
kullanmak istenirse de \texttt{\textbackslash{}nonfrenchspacing} komutu
kullanılır.

Unvan kısaltmasından sonra unvanın ait olduğu kelimeyle birlikte kalması
ve fazladan boşluk bırakılmaması için tilda (\texttt{\textasciitilde{}})
işareti kullanılabilir. Bu işaret hem genişlemeyen bir boşluk bırakır
hem de satırın orada kesilmesini önler.

\hypertarget{aralux131klar}{%
\section{Aralıklar}\label{aralux131klar}}

{\LaTeX}'de hem dikey hem de yatay aralıklar otomatik olarak ayarlanır.
Fazladan aralıklar bırakmak için komutlar kullanılır.

Aralık bırakırken kullanabileceğimiz ölçü birimleri
Tablo~\ref{tbl-olcu}'de gösterilmiştir.

\hypertarget{tbl-olcu}{}
\begin{longtable}[]{@{}ll@{}}
\caption{\label{tbl-olcu}{\LaTeX}'de Ölçü Birimleri}\tabularnewline
\toprule()
Birim & Değer \\
\midrule()
\endfirsthead
\toprule()
Birim & Değer \\
\midrule()
\endhead
mm & milimetre \(\approx 1/25\) inç \\
cm & santimetre = 10 mm \\
in & inç = 25.4 mm \\
pt & punto \(\approx 1/72\) inç \\
em & Kullanılan yazı tipinde `M' harfinin genişliği \\
ex & Kullanılan yazı tipinde `x' harfinin yüksekliği \\
\bottomrule()
\end{longtable}

\hypertarget{dikey-aralux131klar}{%
\subsection{Dikey aralıklar}\label{dikey-aralux131klar}}

Dikey aralık birkaç komutla bırakılabilir. Bunlardan biri
\texttt{\textbackslash{}vspace} olup, komut iki boş satır arasında

\begin{Shaded}
\begin{Highlighting}[]
\FunctionTok{\textbackslash{}vspace}\NormalTok{\{\textless{}uzunluk\textgreater{}\}}
\end{Highlighting}
\end{Shaded}

şeklinde verilir. Komut bu şekilde verildiğinde komutun zorunlu
değişkeninde birimiyle belirtilen uzunluk kadar dikey aralık bırakılır.
Eğer bir sayfanın başında veya sonunda aralık bırakılmak istenirse,
komut \texttt{\textbackslash{}vspace*} şeklinde yıldızlı vermelidir. Bu
komutun aralığa ilave yapan \texttt{\textbackslash{}addvspace} sürümü de
vardır.

Bir paragrafın iki satırı arasında veya bir tablonun satırları arasında
ilave aralık açmak için

\begin{Shaded}
\begin{Highlighting}[]
\FunctionTok{\textbackslash{}\textbackslash{}}\NormalTok{[\textless{}uzunluk\textgreater{}]}
\end{Highlighting}
\end{Shaded}

komutu kullanılır. Bu komutlarda belirtilen uzunluklar negatif de
olabilir.

Sınırsız bir dikey aralık oluşturmak için \texttt{\textbackslash{}vfill}
komutu kullanılır. Bu komuttan sonra gelen her şey sayfanın altına
yaslanır.

Ön tanımlı gelen \texttt{\textbackslash{}smallskip},
\texttt{\textbackslash{}medskip} ve \texttt{\textbackslash{}bigskip}
komutları sırasıyla küçük, orta ve büyük aralıklar bırakır.

\hypertarget{yatay-aralux131klar}{%
\subsection{Yatay aralıklar}\label{yatay-aralux131klar}}

Ön tanımlı yatay aralıklar

\begin{Shaded}
\begin{Highlighting}[]
\FunctionTok{\textbackslash{} } \FunctionTok{\textbackslash{},}  \FunctionTok{\textbackslash{}:}  \FunctionTok{\textbackslash{};}  \FunctionTok{\textbackslash{}quad}  \FunctionTok{\textbackslash{}qquad}  \FunctionTok{\textbackslash{}!}
\end{Highlighting}
\end{Shaded}

komutlarıyla verilir. Bu komutlar sırasıyla bir sözcük arası, \(3/\!18\)
em, \(4/\!18\) em, \(5/\!18\) em, \(1\) em, \(2\) em, \(-3/\!18\) em
uzunlukta yatay aralık bırakır.

Belli bir uzunlukta yatay aralık bırakmak için
\texttt{\textbackslash{}hspace} komutu kullanılır. Yine dikey aralıkta
olduğu gibi yatay aralık negatif de olabilir. Eğer aralık satır başına
veya sonuna rasgelse dahi bu aralığı korumak istiyorsanız, yıldızlı
\texttt{\textbackslash{}hspace*} komutu kullanırsınız.

Komutlardan önce veya sonra boşluk bırakmak farklı sonuçlar üretir.

Sınırsız bir yatay aralık oluşturmak için \texttt{\textbackslash{}hfill}
komutu kullanılır. Bu komuttan sonra gelen her şey satırın sonuna
yaslanır. Hem satır sonuna yaslamak hem de aralığı noktalarla doldurmak
isterseniz \texttt{\textbackslash{}dotfill} komutunu kullanırsınız.
Satır sonuna yaslayıp aralığa çizgi çekmek isterseniz de
\texttt{\textbackslash{}hrulefill} komutunu kullanırsınız.

\begin{Shaded}
\begin{Highlighting}[]
\BuiltInTok{\textbackslash{}documentclass}\NormalTok{\{}\ExtensionTok{article}\NormalTok{\}}
\BuiltInTok{\textbackslash{}usepackage}\NormalTok{[T1]\{}\ExtensionTok{fontenc}\NormalTok{\}}
\BuiltInTok{\textbackslash{}usepackage}\NormalTok{[turkish]\{}\ExtensionTok{babel}\NormalTok{\}}
\KeywordTok{\textbackslash{}begin}\NormalTok{\{}\ExtensionTok{document}\NormalTok{\}}

\FunctionTok{\textbackslash{}noindent} 
\NormalTok{A}\FunctionTok{\textbackslash{}\textbackslash{}}\NormalTok{[1ex]}
\NormalTok{B}

\FunctionTok{\textbackslash{}bigskip}

\FunctionTok{\textbackslash{}noindent} 
\NormalTok{A}\FunctionTok{\textbackslash{}\textbackslash{}}\NormalTok{[{-}2ex]}
\NormalTok{B}

\FunctionTok{\textbackslash{}bigskip}

\FunctionTok{\textbackslash{}noindent} 
\NormalTok{A}\FunctionTok{\textbackslash{}hspace}\NormalTok{\{2cm\}B}\FunctionTok{\textbackslash{}\textbackslash{}}
\NormalTok{A }\FunctionTok{\textbackslash{}hspace}\NormalTok{\{2cm\} B}

\FunctionTok{\textbackslash{}bigskip}

\FunctionTok{\textbackslash{}noindent} 
\NormalTok{A}\FunctionTok{\textbackslash{}hfill}\NormalTok{ B}\FunctionTok{\textbackslash{}\textbackslash{}}
\NormalTok{A}\FunctionTok{\textbackslash{}dotfill}\NormalTok{ B}\FunctionTok{\textbackslash{}\textbackslash{}}
\NormalTok{A}\FunctionTok{\textbackslash{}hrulefill}\NormalTok{ B}

\KeywordTok{\textbackslash{}end}\NormalTok{\{}\ExtensionTok{document}\NormalTok{\}}
\end{Highlighting}
\end{Shaded}

\hypertarget{ortamlar-1}{%
\section{Ortamlar}\label{ortamlar-1}}

\hypertarget{metin-hizalama}{%
\subsection{Metin hizalama}\label{metin-hizalama}}

{\LaTeX}'de metni sola hizalamak için \texttt{flushleft}, sağa hizalamak
için \texttt{flushright} ve ortalı hizalamak için \texttt{center}
ortamları kullanılır.

\begin{Shaded}
\begin{Highlighting}[]
\BuiltInTok{\textbackslash{}documentclass}\NormalTok{\{}\ExtensionTok{article}\NormalTok{\}}
\BuiltInTok{\textbackslash{}usepackage}\NormalTok{[T1]\{}\ExtensionTok{fontenc}\NormalTok{\}}
\BuiltInTok{\textbackslash{}usepackage}\NormalTok{[turkish]\{}\ExtensionTok{babel}\NormalTok{\}}
\KeywordTok{\textbackslash{}begin}\NormalTok{\{}\ExtensionTok{document}\NormalTok{\}}

\KeywordTok{\textbackslash{}begin}\NormalTok{\{}\ExtensionTok{flushleft}\NormalTok{\}}
\NormalTok{ burası sola hizalı}
\KeywordTok{\textbackslash{}end}\NormalTok{\{}\ExtensionTok{flushleft}\NormalTok{\}}
\KeywordTok{\textbackslash{}begin}\NormalTok{\{}\ExtensionTok{flushright}\NormalTok{\}}
\NormalTok{ sağa hizalı}
\KeywordTok{\textbackslash{}end}\NormalTok{\{}\ExtensionTok{flushright}\NormalTok{\}}
\KeywordTok{\textbackslash{}begin}\NormalTok{\{}\ExtensionTok{center}\NormalTok{\}}
\NormalTok{ ve ortalı}
\KeywordTok{\textbackslash{}end}\NormalTok{\{}\ExtensionTok{center}\NormalTok{\}}


\KeywordTok{\textbackslash{}end}\NormalTok{\{}\ExtensionTok{document}\NormalTok{\}}
\end{Highlighting}
\end{Shaded}

\hypertarget{suxfctunlara-ayux131rma}{%
\subsection{Sütunlara ayırma}\label{suxfctunlara-ayux131rma}}

{\LaTeX}'de belgenin tamamının iki sütun dizilmesi için
\texttt{\textbackslash{}documentclass} komutunun seçeneğine
\texttt{twocolumn} yazılabileceğinden bahsettik. Bu, tüm belgenin iki
sütun dizilmesini sağlar. Bazı sayfaları iki, bazılarınıysa tek sütun
dizmek istiyorsanız \texttt{\textbackslash{}twocolumn} ve
\texttt{\textbackslash{}onecolumn} komutlarını kullanmanız gerekir.
\texttt{\textbackslash{}twocolumn} komutunun verildiği sayfadan sonraki
sayfalar iki, \texttt{\textbackslash{}onecolumn} komutunun verildiği
sayfadan sonraki sayfalar tek sütun dizilir.

Eğer metni daha fazla sütuna bölmek ve sütunları istediğiniz yerden
başlatmak gibi daha fazla seçenek istiyorsanız, \texttt{multicols}
ortamını kullanmanız gerekir. Bu ortamı kullanabilmek için

\begin{Shaded}
\begin{Highlighting}[]
\BuiltInTok{\textbackslash{}usepackage}\NormalTok{\{}\ExtensionTok{multicol}\NormalTok{\}}
\end{Highlighting}
\end{Shaded}

komutuyla
\href{http://ftp.ntua.gr/mirror/ctan/macros/latex/required/tools/multicol.pdf}{\textbf{multicol}}
paketini eklemelisiniz.

\begin{Shaded}
\begin{Highlighting}[]
\KeywordTok{\textbackslash{}begin}\NormalTok{\{}\ExtensionTok{multicols}\NormalTok{\}\{\textless{}sütun sayısı\textgreater{}\}}

\KeywordTok{\textbackslash{}end}\NormalTok{\{}\ExtensionTok{multicols}\NormalTok{\}}
\end{Highlighting}
\end{Shaded}

Burada, \texttt{\textless{}sütun\ sayısı\textgreater{}} değişkeninde
oluşturulmak istenen sütun adedi sayıyla belirtilir.

Bu ortamda sütun genişlikleri eşit olup, sütunlar arası boşluk
\texttt{\textbackslash{}columnsep}, sütunlar arasındaki çizginin
kalınlığı \texttt{\textbackslash{}columnseprule} ve sütunlar arasındaki
çizginin rengi \texttt{\textbackslash{}columnseprulecolor} komutlarında
saklıdır. Bu değişkenler \texttt{\textbackslash{}setlength} ya da
\texttt{\textbackslash{}def} komutları kullanılarak değiştirilebilir.

\begin{Shaded}
\begin{Highlighting}[]
\FunctionTok{\textbackslash{}setlength}\NormalTok{\{}\FunctionTok{\textbackslash{}columnsep}\NormalTok{\}\{1cm\}}
\FunctionTok{\textbackslash{}setlength}\NormalTok{\{}\FunctionTok{\textbackslash{}columnseprule}\NormalTok{\}\{1pt\}}
\FunctionTok{\textbackslash{}def\textbackslash{}columnseprulecolor}\NormalTok{\{}\FunctionTok{\textbackslash{}color}\NormalTok{\{blue\}\}}
\end{Highlighting}
\end{Shaded}

Yukarıdaki birinci komutla sütunlar arasındaki boşluk 1 cm, çizgi
kalınlığı 1 pt ve çizgi rengi mavi olarak düzenlenir. Bu komutlar ya
sahanlığa ya da ortamı kullanmadan önce gövdeye yazılmalıdır.

\begin{tcolorbox}[enhanced jigsaw, opacitybacktitle=0.6, coltitle=black, leftrule=.75mm, rightrule=.15mm, toprule=.15mm, bottomtitle=1mm, titlerule=0mm, colbacktitle=quarto-callout-note-color!10!white, breakable, arc=.35mm, opacityback=0, colframe=quarto-callout-note-color-frame, toptitle=1mm, title=\textcolor{quarto-callout-note-color}{\faInfo}\hspace{0.5em}{Not}, bottomrule=.15mm, left=2mm, colback=white]
Şimdiye kadar renk kullanımından bahsetmedik ancak çizgi rengini
değiştirmek için verilen komutun kullanılabilmesi için sahanlığa
\texttt{\textbackslash{}usepackage\{xcolor\}} komutuyla \texttt{xcolor}
paketinin eklenmesi gerekir.
\end{tcolorbox}

Ortam isteğe bağlı bir değişken de alabilir. Bu, çengelli parantezlerden
sonra köşeli parantezler içine yazılır. Köşeli parantezler içinde
yazılanlar bölünmeden ve çok sütunlu metnin üstünde dizilir.

\begin{Shaded}
\begin{Highlighting}[]
\BuiltInTok{\textbackslash{}documentclass}\NormalTok{\{}\ExtensionTok{article}\NormalTok{\}}
\BuiltInTok{\textbackslash{}usepackage}\NormalTok{[T1]\{}\ExtensionTok{fontenc}\NormalTok{\}}
\BuiltInTok{\textbackslash{}usepackage}\NormalTok{[turkish]\{}\ExtensionTok{babel}\NormalTok{\}}
\BuiltInTok{\textbackslash{}usepackage}\NormalTok{\{}\ExtensionTok{multicol}\NormalTok{\}}
\BuiltInTok{\textbackslash{}usepackage}\NormalTok{\{}\ExtensionTok{xcolor}\NormalTok{\}}
\BuiltInTok{\textbackslash{}usepackage}\NormalTok{\{}\ExtensionTok{lipsum}\NormalTok{\}}\CommentTok{\% Örnek verirken anlamsız metinler oluşturmaya yarayan bir paket}
\KeywordTok{\textbackslash{}begin}\NormalTok{\{}\ExtensionTok{document}\NormalTok{\}}

\KeywordTok{\textbackslash{}begin}\NormalTok{\{}\ExtensionTok{multicols}\NormalTok{\}\{2\}}
\NormalTok{  Bilim, yalnızca bir bilgi edinme etkinliği değildir. Onun bir}
\NormalTok{  eylem yönünün olduğu da unutulmamalıdır.}
\KeywordTok{\textbackslash{}end}\NormalTok{\{}\ExtensionTok{multicols}\NormalTok{\}}

\FunctionTok{\textbackslash{}setlength}\NormalTok{\{}\FunctionTok{\textbackslash{}columnsep}\NormalTok{\}\{1cm\}}
\FunctionTok{\textbackslash{}setlength}\NormalTok{\{}\FunctionTok{\textbackslash{}columnseprule}\NormalTok{\}\{1pt\}}
\FunctionTok{\textbackslash{}def\textbackslash{}columnseprulecolor}\NormalTok{\{}\FunctionTok{\textbackslash{}color}\NormalTok{\{blue\}\}}
\KeywordTok{\textbackslash{}begin}\NormalTok{\{}\ExtensionTok{multicols}\NormalTok{\}\{2\}}
  \FunctionTok{\textbackslash{}lipsum}\NormalTok{[1{-}2] }\CommentTok{\% İki paragraf anlamsız örnek metin oluşturma komutu}
\KeywordTok{\textbackslash{}end}\NormalTok{\{}\ExtensionTok{multicols}\NormalTok{\}}


\KeywordTok{\textbackslash{}end}\NormalTok{\{}\ExtensionTok{document}\NormalTok{\}}
\end{Highlighting}
\end{Shaded}

Sütunu kesmek için \texttt{\textbackslash{}columnbreak} komutu
kullanılır. Komutun verildiği yerde sütun kesilir, ardından kesme
noktasından önceki paragraflar tüm kullanılabilir alanı doldurmak için
eşit olarak dağıtılır. Dolayısıyla bazen beklenen sonucu vermeyebilir.

Varsayılan \texttt{multicols} ortamında sütunların her biri aynı
miktarda metin içerecek şekilde dengelenmiştir. Bu, ortamın yıldızlı
sürümü (\texttt{multicols*}) kullanılarak değiştirilebilir.

\hypertarget{listeleme}{%
\subsection{Listeleme}\label{listeleme}}

\hypertarget{temel-listeler}{%
\subsubsection{Temel listeler}\label{temel-listeler}}

{\LaTeX}'de listeleme için değişik ortamlar vardır. Bu ortamlar tek
başına kullanılabileceği gibi birlikte de kullanılabilirler. Her ortamda
maddeler \texttt{\textbackslash{}item} komutuyla belirtilir.

Bir listeyi numaralı şekilde dizmek için \texttt{enumerate} ortamı
kullanılır.

\begin{Shaded}
\begin{Highlighting}[]
\BuiltInTok{\textbackslash{}documentclass}\NormalTok{\{}\ExtensionTok{article}\NormalTok{\}}
\BuiltInTok{\textbackslash{}usepackage}\NormalTok{[T1]\{}\ExtensionTok{fontenc}\NormalTok{\}}
\BuiltInTok{\textbackslash{}usepackage}\NormalTok{[turkish]\{}\ExtensionTok{babel}\NormalTok{\}}
\KeywordTok{\textbackslash{}begin}\NormalTok{\{}\ExtensionTok{document}\NormalTok{\}}

\KeywordTok{\textbackslash{}begin}\NormalTok{\{}\ExtensionTok{enumerate}\NormalTok{\}}
 \FunctionTok{\textbackslash{}item}\NormalTok{ madde 1}
  \KeywordTok{\textbackslash{}begin}\NormalTok{\{}\ExtensionTok{enumerate}\NormalTok{\}}
    \FunctionTok{\textbackslash{}item}\NormalTok{ alt madde 1}
      \KeywordTok{\textbackslash{}begin}\NormalTok{\{}\ExtensionTok{enumerate}\NormalTok{\}}
        \FunctionTok{\textbackslash{}item}\NormalTok{ en alt madde 1}
      \KeywordTok{\textbackslash{}end}\NormalTok{\{}\ExtensionTok{enumerate}\NormalTok{\}}
    \FunctionTok{\textbackslash{}item}\NormalTok{ alt madde 2}
  \KeywordTok{\textbackslash{}end}\NormalTok{\{}\ExtensionTok{enumerate}\NormalTok{\}}
 \FunctionTok{\textbackslash{}item}\NormalTok{ madde 2}
\KeywordTok{\textbackslash{}end}\NormalTok{\{}\ExtensionTok{enumerate}\NormalTok{\}}

\KeywordTok{\textbackslash{}end}\NormalTok{\{}\ExtensionTok{document}\NormalTok{\}}
\end{Highlighting}
\end{Shaded}

Numarasız, özel işaretli listeler için \texttt{itemize} ortamı
kullanılır ve bu ortamda madde işareti değiştirilebilir.

Açıklamalı bir liste içinse \texttt{description} ortamı kullanılır. Bu
ortamda köşeli parantez içine alınan anahtar kelimeler kalın dizilir.

\begin{Shaded}
\begin{Highlighting}[]
\BuiltInTok{\textbackslash{}documentclass}\NormalTok{\{}\ExtensionTok{article}\NormalTok{\}}
\BuiltInTok{\textbackslash{}usepackage}\NormalTok{[T1]\{}\ExtensionTok{fontenc}\NormalTok{\}}
\BuiltInTok{\textbackslash{}usepackage}\NormalTok{[turkish]\{}\ExtensionTok{babel}\NormalTok{\}}
\KeywordTok{\textbackslash{}begin}\NormalTok{\{}\ExtensionTok{document}\NormalTok{\}}

\KeywordTok{\textbackslash{}begin}\NormalTok{\{}\ExtensionTok{itemize}\NormalTok{\}}
  \FunctionTok{\textbackslash{}item}\NormalTok{ madde 1}
  \FunctionTok{\textbackslash{}item}\NormalTok{ madde 2}
  \FunctionTok{\textbackslash{}item}\NormalTok{[}\SpecialStringTok{$}\SpecialCharTok{\textbackslash{}circ}\SpecialStringTok{$}\NormalTok{] madde 3}
  \FunctionTok{\textbackslash{}item}\NormalTok{[+] madde 4}
\KeywordTok{\textbackslash{}end}\NormalTok{\{}\ExtensionTok{itemize}\NormalTok{\}}

\KeywordTok{\textbackslash{}begin}\NormalTok{\{}\ExtensionTok{description}\NormalTok{\}}
  \FunctionTok{\textbackslash{}item}\NormalTok{[Nokta] Boyutu olmayan}
  \FunctionTok{\textbackslash{}item}\NormalTok{[Çember] Bir noktaya eşit}
\NormalTok{  uzaklıktaki noktaların geometrik yeri}
\KeywordTok{\textbackslash{}end}\NormalTok{\{}\ExtensionTok{description}\NormalTok{\}}
\KeywordTok{\textbackslash{}end}\NormalTok{\{}\ExtensionTok{document}\NormalTok{\}}
\end{Highlighting}
\end{Shaded}

\hypertarget{listeleri-uxf6zelleux15ftirmek}{%
\subsubsection{Listeleri
özelleştirmek}\label{listeleri-uxf6zelleux15ftirmek}}

Listelerin özelleştirmek için
\href{http://ftp.ntua.gr/mirror/ctan/macros/latex/required/tools/enumerate.pdf}{\textbf{enumerate}}
paketi kullanılabilir. Paketi
\texttt{\textbackslash{}usepackage\{enumerate\}} komutuyla ekledikten
sonra \texttt{enumerate} ortamını başlatan komutun peşine köşeli
parantezler içinde madde işaretlerinin tipi belirtilebilir:

\begin{Shaded}
\begin{Highlighting}[]
\BuiltInTok{\textbackslash{}documentclass}\NormalTok{\{}\ExtensionTok{article}\NormalTok{\}}
\BuiltInTok{\textbackslash{}usepackage}\NormalTok{[T1]\{}\ExtensionTok{fontenc}\NormalTok{\}}
\BuiltInTok{\textbackslash{}usepackage}\NormalTok{[turkish]\{}\ExtensionTok{babel}\NormalTok{\}}
\BuiltInTok{\textbackslash{}usepackage}\NormalTok{\{}\ExtensionTok{enumerate}\NormalTok{\}}
\KeywordTok{\textbackslash{}begin}\NormalTok{\{}\ExtensionTok{document}\NormalTok{\}}

\KeywordTok{\textbackslash{}begin}\NormalTok{\{}\ExtensionTok{enumerate}\NormalTok{\}[I.]}
 \FunctionTok{\textbackslash{}item}\NormalTok{ bir}
 \FunctionTok{\textbackslash{}item}\NormalTok{ iki }
 \FunctionTok{\textbackslash{}item}\NormalTok{ üç}
\KeywordTok{\textbackslash{}end}\NormalTok{\{}\ExtensionTok{enumerate}\NormalTok{\}}

\KeywordTok{\textbackslash{}end}\NormalTok{\{}\ExtensionTok{document}\NormalTok{\}}
\end{Highlighting}
\end{Shaded}

Bunun dışında çok daha fazla özelleştirmeye izin veren
\href{http://ftp.cc.uoc.gr/mirrors/CTAN/macros/latex/contrib/enumitem/enumitem.pdf}{\textbf{enumitem}}
paketi vardır. Dileyen okur paket belgesini inceleyip listelerini
özelleştirebilir.

\hypertarget{uxf6zet}{%
\subsection{Özet}\label{uxf6zet}}

Makalenize (article) özet koymak isterseniz \texttt{abstract} ortamını
kullanırsınız.

\begin{Shaded}
\begin{Highlighting}[]
\BuiltInTok{\textbackslash{}documentclass}\NormalTok{\{}\ExtensionTok{article}\NormalTok{\}}
\BuiltInTok{\textbackslash{}usepackage}\NormalTok{[T1]\{}\ExtensionTok{fontenc}\NormalTok{\}}
\BuiltInTok{\textbackslash{}usepackage}\NormalTok{[turkish]\{}\ExtensionTok{babel}\NormalTok{\}}
\KeywordTok{\textbackslash{}begin}\NormalTok{\{}\ExtensionTok{document}\NormalTok{\}}

\KeywordTok{\textbackslash{}begin}\NormalTok{\{}\ExtensionTok{abstract}\NormalTok{\}}
\NormalTok{  Bu çalışmada}\FunctionTok{\textbackslash{}dots}
\KeywordTok{\textbackslash{}end}\NormalTok{\{}\ExtensionTok{abstract}\NormalTok{\}}

\KeywordTok{\textbackslash{}end}\NormalTok{\{}\ExtensionTok{document}\NormalTok{\}}
\end{Highlighting}
\end{Shaded}

\hypertarget{yazux131ldux131ux11fux131-gibi-basma}{%
\subsection{Yazıldığı gibi
basma}\label{yazux131ldux131ux11fux131-gibi-basma}}

Bir metin, nasıl yazıldıysa, yani boşluklar ve {\LaTeX} komutları işleme
konmadan olduğu gibi dizilmesi isteniyorsa verbatim ortamında yazılır.

Satır içinde aynı şey yapmak istenirse \texttt{\textbackslash{}verb}
komutu
\texttt{\textbackslash{}verb\textbar{}\textless{}metin\textgreater{}\textbar{}}
şeklinde kullanılır.

Eğer \texttt{verbatim} ortamı yıldızlı (\texttt{verbatim*}) şeklinde
kullanılırsa boşluklar özel bir işaretle gösterilir. Aynı şeyi
\texttt{\textbackslash{}verb*} komutu da yapar.

\begin{Shaded}
\begin{Highlighting}[]
\BuiltInTok{\textbackslash{}documentclass}\NormalTok{\{}\ExtensionTok{article}\NormalTok{\}}
\BuiltInTok{\textbackslash{}usepackage}\NormalTok{[T1]\{}\ExtensionTok{fontenc}\NormalTok{\}}
\BuiltInTok{\textbackslash{}usepackage}\NormalTok{[turkish]\{}\ExtensionTok{babel}\NormalTok{\}}
\KeywordTok{\textbackslash{}begin}\NormalTok{\{}\ExtensionTok{document}\NormalTok{\}}

\KeywordTok{\textbackslash{}begin}\NormalTok{\{}\ExtensionTok{verbatim}\NormalTok{\}}
\VerbatimStringTok{  Nasılsa öyle! \# " \textasciitilde{}}
\KeywordTok{\textbackslash{}end}\NormalTok{\{}\ExtensionTok{verbatim}\NormalTok{\}}

\FunctionTok{\textbackslash{}verb}\NormalTok{|}\FunctionTok{\textbackslash{}$}\NormalTok{| komutuyla }\FunctionTok{\textbackslash{}$}\NormalTok{ olur.}

\KeywordTok{\textbackslash{}begin}\NormalTok{\{}\ExtensionTok{verbatim}\NormalTok{*\}}
\NormalTok{  Boşluklar özel bir işatretle}
\NormalTok{  gösterilir.}
\KeywordTok{\textbackslash{}end}\NormalTok{\{}\ExtensionTok{verbatim*}\NormalTok{\}}

\FunctionTok{\textbackslash{}verb*}\NormalTok{|aynı şekilde.|}


\KeywordTok{\textbackslash{}end}\NormalTok{\{}\ExtensionTok{document}\NormalTok{\}}
\end{Highlighting}
\end{Shaded}

\hypertarget{baux15flux131k-sayfasux131-kapak}{%
\subsection{Başlık sayfası
(kapak)}\label{baux15flux131k-sayfasux131-kapak}}

Başlık sayfası için \texttt{titlepage} ortamı kullanılır.

\begin{Shaded}
\begin{Highlighting}[]
\BuiltInTok{\textbackslash{}documentclass}\NormalTok{\{}\ExtensionTok{article}\NormalTok{\}}
\BuiltInTok{\textbackslash{}usepackage}\NormalTok{[T1]\{}\ExtensionTok{fontenc}\NormalTok{\}}
\BuiltInTok{\textbackslash{}usepackage}\NormalTok{[turkish]\{}\ExtensionTok{babel}\NormalTok{\}}
\KeywordTok{\textbackslash{}begin}\NormalTok{\{}\ExtensionTok{document}\NormalTok{\}}

\KeywordTok{\textbackslash{}begin}\NormalTok{\{}\ExtensionTok{titlepage}\NormalTok{\}}
  \KeywordTok{\textbackslash{}begin}\NormalTok{\{}\ExtensionTok{center}\NormalTok{\}}\CommentTok{\%ortalı}

  \FunctionTok{\textbackslash{}vspace*}\NormalTok{\{}\FunctionTok{\textbackslash{}baselineskip}\NormalTok{\}}\CommentTok{\%boşluk}

\NormalTok{  \{}\FunctionTok{\textbackslash{}Large}\NormalTok{ Yazar adı\}}

  \FunctionTok{\textbackslash{}vspace*}\NormalTok{\{2cm\} }\CommentTok{\%boşluk}

  \FunctionTok{\textbackslash{}textbf}\NormalTok{\{}\FunctionTok{\textbackslash{}LARGE}\NormalTok{ İlk başlık satırı\}}\FunctionTok{\textbackslash{}\textbackslash{}}\NormalTok{[}\FunctionTok{\textbackslash{}baselineskip}\NormalTok{]}
\NormalTok{  \{}\FunctionTok{\textbackslash{}Huge}\NormalTok{ Ana Başlık\}}\FunctionTok{\textbackslash{}\textbackslash{}}\NormalTok{[}\FunctionTok{\textbackslash{}baselineskip}\NormalTok{]}
\NormalTok{  \{}\FunctionTok{\textbackslash{}Large} \FunctionTok{\textbackslash{}textit}\NormalTok{\{Alt başlık \}\}}

  \FunctionTok{\textbackslash{}vfill} \CommentTok{\%sonra gelenlerisayfa altına yasla}

\NormalTok{  \{}\FunctionTok{\textbackslash{}large}\NormalTok{ Yayıncı\}}

  \FunctionTok{\textbackslash{}vspace*}\NormalTok{\{3}\FunctionTok{\textbackslash{}baselineskip}\NormalTok{\}}\CommentTok{\%boşluk}

  \KeywordTok{\textbackslash{}end}\NormalTok{\{}\ExtensionTok{center}\NormalTok{\}}
\KeywordTok{\textbackslash{}end}\NormalTok{\{}\ExtensionTok{titlepage}\NormalTok{\}}

\KeywordTok{\textbackslash{}end}\NormalTok{\{}\ExtensionTok{document}\NormalTok{\}}
\end{Highlighting}
\end{Shaded}

Başlık bu şekilde oluşturulduğunda bu sayfada numara basılmaz ve bu
sayfa bir numaralı sayfa olur. Ayrıca,
\texttt{\textbackslash{}maketitle} komutuyla oluşturulan başlıktan
farklı olarak sayfadaki her şeyi dilediğiniz gibi
biçimlendirebilirsiniz.

\hypertarget{alux131ntux131-yapmak-ve-ux15fiir-dizmek}{%
\subsection{Alıntı yapmak ve şiir
dizmek}\label{alux131ntux131-yapmak-ve-ux15fiir-dizmek}}

{\LaTeX}'de kısa alıntılar için \texttt{quote}, çok paragraflı uzun
alıntılar için \texttt{quotation}, şiir dizmek içinse \texttt{verse}
ortamı kullanılır. \texttt{quotation} ortamında her paragrafın ilk
satırı içerden başlar. \texttt{verse} ortamında şiir satırları
\texttt{\textbackslash{}\textbackslash{}} komutuyla sonlandırılır ve
kıtalar arası boş bir satır bırakılır.

\begin{Shaded}
\begin{Highlighting}[]
\BuiltInTok{\textbackslash{}documentclass}\NormalTok{\{}\ExtensionTok{article}\NormalTok{\}}
\BuiltInTok{\textbackslash{}usepackage}\NormalTok{[T1]\{}\ExtensionTok{fontenc}\NormalTok{\}}
\BuiltInTok{\textbackslash{}usepackage}\NormalTok{[turkish]\{}\ExtensionTok{babel}\NormalTok{\}}
\KeywordTok{\textbackslash{}begin}\NormalTok{\{}\ExtensionTok{document}\NormalTok{\}}

\NormalTok{Marx, Kapital\textquotesingle{}e şöyle başlar:}
\KeywordTok{\textbackslash{}begin}\NormalTok{\{}\ExtensionTok{quote}\NormalTok{\}}
\NormalTok{  Kapitalist üretim tarzının egemen olduğu toplumların zenginliği,}
\NormalTok{  \textasciigrave{}muazzam bir meta yığını\textquotesingle{} olarak görünür; bunun basit biçimi tek}
\NormalTok{  bir metadır. Bu nedenle, incelememiz, metanın analiziyle başlıyor.}
\KeywordTok{\textbackslash{}end}\NormalTok{\{}\ExtensionTok{quote}\NormalTok{\}}

\NormalTok{Şimdi Cemal Süreya\textquotesingle{}nın }\FunctionTok{\textbackslash{}emph}\NormalTok{\{Bu Bizimki\} şiiriyle devam edelim:}

\KeywordTok{\textbackslash{}begin}\NormalTok{\{}\ExtensionTok{verse}\NormalTok{\}}
\NormalTok{  Yıkıcı bir aşk bu,}\FunctionTok{\textbackslash{}\textbackslash{}}
\NormalTok{  Yıkıyor milletin ortasına}\FunctionTok{\textbackslash{}\textbackslash{}}
\NormalTok{  Tutku yükünü.}

\NormalTok{  Bölücü bir aşk,}\FunctionTok{\textbackslash{}\textbackslash{}}
  \FunctionTok{\textbackslash{}dots}
\KeywordTok{\textbackslash{}end}\NormalTok{\{}\ExtensionTok{verse}\NormalTok{\}}

\KeywordTok{\textbackslash{}end}\NormalTok{\{}\ExtensionTok{document}\NormalTok{\}}
\end{Highlighting}
\end{Shaded}

\hypertarget{uxf6zel-kelimeler}{%
\subsection{Özel kelimeler}\label{uxf6zel-kelimeler}}

{\LaTeX}'de bazı özel kelimelerin dizilişlerinin kelimeyi oluşturan
harflerin yazılmasıyla elde edilemeyeceğini fark etmiş olmalısınız. Bu
özel kelimeler için komutlar kullanılır.

\begin{longtable}[]{@{}lll@{}}
\toprule()
Komut & Çlktı & Gerekli Paket \\
\midrule()
\endhead
\texttt{\textbackslash{}today} & \(\textrm{\today}\) & \\
\textbackslash{} TeX & TEX & \\
\textbackslash{} LaTeX & IATEX & \\
\textbackslash{} LaTeXe & LTEX 2epsi & \\
\textbackslash{} AmS & AMS & amsmath \\
\textbackslash{} BibTeX & & dtk-logos \\
\textbackslash{} MiKTeX & & dtk-logos \\
\bottomrule()
\end{longtable}

\hypertarget{yazux131tipleri}{%
\section{Yazıtipleri}\label{yazux131tipleri}}

\hypertarget{giriux15f}{%
\subsection{Giriş}\label{giriux15f}}

Yazıtipi konusu \emph{kodlama}, \emph{aile}, \emph{biçem} ve
\emph{boyut} olmak üzere dört alt başlıkta incelenebilir. Kodlama çok
teknik bir konu olup amacımız dışındadır, ancak sadece şunu belirtelim
ki kodlama işini {\LaTeX}'de daha önce bahsettiğimiz \textbf{fontenc}
paketi üstlenir. Bu paketi belgenize eklemiş olduğunuzu varsayarak devam
edeceğiz.

Bu yazıda anlatacağımız şeylerden bazıları bir bakıma {\LaTeX}'in
felsefesine aykırı olacak. Nitekim {\LaTeX},
\texttt{\textbackslash{}documentclass} komutunda belirtilen ana yazıtipi
boyutuna göre, dipnot ya da başlık gibi ana yazıtipi boyutundan farklı
dizilen şeylerin boyutunu olabilecek en güzel ve doğru şekilde ayarlar.
O yüzden bu konudaki klasik uyarıyı biz de yineleyelim:

\includegraphics[width=1\textwidth,height=\textheight]{./images/uyari.png}

\hypertarget{aile}{%
\subsection{Aile}\label{aile}}

Yazıtipleri Roman ya da Serif, Sans Serif ve Typewriter olmak üzere üç
ailede toplanabilir. Roman ailesi tırnaklı ya da süslü diyebileceğimiz
yazıtiplerini, Sans Serif ailesi tırnaksız ya da süssüz yazıtiplerini ve
Typewriter ailesi de daktilo yazıtiplerini barındırır.

{\LaTeX}'de her belge sınıfı varsayılan yazıtipi ailesiyle gelir.
\texttt{beamer} sınıfının varsayılan ailesi Sans Serif olup, diğer
sınıfların varsayılan ailesi Roman'dır.

Varsayılan aile \texttt{\textbackslash{}familydefault} komutunda saklı
olup, \texttt{\textbackslash{}renewcommand} komutuyla değiştirilebilir.

\begin{Shaded}
\begin{Highlighting}[]
\FunctionTok{\textbackslash{}renewcommand}\NormalTok{\{}\ExtensionTok{\textbackslash{}familydefault}\NormalTok{\}\{}\FunctionTok{\textbackslash{}rmdefault}\NormalTok{\}  }
\FunctionTok{\textbackslash{}renewcommand}\NormalTok{\{}\ExtensionTok{\textbackslash{}familydefault}\NormalTok{\}\{}\FunctionTok{\textbackslash{}sfdefault}\NormalTok{\}  }
\FunctionTok{\textbackslash{}renewcommand}\NormalTok{\{}\ExtensionTok{\textbackslash{}familydefault}\NormalTok{\}\{}\FunctionTok{\textbackslash{}ttdefault}\NormalTok{\} }
\end{Highlighting}
\end{Shaded}

Birinci komut sahanlığa yazılırsa, belge sınıfından bağımsız olarak
varsayılan aile Roman, ikincisi yazılırsa Sans Serif, üçüncüsü yazılırsa
Typewriter olur.

Eğer belgenin tamamının değilde bazı kelime ya da cümlelerin farklı
aileden yazılması istenirse --ki genelde böyle kullanılır-- aşağıdaki
komut ya da bildirimler kullanılır.

\begin{longtable}[]{@{}lll@{}}
\toprule()
Komut & Bildirim & Aile \\
\midrule()
\endhead
\texttt{\textbackslash{}textrm} & \texttt{\textbackslash{}rmfamily} &
\(\textrm{Roman (Serif)}\) \\
\texttt{\textbackslash{}textsf} & \texttt{\textbackslash{}sffamily} &
\(\textsf{Sans Serif}\) \\
\texttt{\textbackslash{}texttt} & \texttt{\textbackslash{}ttfamily} &
\(\texttt{Typewriter}\) \\
\bottomrule()
\end{longtable}

{\LaTeX}'de varsayılan yazıtipi Computer Modern olup, ek bir pakete
ihtiyaç duymadan kullanılabilecek yazıtipleri aşağıda gösterilmiştir.

\begin{figure}

{\centering \includegraphics[width=0.5\textwidth,height=\textheight]{./images/yazitipi2.png}

}

\caption{Roman Yazıtipleri}

\end{figure}

\begin{figure}

{\centering \includegraphics[width=0.5\textwidth,height=\textheight]{./images/yazitipi3.png}

}

\caption{Sans Serif Yazıtipleri}

\end{figure}

\begin{figure}

{\centering \includegraphics[width=0.5\textwidth,height=\textheight]{./images/yazitipi4.png}

}

\caption{Typewriter Yazıtipleri\\
}

\end{figure}

\begin{figure}

{\centering \includegraphics[width=0.25\textwidth,height=\textheight]{./images/yazitipi5.png}

}

\caption{Elyazısı}

\end{figure}

Varsayılan yazıtipleri \texttt{\textbackslash{}rmdefault},
\texttt{\textbackslash{}sfdefault} ve \texttt{\textbackslash{}ttdefault}
komutlarında saklı olup, \texttt{\textbackslash{}renewcommand} komutuyla
değiştirilebilirler.

\begin{verbatim}
\renewcommand{\rmdefault}{<kısaltma>}
\end{verbatim}

Burada \texttt{\textless{}kısaltma\textgreater{}}, tablolarda belirtilen
kısaltmalardır. Örneğin

\begin{verbatim}
\renewcommand{\rmdefault}{put}
\end{verbatim}

komutu sahanlığa yazıldığında, eğer varsayılan aile Roman ise belgenizin
ana yazıtipi Utopia olur.

Eğer tüm belgenin değil, bazı kelime ya da cümlelerin farklı yazıtipinde
yazılması istenirse \texttt{\textbackslash{}fontfamily} komutuyla
\texttt{\textbackslash{}selectfont} komutu birlikte aşağıdaki şekilde
kullanılır.

\begin{verbatim}
{\fontfamily{pbk}\selectfont Bookman yazıtipi.} Ana yazıtipi.
\end{verbatim}

Varsayılan yazıtipi paket ekleyerek de değiştirilebilir. Bu hem
pratiktir hem de bazı paketler matematiksel ifadelerin yazıtipine de
etki eder. Bu paketlerin bazıları tabloda gösterilmiştir.

\begin{figure}

{\centering \includegraphics[width=0.7\textwidth,height=\textheight]{./images/yazitipi6.png}

}

\caption{Yazıtipi değiştiren paketler}

\end{figure}

Bunların dışında beğenebileceğiniz birçok yazıtipini
\href{https://tug.org/FontCatalogue/}{{\LaTeX} Yazıtipi Kataloğu}'nda
bulabilirsiniz.

\hypertarget{biuxe7em}{%
\subsection{Biçem}\label{biuxe7em}}

Metin içinde kelimeleri bazen italik bazen de kalın dizmek
isteyebilirsiniz. Bu değişimler aşağıdaki tablodaki komut ya da
bildirimlerle yapılır.

\begin{figure}

{\centering \includegraphics[width=0.5\textwidth,height=\textheight]{./images/yazitipi7.png}

}

\caption{Yazıtipi Biçemleri}

\end{figure}

\begin{Shaded}
\begin{Highlighting}[]
\NormalTok{İzleyen kelime }\FunctionTok{\textbackslash{}textit}\NormalTok{\{italik\} }
\NormalTok{harflerle yazılmıştır.}
\NormalTok{Metnin geri kalan kısmı}
\NormalTok{normaldir.}
\end{Highlighting}
\end{Shaded}

\begin{Shaded}
\begin{Highlighting}[]
\NormalTok{İzleyen ifade \{}\FunctionTok{\textbackslash{}slshape}\NormalTok{ \{}\FunctionTok{\textbackslash{}bfseries}\NormalTok{ eğik kalındır\}\}.}
\end{Highlighting}
\end{Shaded}

\begin{Shaded}
\begin{Highlighting}[]
\NormalTok{İzleyen ifade }\FunctionTok{\textbackslash{}textit}\NormalTok{\{}\FunctionTok{\textbackslash{}textbf}\NormalTok{\{italik kalın\}\}, ama bu}
\FunctionTok{\textbackslash{}textsc}\NormalTok{\{}\FunctionTok{\textbackslash{}textit}\NormalTok{\{büyük küçük harf değil\}\}.}
\end{Highlighting}
\end{Shaded}

Eğer vurgulu metin içinde bazı kelimeler tekrar vurgulanırsa bu
kelimeler normale döner.

\begin{Shaded}
\begin{Highlighting}[]
\NormalTok{\{}\FunctionTok{\textbackslash{}em}\NormalTok{ Vurgulu metinde tekrar}
\NormalTok{vurgu yapılırsa \{}\FunctionTok{\textbackslash{}em}\NormalTok{ normale\}}
\NormalTok{döner.\}}
\end{Highlighting}
\end{Shaded}

{\LaTeX}'de vurgu yukarıdaki gibi yapılsa da altını çizerek vurgu yapmak
isteyen olabilir. Kuyruklu harflerin altı çizildiğinde varsayılan satır
aralığı değiştiğinden vurguyu bu şekilde yapmamanız daha doğrudur. Ancak
illa altını çizmek isterseniz \texttt{\textbackslash{}underline}
komutunu kullanabilirsiniz.

\hypertarget{boyut}{%
\subsection{Boyut}\label{boyut}}

Yazıtipi boyutunu değiştirmek için aşağıdaki bildirimler kullanılır.

\begin{figure}

{\centering \includegraphics[width=0.5\textwidth,height=\textheight]{./images/yazitipi8.png}

}

\caption{Yazıtipi Boyutu Değiştiren Bildirimler}

\end{figure}

\begin{Shaded}
\begin{Highlighting}[]
\NormalTok{\{}\FunctionTok{\textbackslash{}Large}\NormalTok{ Büyük\} ve}
\NormalTok{\{}\FunctionTok{\textbackslash{}scriptsize}\NormalTok{ küçük\} harfler.}
\end{Highlighting}
\end{Shaded}

Bu bildirimlerin aynı zamanda satır aralığını da değiştirdiğine dikkat
edilmelidir. Aşağıdaki iki örnekte, \texttt{\textbackslash{}par}
(paragraf) komutunun verdiğiniz yere bağlı olarak farklı sonuçlar
ürettiği gösterilmiştir. Doğru kullanım ikincisidir.

\begin{Shaded}
\begin{Highlighting}[]
\NormalTok{\{}\FunctionTok{\textbackslash{}large} 
\NormalTok{Sokrates: Platon}
\NormalTok{yalan söyleyecek}
\NormalTok{aşağıdaki cümlede.\}}\FunctionTok{\textbackslash{}par}
\end{Highlighting}
\end{Shaded}

\begin{Shaded}
\begin{Highlighting}[]
\NormalTok{\{}\FunctionTok{\textbackslash{}large}\NormalTok{ Platon: Sokrates}
\NormalTok{doğruyu söyledi}
\NormalTok{önceki cümlede.}\FunctionTok{\textbackslash{}par}\NormalTok{\}}
\end{Highlighting}
\end{Shaded}

Bu bildirimlerin etkisi belge ana yazıtipi boyutuna bağımlıdır. Mutlak
boyutlar aşağıdaki tabloda gösterilmiştir.

\begin{figure}

{\centering \includegraphics[width=0.7\textwidth,height=\textheight]{./images/yazitipi9.png}

}

\caption{Yazıtipleri Mutlak Boyutları}

\end{figure}

Bağımsız bir yazıtipi boyutu elde etmek için
\texttt{\textbackslash{}fontsize} ile
\texttt{\textbackslash{}selectfont} komutları birlikte kullanılır.

\begin{Shaded}
\begin{Highlighting}[]
\NormalTok{\{}\FunctionTok{\textbackslash{}fontsize}\NormalTok{\{\textless{}boyut\textgreater{}\}\{\textless{}aralık\textgreater{}\}}\FunctionTok{\textbackslash{}selectfont}\NormalTok{ \textless{}metin\textgreater{}\}}
\end{Highlighting}
\end{Shaded}

Buradaki \texttt{\textless{}boyut\textgreater{}} yazıtipi boyutu,
\texttt{\textless{}aralık\textgreater{}} ise satır aralığıdır. İkisinin
de ölçü birimi punto (pt) olup, temel kural, aralığın boyutun \(1.2\)
katı olmasıdır.

\begin{Shaded}
\begin{Highlighting}[]
\NormalTok{\{}\FunctionTok{\textbackslash{}fontsize}\NormalTok{\{30\}\{36\}}\FunctionTok{\textbackslash{}selectfont}
\NormalTok{Yazı tipi boyutu 30 punto,}
\NormalTok{satır aralığı 36 punto.\}}
\end{Highlighting}
\end{Shaded}

Ana yazıtipi boyutu \texttt{\textbackslash{}normalsize} komutunda saklı
olup, \texttt{\textbackslash{}renewcommand} komutuyla değiştirilebilir.

\begin{Shaded}
\begin{Highlighting}[]
\FunctionTok{\textbackslash{}renewcommand}\NormalTok{\{}\ExtensionTok{\textbackslash{}normalsize}\NormalTok{\}\{}\FunctionTok{\textbackslash{}fontsize}\NormalTok{\{30\}\{36\}}\FunctionTok{\textbackslash{}selectfont}\NormalTok{\}}
\end{Highlighting}
\end{Shaded}

Yukarıdaki komutu sahanlığa yazarsanız belgenizin ana yazıtipi boyutu 30
pt, satır aralığı ise 36 pt olur.

\hypertarget{renk-kullanma}{%
\section{Renk Kullanma}\label{renk-kullanma}}

{\LaTeX}'de renkler
\href{http://ftp.cc.uoc.gr/mirrors/CTAN/macros/latex/required/graphics/color.pdf}{\textbf{color}}
ya da
\href{http://ftp.ntua.gr/mirror/ctan/macros/latex/contrib/xcolor/xcolor.pdf}{\textbf{xcolor}}
paketleriyle kullanılır. İkinci paket daha güçlüdür.

\begin{Shaded}
\begin{Highlighting}[]
\BuiltInTok{\textbackslash{}usepackage}\NormalTok{\{}\ExtensionTok{xcolor}\NormalTok{\}}
\end{Highlighting}
\end{Shaded}

komutuyla paketi ekledikten sonra \texttt{\textbackslash{}color} ya da
\texttt{\textbackslash{}textcolor} komutları aşağıdaki şekillerde
kullanılır.

\begin{Shaded}
\begin{Highlighting}[]
\FunctionTok{\textbackslash{}color}\NormalTok{\{red\}\{Kırmızı\}}\FunctionTok{\textbackslash{}\textbackslash{}}
\NormalTok{\{}\FunctionTok{\textbackslash{}color}\NormalTok{\{blue\} Tamamı mavi\}}\FunctionTok{\textbackslash{}\textbackslash{}}
\FunctionTok{\textbackslash{}textcolor}\NormalTok{\{pink\}\{Pembe\}}
\end{Highlighting}
\end{Shaded}

Kullanılabilecek renk adları black, white, gray, darkgray, lightgray,
brown, red, green, blue, cyan, magenta, lime, olive, orange, pink,
purple, teal, violet ve yellow'dur.

Eğer \texttt{\textbackslash{}documentclass} komutunun seçeneğine
\texttt{divpsnames} yazılırsa aşağıdaki renkler de kullanılabilir duruma
gelir.

\begin{figure}

{\centering \includegraphics[width=1\textwidth,height=\textheight]{./images/renkler.png}

}

\caption{\label{fig-renkler}\textbf{dvipsnames} seçeneğiyle gelen
renkler}

\end{figure}

Bunlar da yeterli gelmiyorsa seçeneklere bir de \texttt{svgnames} yazın
ve \href{https://www.latextemplates.com/svgnames-colors}{buradaki}
yüzden fazla rengi kullanılabilir duruma getirin.

\begin{Shaded}
\begin{Highlighting}[]
\BuiltInTok{\textbackslash{}documentclass}\NormalTok{[svgnames,dvipsnames]\{}\ExtensionTok{article}\NormalTok{\}}
\end{Highlighting}
\end{Shaded}

\hypertarget{yeni-renkler-oluux15fturmak}\NormalTok{80 mavi }\FunctionTok{\textbackslash{}\%}\NormalTok{20 beyaz\}}\FunctionTok{\textbackslash{}\textbackslash{}}
\FunctionTok{\textbackslash{}color}\NormalTok{\{red!40!blue\}\{}\FunctionTok{\textbackslash{}\%}\NormalTok{40 kırmızı }\FunctionTok{\textbackslash{}\%}\NormalTok{60 mavi\} }\FunctionTok{\textbackslash{}\textbackslash{}}
\NormalTok{\{}\FunctionTok{\textbackslash{}color}\NormalTok{\{yellow!20!green!75!black\} }\FunctionTok{\textbackslash{}\%}\NormalTok{20x0.75=15 sarı}
\FunctionTok{\textbackslash{}\%}\NormalTok{(100{-}20)x0.75=60 yeşil }\FunctionTok{\textbackslash{}\%}\NormalTok{100{-}75=25 siyah\}}
\end{Highlighting}
\end{Shaded}

İkinci yol \texttt{\textbackslash{}definecolor} komutuyla renk modelleri
kullanmaktır.

\begin{longtable}[]{@{}
  >{\raggedright\arraybackslash}p{(\columnwidth - 2\tabcolsep) * \real{0.5000}}
  >{\raggedright\arraybackslash}p{(\columnwidth - 2\tabcolsep) * \real{0.5000}}@{}}
\caption{(\#tab:renkmod) Renk Modelleri}\tabularnewline
\toprule()
\begin{minipage}[b]{\linewidth}\raggedright
\textbf{Model}
\end{minipage} & \begin{minipage}[b]{\linewidth}\raggedright
\textbf{Açıklama}
\end{minipage} \\
\midrule()
\endfirsthead
\toprule()
\begin{minipage}[b]{\linewidth}\raggedright
\textbf{Model}
\end{minipage} & \begin{minipage}[b]{\linewidth}\raggedright
\textbf{Açıklama}
\end{minipage} \\
\midrule()
\endhead
gray & Grinin tonları (0-1). 0 (siyah), 1 (beyaz). Yani 0.95 çok açık
gri, 0.30 koyu gri olur. \\
rgb & Red, Green, Blue (0-1). Üç renk (kırmızı, yeșil, mavi) 0 ile 1
arasında bir sayıyla temsil edilir. \\
RGB & Red, Green, Blue (0-255). Üç renk (karmızı, yeșil, mavi) 0 ile 255
arasında bir sayıyla temsil edilir. \\
HTML & Red, Green, Blue (OO-FF). Kodlar RRGGBB şeklinde verilir. \\
cmyk & Cyan, Magenta, Yellow, Black (0-1). Dört renk (camgöbeği,
eflatun, sarı, siyah) 0 ile 1 arasında bir sayıyla temsil edilir. \\
\bottomrule()
\end{longtable}

Komut

\begin{Shaded}
\begin{Highlighting}[]
\FunctionTok{\textbackslash{}definecolor}\NormalTok{\{\textless{}isim\textgreater{}\}\{\textless{}renk modeli\textgreater{}\}\{\textless{}kod\textgreater{}\}}
\end{Highlighting}
\end{Shaded}

şeklinde verilir. Burada \texttt{\textless{}isim\textgreater{}}, sizin
daha sonra kullanmak üzere vereceğiniz bir isimdir. Kodlar için
\href{https://www.huesnap.com/color}{HueSnap} adresinden
yararlanabilirsiniz. Örneğin

\begin{Shaded}
\begin{Highlighting}[]
\FunctionTok{\textbackslash{}definecolor}\NormalTok{\{renk1\}\{gray\}\{0.50\}}
\FunctionTok{\textbackslash{}definecolor}\NormalTok{\{renk2\}\{rgb\}\{1,0.7,0.3\}}
\FunctionTok{\textbackslash{}definecolor}\NormalTok{\{renk3\}\{RGB\}\{125,32,200\}}
\FunctionTok{\textbackslash{}definecolor}\NormalTok{\{renk4\}\{HTML\}\{CC33CC\}}
\FunctionTok{\textbackslash{}definecolor}\NormalTok{\{renk5\}\{cmyk\}\{0,0.7,1,0.5\}}
\end{Highlighting}
\end{Shaded}

komutlarıyla \texttt{renk1}, \texttt{renk2}, \texttt{renk3},
\texttt{renk4} ve \texttt{renk5} adında beş adet renk tanımlanmış olur.
Bu komutlar ya sahanlığa ya da bu renkleri kullanacağınız satırdan önce
gövdeye yazılmalıdır.

Yeni oluşturduğunuz rengi tek seferlik kullanacaksanız

\begin{Shaded}
\begin{Highlighting}[]
\FunctionTok{\textbackslash{}color}\NormalTok{[\textless{}model\textgreater{}]\{\textless{}kod\textgreater{}\}}
\FunctionTok{\textbackslash{}textcolor}\NormalTok{[\textless{}model\textgreater{}]\{\textless{}kod\textgreater{}\}\{\textless{}metin\textgreater{}\}}
\end{Highlighting}
\end{Shaded}

komutlarını kullanabilirsiniz.

\begin{Shaded}
\begin{Highlighting}[]
\NormalTok{\{}\FunctionTok{\textbackslash{}color}\NormalTok{[RGB]\{102,0,51\} Bol\}}\FunctionTok{\textbackslash{}\textbackslash{}}
\FunctionTok{\textbackslash{}textcolor}\NormalTok{[RGB]\{0,76,153\}\{Çeşit bol\}}
\end{Highlighting}
\end{Shaded}

Metnin arka planını renklendirmek için \texttt{\textbackslash{}colorbox}
komutu kullanılır. Benzer \texttt{\textbackslash{}fcolorbox} komutu aynı
şeyi çizgi çizerek yapar.

\begin{Shaded}
\begin{Highlighting}[]
\FunctionTok{\textbackslash{}colorbox}\NormalTok{\{\textless{}arka plan rengi\textgreater{}\}\{\textless{}metin\textgreater{}\}}
\FunctionTok{\textbackslash{}colorbox}\NormalTok{[\textless{}renk modeli\textgreater{}]\{\textless{}arka plan rengi\textgreater{}\}\{\textless{}metin\textgreater{}\}}
\FunctionTok{\textbackslash{}fcolorbox}\NormalTok{\{\textless{}çizgi rengi\textgreater{}\}\{\textless{}arka plan rengi\textgreater{}\}\{\textless{}metin\textgreater{}\}}
\FunctionTok{\textbackslash{}fcolorbox}\NormalTok{[\textless{}renk modeli\textgreater{}]\{\textless{}çizgi rengi\textgreater{}\}\{\textless{}arka plan rengi\textgreater{}\}\{\textless{}metin\textgreater{}\}}
\end{Highlighting}
\end{Shaded}

Komutlar yukarıda gösterildiği şekilde kullanılabilir.

\begin{Shaded}
\begin{Highlighting}[]
\FunctionTok{\textbackslash{}colorbox}\NormalTok{\{Cyan\}\{Metin\}}
\FunctionTok{\textbackslash{}colorbox}\NormalTok{[rgb]\{0.4,0.4,0.5\}\{Metin\}}
\FunctionTok{\textbackslash{}fcolorbox}\NormalTok{\{red\}\{yellow\}\{Metin\}}
\FunctionTok{\textbackslash{}fcolorbox}\NormalTok{[RGB]\{0,0,0\}\{255,204,255\}\{Metin\}}
\end{Highlighting}
\end{Shaded}

Sayfaları renklendirmek için ise \texttt{\textbackslash{}pagecolor}
komutu kullanılır. Komutun zorunlu değişkeninde renk belirtilir. Tekrar
normale (beyaz) dönmek için \texttt{\textbackslash{}nopagecolor} komutu
kullanılır. Eğer bu işe yaramazsa
\texttt{\textbackslash{}pagecolor\{white\}} komutunu kullanabilirsiniz.

\bookmarksetup{startatroot}

\hypertarget{matematiksel-ifadeler}{%
\chapter{Matematiksel İfadeler}\label{matematiksel-ifadeler}}

Matematik formüllerini dizmek, kuşkusuz, {\LaTeX}'in en güçlü olduğu
konulardan biridir. Çok fazla matematiksel gösterimin varlığından dolayı
da büyük bir konudur. Bu bölümde ileri bir matematik kitabını dizmek
için gereken birçok şey anlatılacaktır ancak işin sınırları göz önüne
alındığında başka kaynaklara da başvurmanız gerekebilir.

\hypertarget{giriux15f-1}{%
\section{Giriş}\label{giriux15f-1}}

Belgenizde yalnızca birkaç basit matematiksel formül kullanacaksanız
herhangi bir pakete gerek olmadan yazabilirsiniz. Ancak çok sayıda
karmaşık formül içeren bilimsel bir belge yazma niyetindeyseniz temel
\href{https://www.ams.org/publications/authors/tex/amslatex}{AMS}
paketlerini kullanmanız gerekir. Bu paketler
\href{http://ftp.ntua.gr/mirror/ctan/macros/latex/required/amsmath/amsmath.pdf}{amsmath}
,
\href{https://texdoc.net/texmf-dist/doc/fonts/amsfonts/amssymb.pdf}{amssymb}
ve
\href{http://ftp.ntua.gr/mirror/ctan/fonts/amsfonts/doc/amsfonts.pdf}{amsfonts}'dir.

\begin{Shaded}
\begin{Highlighting}[]
\BuiltInTok{\textbackslash{}usepackage}\NormalTok{\{}\ExtensionTok{amsmath,amssymb,amsfonts}\NormalTok{\}}
\end{Highlighting}
\end{Shaded}

Yukarıdaki komutu sahanlığa yazarak paketleri belgenize ekleyiniz.
Bundan sonra bu paketleri eklediğinizi varsayarak devam edeceğiz ve
bunların dışında bir pakete ihtiyaç duyarsak ayrıca belirteceğiz.

\hypertarget{genel}{%
\subsection{Genel}\label{genel}}

Belgenizin metnini oluştururken {\LaTeX}'in metnin ne zaman matematiksel
olduğunu bilmesi gerekir. Bunun nedeni, {\LaTeX}'in matematiksel
ifadeleri normal metinden farklı bir şekilde dizmesidir. Bu nedenle
matematiksel ifadeler, normal metinden farklı olarak bazı ortamlarda
girilirler.

Matematik özel ortamlar gerektirdiğinden, doğal olarak standart şekilde
kullanabileceğiniz uygun ortam adları vardır. Bununla birlikte, diğer
ortamların çoğundan farklı olarak, formülünüzü bildirmek için bazı
kullanışlı kısaltmalar vardır. {\LaTeX}'de bu ortamlar ya da kısaltmalar
kullanılarak formüller iki türlü dizilir:

\begin{itemize}
\tightlist
\item
  Formüller satırın içinde, yani bildirildiği metnin gövdesi içine
  yazılır:
  \(\lim_{n \to \infty} \sum_{k=1}^n \frac{1}{k^2} =\frac{\pi^2}{6}\).
  Görüldüğü gibi {\LaTeX}, paragraf yapısını bozmamak için sembolleri
  olabildiğince sıkıştırır ve gerek görürse alttakileri yana kaydırır.
\item
  Formüller ayrı bir satırda tek başlarına tüm detaylarıyla sergilenir:
  \[\lim_{n \to \infty} \sum_{k=1}^n
  \frac{1}{k^2} =\frac{\pi^2}{6}.\]
\end{itemize}

Formülün satır içerisinde dizilmesi için ya \texttt{\$...\$} arasına
arasına, ya \texttt{\textbackslash{}(...\textbackslash{})} arasına ya da
\texttt{\textbackslash{}begin\{math\}} ile
\texttt{\textbackslash{}end\{math\}}arasına, yani \texttt{math}
ortamında yazılması gerekir. Üçü de aynı sonucu verir.

Formülün sergilenmesi içinse ya
\texttt{\textbackslash{}{[}...\textbackslash{}{]}} arasına ya
\texttt{displaymath} ortamında ya da \texttt{equation} ortamında
yazılması gerekir. \texttt{equation}ortamında yazılan formülü {\LaTeX}
otomatik numaralandırır. Numara verilmesini istemezseniz ortamı
\texttt{equation*} şeklinde kullanmanız gerekir.

\begin{tcolorbox}[enhanced jigsaw, opacitybacktitle=0.6, coltitle=black, leftrule=.75mm, rightrule=.15mm, toprule=.15mm, bottomtitle=1mm, titlerule=0mm, colbacktitle=quarto-callout-note-color!10!white, breakable, arc=.35mm, opacityback=0, colframe=quarto-callout-note-color-frame, toptitle=1mm, title=\textcolor{quarto-callout-note-color}{\faInfo}\hspace{0.5em}{Not}, bottomrule=.15mm, left=2mm, colback=white]
{\TeX}'in eski sürümlerinde formüller, sergilenmeleri için
\texttt{\$\$...\$\$} arasına yazılırdı. Bu kullanım hala geçerlidir
ancak bazı sorunlara yol açabildiğinden (örneğin belge seçeneğine
\texttt{fleqn} yazıldığında) kullanımı önerilmez.
\end{tcolorbox}

\begin{Shaded}
\begin{Highlighting}[]
\BuiltInTok{\textbackslash{}documentclass}\NormalTok{\{}\ExtensionTok{article}\NormalTok{\}}
\BuiltInTok{\textbackslash{}usepackage}\NormalTok{[T1]\{}\ExtensionTok{fontenc}\NormalTok{\}}
\BuiltInTok{\textbackslash{}usepackage}\NormalTok{[turkish]\{}\ExtensionTok{babel}\NormalTok{\}}
\BuiltInTok{\textbackslash{}usepackage}\NormalTok{\{}\ExtensionTok{amsmath,amssymb,amsfonts}\NormalTok{\}}
\KeywordTok{\textbackslash{}begin}\NormalTok{\{}\ExtensionTok{document}\NormalTok{\}}

 \SpecialStringTok{$x}\SpecialCharTok{\textbackslash{}in}\SpecialStringTok{ }\SpecialCharTok{\textbackslash{}mathbb}\SpecialStringTok{\{R\}$}\NormalTok{ için }\SpecialStringTok{$|x|\textless{}1$}\NormalTok{ ise}
\KeywordTok{\textbackslash{}begin}\NormalTok{\{}\ExtensionTok{equation*}\NormalTok{\}}
\SpecialStringTok{  {-}1\textless{}x\textless{}1}
\KeywordTok{\textbackslash{}end}\NormalTok{\{}\ExtensionTok{equation*}\NormalTok{\} olur.}

\KeywordTok{\textbackslash{}end}\NormalTok{\{}\ExtensionTok{document}\NormalTok{\}}
\end{Highlighting}
\end{Shaded}

Numara verilen bir formülü \texttt{\textbackslash{}label} komutuyla
etiketleyip, \texttt{\textbackslash{}ref} ya da
\texttt{\textbackslash{}eqref} komutuyla formüle atıf yapılabilir. Atıf
\texttt{\textbackslash{}eqref} komutuyla yapılırsa formülün numarası
parantez içinde yazılır.

\begin{Shaded}
\begin{Highlighting}[]
\BuiltInTok{\textbackslash{}documentclass}\NormalTok{\{}\ExtensionTok{article}\NormalTok{\}}
\BuiltInTok{\textbackslash{}usepackage}\NormalTok{[T1]\{}\ExtensionTok{fontenc}\NormalTok{\}}
\BuiltInTok{\textbackslash{}usepackage}\NormalTok{[turkish]\{}\ExtensionTok{babel}\NormalTok{\}}
\BuiltInTok{\textbackslash{}usepackage}\NormalTok{\{}\ExtensionTok{amsmath,amssymb,amsfonts}\NormalTok{\}}
\KeywordTok{\textbackslash{}begin}\NormalTok{\{}\ExtensionTok{document}\NormalTok{\}}

\KeywordTok{\textbackslash{}begin}\NormalTok{\{}\ExtensionTok{equation}\NormalTok{\}}
\SpecialCharTok{\textbackslash{}label}\SpecialStringTok{\{eq:euler\}}
\SpecialStringTok{  e\^{}\{i}\SpecialCharTok{\textbackslash{}pi}\SpecialStringTok{\}+1=0}
\KeywordTok{\textbackslash{}end}\NormalTok{\{}\ExtensionTok{equation}\NormalTok{\}}
\NormalTok{Euler\textquotesingle{}in }\KeywordTok{\textbackslash{}eqref}\NormalTok{\{}\ExtensionTok{eq:euler}\NormalTok{\} formülüne göre}\FunctionTok{\textbackslash{}dots}

\KeywordTok{\textbackslash{}end}\NormalTok{\{}\ExtensionTok{document}\NormalTok{\}}
\end{Highlighting}
\end{Shaded}

\hypertarget{matematik-kipiyle-metin-kipi-arasux131ndaki-farklar}{%
\subsection{Matematik kipiyle metin kipi arasındaki
farklar}\label{matematik-kipiyle-metin-kipi-arasux131ndaki-farklar}}

Matematiksel ifadeleri girerken düz metin girişinden farklı olarak
dikkat edilmesi gereken bazı noktalar vardır:

\begin{enumerate}
\def\labelenumi{\arabic{enumi}.}
\tightlist
\item
  Boşlukların ve satır kesmelerinin çoğunun önemi yoktur, çünkü tüm
  boşluklar ya matematiksel ifadelerden mantıksal olarak türetilir, ya
  da özel komutlarla belirtilmesi gerekir.
\item
  Boş satırlara izin verilmez.
\item
  Aksanlı harfler kullanılmaz.
\item
  Her harf matematiksel bir değişken olarak kabul edilir ve italik
  dizilir. Eğer düz yazıyla ve normal aralıklarla bir metin girilecekse
  \texttt{\textbackslash{}textrm\{\}} ya da
  \texttt{\textbackslash{}text\{\}} komutları kullanılmalıdır. Bu
  komutlarla metin kipine geçiş yapılmış olur ve metin artık düz ve
  normal aralıklarla dizilir. İtalik ve normal aralıklarla metin
  girilecekse \texttt{\textbackslash{}textit\{\}} komutu kullanılabilir.
  Ayrıca bu komutlarla aksanlı harfler de kullanılabilir.
\end{enumerate}

Aşağıdaki örnek matematik kipi ile metin kipi arasındaki farkları
gösterir.

\begin{Shaded}
\begin{Highlighting}[]
\BuiltInTok{\textbackslash{}documentclass}\NormalTok{\{}\ExtensionTok{article}\NormalTok{\}}
\BuiltInTok{\textbackslash{}usepackage}\NormalTok{[T1]\{}\ExtensionTok{fontenc}\NormalTok{\}}
\BuiltInTok{\textbackslash{}usepackage}\NormalTok{[turkish]\{}\ExtensionTok{babel}\NormalTok{\}}
\BuiltInTok{\textbackslash{}usepackage}\NormalTok{\{}\ExtensionTok{amsmath,amssymb,amsfonts}\NormalTok{\}}
\KeywordTok{\textbackslash{}begin}\NormalTok{\{}\ExtensionTok{document}\NormalTok{\}}
 \SpecialStringTok{$x\^{}2{-}1=0 }\SpecialCharTok{\textbackslash{}text}\NormalTok{\{ise }\SpecialStringTok{$x=\textbackslash{}pm 1$}\NormalTok{\}}\SpecialStringTok{.$}
\KeywordTok{\textbackslash{}end}\NormalTok{\{}\ExtensionTok{document}\NormalTok{\}}
\end{Highlighting}
\end{Shaded}

\hypertarget{gruplandux131rma}{%
\subsection{Gruplandırma}\label{gruplandux131rma}}

Formülleri dizerken dikkat edilmesi gereken noktalardan biri komutların
çoğunun kendisinden sonra ilk gelen karaktere etki etmesidir. Bu yüzden
bir komutun çok sayıda karaktere etki etmesi istenirse bu karakterler
iki çengelli parantez \texttt{\{…\}} arasına yazılarak
gruplandırılmalıdır.

\begin{Shaded}
\begin{Highlighting}[]
\BuiltInTok{\textbackslash{}documentclass}\NormalTok{\{}\ExtensionTok{article}\NormalTok{\}}
\BuiltInTok{\textbackslash{}usepackage}\NormalTok{[T1]\{}\ExtensionTok{fontenc}\NormalTok{\}}
\BuiltInTok{\textbackslash{}usepackage}\NormalTok{[turkish]\{}\ExtensionTok{babel}\NormalTok{\}}
\BuiltInTok{\textbackslash{}usepackage}\NormalTok{\{}\ExtensionTok{amsmath,amssymb,amsfonts}\NormalTok{\}}
\KeywordTok{\textbackslash{}begin}\NormalTok{\{}\ExtensionTok{document}\NormalTok{\}}
 \SpecialStringTok{$a\^{}x+y }\SpecialCharTok{\textbackslash{}neq}\SpecialStringTok{ a\^{}\{x+y\}$}
\KeywordTok{\textbackslash{}end}\NormalTok{\{}\ExtensionTok{document}\NormalTok{\}}
\end{Highlighting}
\end{Shaded}

\hypertarget{boux15fluklar-1}{%
\subsection{Boşluklar}\label{boux15fluklar-1}}

Bazen {\LaTeX} formülleri dizerken olması gerektiği gibi boşluk bırakma
konusunda yetersiz kalabilir. Bu durumda boşluklar elle oluşturulur.
Boşluklar için kullanılabilecek komutlar tablodaki gibidir.

\begin{longtable}[]{@{}ll@{}}
\toprule()
\endhead
Negatif & \texttt{\textbackslash{}!} \\
İnce & \texttt{\textbackslash{},} \\
Orta & \texttt{\textbackslash{}:} \\
Kalın & \texttt{\textbackslash{};} \\
Sözcük arası & \texttt{\textbackslash{}} \\
Bir quad & \texttt{\textbackslash{}quad} \\
İki quad & \texttt{\textbackslash{}qquad} \\
\bottomrule()
\end{longtable}

Örneğin \texttt{\textbackslash{},} komutunun bıraktığı ince boşluk bazı
formüllerde çok kullanışlıdır.

\begin{Shaded}
\begin{Highlighting}[]
\BuiltInTok{\textbackslash{}documentclass}\NormalTok{\{}\ExtensionTok{article}\NormalTok{\}}
\BuiltInTok{\textbackslash{}usepackage}\NormalTok{[T1]\{}\ExtensionTok{fontenc}\NormalTok{\}}
\BuiltInTok{\textbackslash{}usepackage}\NormalTok{[turkish]\{}\ExtensionTok{babel}\NormalTok{\}}
\BuiltInTok{\textbackslash{}usepackage}\NormalTok{\{}\ExtensionTok{amsmath,amssymb,amsfonts}\NormalTok{\}}
\KeywordTok{\textbackslash{}begin}\NormalTok{\{}\ExtensionTok{document}\NormalTok{\}}

\SpecialStringTok{\textbackslash{}[}
\SpecialCharTok{\textbackslash{}int}\SpecialStringTok{\_a\^{}b f(x) dx }\SpecialCharTok{\textbackslash{}quad}\SpecialStringTok{ }\SpecialCharTok{\textbackslash{}sqrt}\SpecialStringTok{\{2\} a}
\SpecialCharTok{\textbackslash{}quad}\SpecialStringTok{ }\SpecialCharTok{\textbackslash{}sqrt}\SpecialStringTok{\{}\SpecialCharTok{\textbackslash{}log}\SpecialStringTok{ x\}}
\SpecialStringTok{\textbackslash{}]}
\SpecialStringTok{\textbackslash{}[}
\SpecialCharTok{\textbackslash{}int}\SpecialStringTok{\_a\^{}b f(x)}\SpecialCharTok{\textbackslash{},}\SpecialStringTok{dx }\SpecialCharTok{\textbackslash{}quad}
\SpecialCharTok{\textbackslash{}sqrt}\SpecialStringTok{\{2\}}\SpecialCharTok{\textbackslash{},}\SpecialStringTok{a }\SpecialCharTok{\textbackslash{}quad}\SpecialStringTok{ }\SpecialCharTok{\textbackslash{}sqrt}\SpecialStringTok{\{}\SpecialCharTok{\textbackslash{},\textbackslash{}log}\SpecialStringTok{ x\}}
\SpecialStringTok{\textbackslash{}]}

\KeywordTok{\textbackslash{}end}\NormalTok{\{}\ExtensionTok{document}\NormalTok{\}}
\end{Highlighting}
\end{Shaded}

Negatif aralık bırakan \texttt{\textbackslash{}!} komutu da fazla
aralıklı ifadeleri birbirine yaklaştırmak için kullanılır.

\begin{Shaded}
\begin{Highlighting}[]
\BuiltInTok{\textbackslash{}documentclass}\NormalTok{\{}\ExtensionTok{article}\NormalTok{\}}
\BuiltInTok{\textbackslash{}usepackage}\NormalTok{[T1]\{}\ExtensionTok{fontenc}\NormalTok{\}}
\BuiltInTok{\textbackslash{}usepackage}\NormalTok{[turkish]\{}\ExtensionTok{babel}\NormalTok{\}}
\BuiltInTok{\textbackslash{}usepackage}\NormalTok{\{}\ExtensionTok{amsmath,amssymb,amsfonts}\NormalTok{\}}
\KeywordTok{\textbackslash{}begin}\NormalTok{\{}\ExtensionTok{document}\NormalTok{\}}

\SpecialStringTok{\textbackslash{}[}
\SpecialStringTok{x\^{}2/2 }\SpecialCharTok{\textbackslash{}quad}\SpecialStringTok{ a/}\SpecialCharTok{\textbackslash{}sin}\SpecialStringTok{ b}
\SpecialStringTok{\textbackslash{}]}
\SpecialStringTok{\textbackslash{}[}
\SpecialStringTok{x\^{}2}\SpecialCharTok{\textbackslash{}!}\SpecialStringTok{/2 }\SpecialCharTok{\textbackslash{}quad}\SpecialStringTok{ a/}\SpecialCharTok{\textbackslash{}!\textbackslash{}sin}\SpecialStringTok{ b}
\SpecialStringTok{\textbackslash{}]}

\KeywordTok{\textbackslash{}end}\NormalTok{\{}\ExtensionTok{document}\NormalTok{\}}
\end{Highlighting}
\end{Shaded}

\hypertarget{parantezler-gruplandux131rux131cux131lar-ve-oklar}{%
\section{Parantezler, Gruplandırıcılar ve
Oklar}\label{parantezler-gruplandux131rux131cux131lar-ve-oklar}}

{\LaTeX}'de her türlü parantez ve gruplandırıcı kullanılabilir. Yuvarlak
ve köşeli parantezler klavyedeki yerlerinden, çengelli parantez ise
\texttt{\textbackslash{}\{} ve \texttt{\textbackslash{}\}} komutları
kullanılarak girilir.

\begin{Shaded}
\begin{Highlighting}[]
\BuiltInTok{\textbackslash{}documentclass}\NormalTok{\{}\ExtensionTok{article}\NormalTok{\}}
\BuiltInTok{\textbackslash{}usepackage}\NormalTok{[T1]\{}\ExtensionTok{fontenc}\NormalTok{\}}
\BuiltInTok{\textbackslash{}usepackage}\NormalTok{[turkish]\{}\ExtensionTok{babel}\NormalTok{\}}
\BuiltInTok{\textbackslash{}usepackage}\NormalTok{\{}\ExtensionTok{amsmath,amssymb,amsfonts}\NormalTok{\}}
\KeywordTok{\textbackslash{}begin}\NormalTok{\{}\ExtensionTok{document}\NormalTok{\}}

\SpecialStringTok{$\{a,b,c\}}\SpecialCharTok{\textbackslash{}neq\textbackslash{}\{}\SpecialStringTok{a,b,c}\SpecialCharTok{\textbackslash{}\}}\SpecialStringTok{$}

\KeywordTok{\textbackslash{}end}\NormalTok{\{}\ExtensionTok{document}\NormalTok{\}}
\end{Highlighting}
\end{Shaded}

Kullanılabilecek tüm gruplandırıcı işaretler Tablo~\ref{tbl-grup}'de
gösterilmiştir.

\hypertarget{tbl-grup}{}
\begin{longtable}[]{@{}
  >{\raggedright\arraybackslash}p{(\columnwidth - 4\tabcolsep) * \real{0.3077}}
  >{\raggedright\arraybackslash}p{(\columnwidth - 4\tabcolsep) * \real{0.3846}}
  >{\raggedright\arraybackslash}p{(\columnwidth - 4\tabcolsep) * \real{0.3077}}@{}}
\caption{\label{tbl-grup}Gruplandırıcılar}\tabularnewline
\toprule()
\endhead
( \texttt{(} ) \texttt{)} & \(\uparrow\)
\texttt{\textbackslash{}uparrow} & \([\) \texttt{{[}} ya da
\texttt{\textbackslash{}lbrack} \\
\(]\) \texttt{{]}} ya da \texttt{\textbackslash{}rbrack} &
\(\downarrow\) \texttt{\textbackslash{}downarrow} & \(\lbrace\)
\texttt{\textbackslash{}\{} ya da \texttt{\textbackslash{}lbrace} \\
\(\rbrace\) \texttt{\textbackslash{}\}} ya da
\texttt{\textbackslash{}rbrace} & \(\updownarrow\)
\texttt{\textbackslash{}updownarrow} & \(\langle\)
\texttt{\textbackslash{}langle} \\
\(\rangle\) \texttt{\textbackslash{}rangle} & \(\vert\)
\texttt{\textbar{}} ya da \texttt{\textbackslash{}vert} & \(\lfloor\)
\texttt{\textbackslash{}lfloor} \\
\(\rfloor\) \texttt{\textbackslash{}rfloor} & \(\lceil\)
\texttt{\textbackslash{}lceil} & \(\rceil\)
\texttt{\textbackslash{}rceil} \\
\(/\) \texttt{/} & \(\backslash\) \texttt{\textbackslash{}backslash} &
\(\Updownarrow\) \texttt{\textbackslash{}Updownarrow} \\
\bottomrule()
\end{longtable}

Grup açıcı bir sembolün önüne \texttt{\textbackslash{}left} komutu, grup
kapatıcı bir sembolün önüne de \texttt{\textbackslash{}right} komutu
yazılırsa {\LaTeX} onları en uygun boyda dizer. Her bir
\texttt{\textbackslash{}left} komutuna karşılık mutlaka bir
\texttt{\textbackslash{}right} komutu bulunmalıdır. Bunların doğru boyda
dizilmesi için iki komutunda aynı satırda yer almasına dikkat
edilmelidir. Sol/sağ tarafta gruplandırıcı bir işaret istenmiyorsa,
görünmeyen
\texttt{\textbackslash{}left.}/\texttt{\textbackslash{}right.} komutu
kullanılır.

\begin{Shaded}
\begin{Highlighting}[]
\BuiltInTok{\textbackslash{}documentclass}\NormalTok{\{}\ExtensionTok{article}\NormalTok{\}}
\BuiltInTok{\textbackslash{}usepackage}\NormalTok{[T1]\{}\ExtensionTok{fontenc}\NormalTok{\}}
\BuiltInTok{\textbackslash{}usepackage}\NormalTok{[turkish]\{}\ExtensionTok{babel}\NormalTok{\}}
\BuiltInTok{\textbackslash{}usepackage}\NormalTok{\{}\ExtensionTok{amsmath,amssymb,amsfonts}\NormalTok{\}}
\KeywordTok{\textbackslash{}begin}\NormalTok{\{}\ExtensionTok{document}\NormalTok{\}}

\SpecialStringTok{\textbackslash{}[}
\SpecialCharTok{\textbackslash{}left}\SpecialStringTok{(1+}\SpecialCharTok{\textbackslash{}frac}\SpecialStringTok{\{1\}\{n\}}\SpecialCharTok{\textbackslash{}right}\SpecialStringTok{)\^{}n}\SpecialCharTok{\textbackslash{}quad}
\SpecialCharTok{\textbackslash{}left}\SpecialStringTok{.}\SpecialCharTok{\textbackslash{}frac}\SpecialStringTok{\{x\^{}3\}\{3\}}\SpecialCharTok{\textbackslash{}right}\SpecialStringTok{|\_0\^{}1}
\SpecialStringTok{\textbackslash{}]}

\KeywordTok{\textbackslash{}end}\NormalTok{\{}\ExtensionTok{document}\NormalTok{\}}
\end{Highlighting}
\end{Shaded}

Bazen gruplandırıcı sembolün boyunu elle ayarlamak gerekebilir. Bunun
için, gruplandırıcı komutun önüne \texttt{\textbackslash{}big},
\texttt{\textbackslash{}Big}, \texttt{\textbackslash{}bigg} veya
\texttt{\textbackslash{}Bigg} komutlarından biri verilir.
\texttt{\textbackslash{}bigl} (büyük sol) ve
\texttt{\textbackslash{}bigr} (büyük sağ) komutları da parantezleri
biraz büyütür.

\begin{Shaded}
\begin{Highlighting}[]
\BuiltInTok{\textbackslash{}documentclass}\NormalTok{\{}\ExtensionTok{article}\NormalTok{\}}
\BuiltInTok{\textbackslash{}usepackage}\NormalTok{[T1]\{}\ExtensionTok{fontenc}\NormalTok{\}}
\BuiltInTok{\textbackslash{}usepackage}\NormalTok{[turkish]\{}\ExtensionTok{babel}\NormalTok{\}}
\BuiltInTok{\textbackslash{}usepackage}\NormalTok{\{}\ExtensionTok{amsmath,amssymb,amsfonts}\NormalTok{\}}
\KeywordTok{\textbackslash{}begin}\NormalTok{\{}\ExtensionTok{document}\NormalTok{\}}

\SpecialStringTok{\textbackslash{}[}
\SpecialCharTok{\textbackslash{}big}\SpecialStringTok{(}\SpecialCharTok{\textbackslash{}Big}\SpecialStringTok{(}\SpecialCharTok{\textbackslash{}bigg}\SpecialStringTok{(}\SpecialCharTok{\textbackslash{}Bigg}\SpecialStringTok{(}\SpecialCharTok{\textbackslash{}quad}
\SpecialCharTok{\textbackslash{}big\textbackslash{}\}\textbackslash{}Big\textbackslash{}\}\textbackslash{}bigg\textbackslash{}\}\textbackslash{}Bigg\textbackslash{}\}}
\SpecialCharTok{\textbackslash{}quad}
\SpecialCharTok{\textbackslash{}big\textbackslash{}|\textbackslash{}Big\textbackslash{}|\textbackslash{}bigg\textbackslash{}|\textbackslash{}Bigg\textbackslash{}|}
\SpecialStringTok{\textbackslash{}]}

\KeywordTok{\textbackslash{}end}\NormalTok{\{}\ExtensionTok{document}\NormalTok{\}}
\end{Highlighting}
\end{Shaded}

Oklar için Tablo~\ref{tbl-oklar}'deki komutlar kullanılır.

\hypertarget{tbl-oklar}{}
\begin{longtable}[]{@{}
  >{\raggedright\arraybackslash}p{(\columnwidth - 6\tabcolsep) * \real{0.2500}}
  >{\raggedright\arraybackslash}p{(\columnwidth - 6\tabcolsep) * \real{0.2500}}
  >{\raggedright\arraybackslash}p{(\columnwidth - 6\tabcolsep) * \real{0.2500}}
  >{\raggedright\arraybackslash}p{(\columnwidth - 6\tabcolsep) * \real{0.2500}}@{}}
\caption{\label{tbl-oklar}Oklar}\tabularnewline
\toprule()
\endhead
\(\leftarrow\) \texttt{\textbackslash{}leftarrow} ya da
\texttt{\textbackslash{}gets} & \(\longleftarrow\)
\texttt{\textbackslash{}longleftarrow} & \(\uparrow\)
\texttt{\textbackslash{}uparrow} & \(\Leftarrow\)
\texttt{\textbackslash{}Leftarrow} \\
\(\Longleftarrow\) \texttt{\textbackslash{}Longleftarrow} & \(\Uparrow\)
\texttt{\textbackslash{}Uparrow} & \(\rightarrow\)
\texttt{\textbackslash{}rightarrow} ya da \texttt{\textbackslash{}to} &
\(\longrightarrow\) \texttt{\textbackslash{}longrightarrow} \\
\(\downarrow\) \texttt{\textbackslash{}downarrow} & \(\Rightarrow\)
\texttt{\textbackslash{}Rightarrow} & \(\Longrightarrow\)
\texttt{\textbackslash{}Longrightarrow} & \(\Downarrow\)
\texttt{\textbackslash{}Downarrow} \\
\(\leftrightarrow\) \texttt{\textbackslash{}leftrightarrow} &
\(\longleftrightarrow\) \texttt{\textbackslash{}longleftrightarrow} &
\(\updownarrow\) \texttt{\textbackslash{}updownarrow} &
\(\Leftrightarrow\) \texttt{\textbackslash{}Leftrightarrow} \\
\(\Longleftrightarrow\) \texttt{\textbackslash{}Longleftrightarrow} &
\(\Updownarrow\) \texttt{\textbackslash{}Updownarrow} & \(\mapsto\)
\texttt{\textbackslash{}mapsto} & \(\longmapsto\)
\texttt{\textbackslash{}longmapsto} \\
\(\nearrow\) \texttt{\textbackslash{}nearrow} & \(\hookleftarrow\)
\texttt{\textbackslash{}hookleftarrow} & \(\hookrightarrow\)
\texttt{\textbackslash{}hookrightarrow} & \(\searrow\)
\texttt{\textbackslash{}searrow} \\
\(\leftharpoonup\) \texttt{\textbackslash{}leftharpoonup} &
\(\rightharpoonup\) \texttt{\textbackslash{}rightharpoonup} &
\(\swarrow\) \texttt{\textbackslash{}swarrow} & \(\leftharpoondown\)
\texttt{\textbackslash{}leftharpoondown} \\
\(\rightharpoondown\) \texttt{\textbackslash{}rightharpoondown} &
\(\nwarrow\) \texttt{\textbackslash{}nwarrow} & \(\rightleftharpoons\)
\texttt{\textbackslash{}rightleftharpoons} & \(\leadsto\)
\texttt{\textbackslash{}leadsto} \\
\bottomrule()
\end{longtable}

\begin{Shaded}
\begin{Highlighting}[]
\BuiltInTok{\textbackslash{}documentclass}\NormalTok{\{}\ExtensionTok{article}\NormalTok{\}}
\BuiltInTok{\textbackslash{}usepackage}\NormalTok{[T1]\{}\ExtensionTok{fontenc}\NormalTok{\}}
\BuiltInTok{\textbackslash{}usepackage}\NormalTok{[turkish]\{}\ExtensionTok{babel}\NormalTok{\}}
\BuiltInTok{\textbackslash{}usepackage}\NormalTok{\{}\ExtensionTok{amsmath,amssymb,amsfonts}\NormalTok{\}}
\KeywordTok{\textbackslash{}begin}\NormalTok{\{}\ExtensionTok{document}\NormalTok{\}}

\SpecialStringTok{\textbackslash{}(}
\SpecialCharTok{\textbackslash{}downarrow}
\SpecialCharTok{\textbackslash{}big\textbackslash{}downarrow}
\SpecialCharTok{\textbackslash{}Big\textbackslash{}downarrow}
\SpecialCharTok{\textbackslash{}bigg\textbackslash{}downarrow}
\SpecialCharTok{\textbackslash{}Bigg\textbackslash{}downarrow}
\SpecialStringTok{\textbackslash{})}

\KeywordTok{\textbackslash{}end}\NormalTok{\{}\ExtensionTok{document}\NormalTok{\}}
\end{Highlighting}
\end{Shaded}

Bunların dışında altlarına ya da üstlerine matematiksel ifadeler
yazılabilen \texttt{\textbackslash{}xleftarrow} ve
\texttt{\textbackslash{}xrightarrow} komutları vardır.

\begin{Shaded}
\begin{Highlighting}[]
\BuiltInTok{\textbackslash{}documentclass}\NormalTok{\{}\ExtensionTok{article}\NormalTok{\}}
\BuiltInTok{\textbackslash{}usepackage}\NormalTok{[T1]\{}\ExtensionTok{fontenc}\NormalTok{\}}
\BuiltInTok{\textbackslash{}usepackage}\NormalTok{[turkish]\{}\ExtensionTok{babel}\NormalTok{\}}
\BuiltInTok{\textbackslash{}usepackage}\NormalTok{\{}\ExtensionTok{amsmath,amssymb,amsfonts}\NormalTok{\}}
\KeywordTok{\textbackslash{}begin}\NormalTok{\{}\ExtensionTok{document}\NormalTok{\}}

\SpecialStringTok{\textbackslash{}(}
\SpecialCharTok{\textbackslash{}xleftarrow}\SpecialStringTok{\{a\}}
\SpecialCharTok{\textbackslash{}xrightarrow}\SpecialStringTok{[X]\{a+b\}}
\SpecialStringTok{\textbackslash{})}

\KeywordTok{\textbackslash{}end}\NormalTok{\{}\ExtensionTok{document}\NormalTok{\}}
\end{Highlighting}
\end{Shaded}

\hypertarget{yunan-harfleri}{%
\section{Yunan Harfleri}\label{yunan-harfleri}}

Yunan harfleri matematikte yaygın olarak kullanılır. Bu harfler ters
eğik çizgiden sonra harfin adı yazılarak elde edilir. Eğer ilk harf
küçük ise küçük, büyükse de büyük harf elde edilir. Bazı büyük Yunanca
harfler Latin harfleri gibi göründüğünden (örneğin, büyük harf Alpha ve
Beta yalnızca sırasıyla ``A'' ve ``B''dir) ayrıca tanımlanmamışlardır.
Küçük harf epsilon, theta, kappa, phi, pi, rho ve sigma iki farklı
sürümde sunulmaktadır. Alternatif sürümü, harf adının önüne ``var''
eklenerek oluşturulur.

\hypertarget{tbl-yunan}{}
\begin{longtable}[]{@{}
  >{\raggedright\arraybackslash}p{(\columnwidth - 6\tabcolsep) * \real{0.2500}}
  >{\raggedright\arraybackslash}p{(\columnwidth - 6\tabcolsep) * \real{0.2500}}
  >{\raggedright\arraybackslash}p{(\columnwidth - 6\tabcolsep) * \real{0.2500}}
  >{\raggedright\arraybackslash}p{(\columnwidth - 6\tabcolsep) * \real{0.2500}}@{}}
\caption{\label{tbl-yunan}Yunan Harfleri}\tabularnewline
\toprule()
\endhead
\(\alpha\) \texttt{\textbackslash{}alpha} & \(\theta\)
\texttt{\textbackslash{}theta} & \(o\) \texttt{o} & \(\tau\)
\texttt{\textbackslash{}tau} \\
\(\beta\) \texttt{\textbackslash{}beta} & \(\vartheta\)
\texttt{\textbackslash{}vartheta} & \(\pi\) \texttt{\textbackslash{}pi}
& \(\upsilon\) \texttt{\textbackslash{}upsilon} \\
\(\gamma\) \texttt{\textbackslash{}gamma} & \(\iota\)
\texttt{\textbackslash{}iota} & \(\varpi\)
\texttt{\textbackslash{}varpi} & \(\phi\)
\texttt{\textbackslash{}phi} \\
\(\delta\) \texttt{\textbackslash{}delta} & \(\kappa\)
\texttt{\textbackslash{}kappa} & \(\rho\) \texttt{\textbackslash{}rho} &
\(\varphi\) \texttt{\textbackslash{}varphi} \\
\(\epsilon\) \texttt{\textbackslash{}epsilon} & \(\lambda\)
\texttt{\textbackslash{}lambda} & \(\varrho\)
\texttt{\textbackslash{}varrho} & \(\chi\)
\texttt{\textbackslash{}chi} \\
\(\varepsilon\) \texttt{\textbackslash{}varepsilon} & \(\mu\)
\texttt{\textbackslash{}mu} & \(\sigma\) \texttt{\textbackslash{}sigma}
& \(\psi\) \texttt{\textbackslash{}psi} \\
\(\zeta\) \texttt{\textbackslash{}zeta} & \(\nu\)
\texttt{\textbackslash{}nu} & \(\varsigma\)
\texttt{\textbackslash{}varsigma} & \(\omega\)
\texttt{\textbackslash{}omega} \\
\(\eta\) \texttt{\textbackslash{}eta} & \(\xi\)
\texttt{\textbackslash{}xi} & & \\
& & & \\
\(\Gamma\) \texttt{\textbackslash{}Gamma} & \(\Lambda\)
\texttt{\textbackslash{}Lambda} & \(\Sigma\)
\texttt{\textbackslash{}Sigma} & \(\Psi\)
\texttt{\textbackslash{}Psi} \\
\(\Delta\) \texttt{\textbackslash{}Delta} & \(\Xi\)
\texttt{\textbackslash{}Xi} & \(\Upsilon\)
\texttt{\textbackslash{}Upsilon} & \(\Omega\)
\texttt{\textbackslash{}Omega} \\
\(\Theta\) \texttt{\textbackslash{}Theta} & \(\Pi\)
\texttt{\textbackslash{}Pi} & \(\Phi\) \texttt{\textbackslash{}Phi} & \\
\bottomrule()
\end{longtable}

\begin{Shaded}
\begin{Highlighting}[]
\BuiltInTok{\textbackslash{}documentclass}\NormalTok{\{}\ExtensionTok{article}\NormalTok{\}}
\BuiltInTok{\textbackslash{}usepackage}\NormalTok{[T1]\{}\ExtensionTok{fontenc}\NormalTok{\}}
\BuiltInTok{\textbackslash{}usepackage}\NormalTok{[turkish]\{}\ExtensionTok{babel}\NormalTok{\}}
\BuiltInTok{\textbackslash{}usepackage}\NormalTok{\{}\ExtensionTok{amsmath,amssymb,amsfonts}\NormalTok{\}}
\KeywordTok{\textbackslash{}begin}\NormalTok{\{}\ExtensionTok{document}\NormalTok{\}}

\SpecialStringTok{$}\SpecialCharTok{\textbackslash{}forall}\SpecialStringTok{ }\SpecialCharTok{\textbackslash{}epsilon}\SpecialStringTok{\textgreater{}0$}\NormalTok{ için}

\KeywordTok{\textbackslash{}end}\NormalTok{\{}\ExtensionTok{document}\NormalTok{\}}
\end{Highlighting}
\end{Shaded}

\hypertarget{fonksiyonlar}{%
\section{Fonksiyonlar}\label{fonksiyonlar}}

{\LaTeX}'de fonksiyonlar aşağıdaki komutlarla dizilirler.

\texttt{\textbackslash{}arccos\ \ \textbackslash{}cos\ \ \textbackslash{}csc\ \ \textbackslash{}exp\ \ \textbackslash{}ker\ \ \textbackslash{}limsup\ \ \textbackslash{}min\ \ \textbackslash{}sinh\ \textbackslash{}arcsin\ \ \textbackslash{}cosh\ \ \textbackslash{}deg\ \ \textbackslash{}gcd\ \ \textbackslash{}lg\ \ \textbackslash{}ln\ \ \textbackslash{}Pr\ \ \textbackslash{}sup\ \textbackslash{}arctan\ \ \textbackslash{}cot\ \ \textbackslash{}det\ \ \textbackslash{}hom\ \ \textbackslash{}lim\ \ \textbackslash{}log\ \ \textbackslash{}sec\ \ \textbackslash{}tan\ \textbackslash{}arg\ \ \textbackslash{}coth\ \ \textbackslash{}dim\ \ \textbackslash{}inf\ \ \textbackslash{}liminf\ \ \textbackslash{}max\ \ \textbackslash{}sin\ \ \textbackslash{}tanh}

Matematik kipinde fonksiyonlar diğer değişkenler gibi italik değil düz
yazılırlar ve boşluklar otomatik ayarlanır.

\begin{Shaded}
\begin{Highlighting}[]
\BuiltInTok{\textbackslash{}documentclass}\NormalTok{\{}\ExtensionTok{article}\NormalTok{\}}
\BuiltInTok{\textbackslash{}usepackage}\NormalTok{[T1]\{}\ExtensionTok{fontenc}\NormalTok{\}}
\BuiltInTok{\textbackslash{}usepackage}\NormalTok{[turkish]\{}\ExtensionTok{babel}\NormalTok{\}}
\BuiltInTok{\textbackslash{}usepackage}\NormalTok{\{}\ExtensionTok{amsmath,amssymb,amsfonts}\NormalTok{\}}
\KeywordTok{\textbackslash{}begin}\NormalTok{\{}\ExtensionTok{document}\NormalTok{\}}

\SpecialStringTok{$}\SpecialCharTok{\textbackslash{}sin}\SpecialStringTok{ x$}\NormalTok{, }\SpecialStringTok{$}\SpecialCharTok{\textbackslash{}exp}\SpecialStringTok{ x$}\NormalTok{, }\SpecialStringTok{$}\SpecialCharTok{\textbackslash{}log}\SpecialStringTok{ x$}\NormalTok{,}
\SpecialStringTok{$}\SpecialCharTok{\textbackslash{}det}\SpecialStringTok{ A$}\NormalTok{, }\SpecialStringTok{$}\SpecialCharTok{\textbackslash{}min}\SpecialStringTok{\_\{x}\SpecialCharTok{\textbackslash{}in}\SpecialStringTok{ A\} f(x)$}

\KeywordTok{\textbackslash{}end}\NormalTok{\{}\ExtensionTok{document}\NormalTok{\}}
\end{Highlighting}
\end{Shaded}

Bunların dışında bir fonksiyon tanımlamak için
\texttt{\textbackslash{}DeclareMathOperator} komutu kullanılır.
\texttt{\textbackslash{}DeclareMathOperator\{\textbackslash{}obeb\}\{obeb\ \}}
komutundan sonra artık kullanabileceğiniz bir ``obeb'' fonksiyonu olur.

\begin{Shaded}
\begin{Highlighting}[]
\BuiltInTok{\textbackslash{}documentclass}\NormalTok{\{}\ExtensionTok{article}\NormalTok{\}}
\BuiltInTok{\textbackslash{}usepackage}\NormalTok{[T1]\{}\ExtensionTok{fontenc}\NormalTok{\}}
\BuiltInTok{\textbackslash{}usepackage}\NormalTok{[turkish]\{}\ExtensionTok{babel}\NormalTok{\}}
\BuiltInTok{\textbackslash{}usepackage}\NormalTok{\{}\ExtensionTok{amsmath,amssymb,amsfonts}\NormalTok{\}}
\FunctionTok{\textbackslash{}DeclareMathOperator}\NormalTok{\{}\FunctionTok{\textbackslash{}obeb}\NormalTok{\}\{obeb\}}
\KeywordTok{\textbackslash{}begin}\NormalTok{\{}\ExtensionTok{document}\NormalTok{\}}

\SpecialStringTok{$}\SpecialCharTok{\textbackslash{}obeb}\SpecialStringTok{(12,16)=4$}

\KeywordTok{\textbackslash{}end}\NormalTok{\{}\ExtensionTok{document}\NormalTok{\}}
\end{Highlighting}
\end{Shaded}

Bu komutun sınır değerleri sağ taraf yerine alta dizen yıldızlı sürümü
vardır: \texttt{\textbackslash{}DeclareMathOperator*}. Örneğin
\texttt{\textbackslash{}DeclareMathOperator*\{\textbackslash{}Max\}\{Max\}}
komutunu sahanlıkta verdikten sonra belgede kullanırsanız şöyle bir
çıktı alırsınız:

\begin{Shaded}
\begin{Highlighting}[]
\BuiltInTok{\textbackslash{}documentclass}\NormalTok{\{}\ExtensionTok{article}\NormalTok{\}}
\BuiltInTok{\textbackslash{}usepackage}\NormalTok{[T1]\{}\ExtensionTok{fontenc}\NormalTok{\}}
\BuiltInTok{\textbackslash{}usepackage}\NormalTok{[turkish]\{}\ExtensionTok{babel}\NormalTok{\}}
\BuiltInTok{\textbackslash{}usepackage}\NormalTok{\{}\ExtensionTok{amsmath,amssymb,amsfonts}\NormalTok{\}}
\FunctionTok{\textbackslash{}DeclareMathOperator*}\NormalTok{\{}\FunctionTok{\textbackslash{}Max}\NormalTok{\}\{Max\}}
\KeywordTok{\textbackslash{}begin}\NormalTok{\{}\ExtensionTok{document}\NormalTok{\}}

\KeywordTok{\textbackslash{}begin}\NormalTok{\{}\ExtensionTok{equation*}\NormalTok{\}}
\SpecialStringTok{ }\SpecialCharTok{\textbackslash{}Max}\SpecialStringTok{\_\{x}\SpecialCharTok{\textbackslash{}in}\SpecialStringTok{ A\} f(x)}
\KeywordTok{\textbackslash{}end}\NormalTok{\{}\ExtensionTok{equation*}\NormalTok{\}}

\KeywordTok{\textbackslash{}end}\NormalTok{\{}\ExtensionTok{document}\NormalTok{\}}
\end{Highlighting}
\end{Shaded}

Modülo fonksiyonu içinse \texttt{\textbackslash{}mod} ya da
\texttt{\textbackslash{}pmod} komutları verilir. İkinci komut fonksiyonu
parantez içinde yazar.

\begin{Shaded}
\begin{Highlighting}[]
\BuiltInTok{\textbackslash{}documentclass}\NormalTok{\{}\ExtensionTok{article}\NormalTok{\}}
\BuiltInTok{\textbackslash{}usepackage}\NormalTok{[T1]\{}\ExtensionTok{fontenc}\NormalTok{\}}
\BuiltInTok{\textbackslash{}usepackage}\NormalTok{[turkish]\{}\ExtensionTok{babel}\NormalTok{\}}
\BuiltInTok{\textbackslash{}usepackage}\NormalTok{\{}\ExtensionTok{amsmath,amssymb,amsfonts}\NormalTok{\}}
\KeywordTok{\textbackslash{}begin}\NormalTok{\{}\ExtensionTok{document}\NormalTok{\}}

\SpecialStringTok{$a}\SpecialCharTok{\textbackslash{}equiv}\SpecialStringTok{ b}\SpecialCharTok{\textbackslash{}pmod}\SpecialStringTok{ p$}\NormalTok{ ise }\SpecialStringTok{$p}\SpecialCharTok{\textbackslash{}mid}\SpecialStringTok{ a{-}b$}\NormalTok{\textquotesingle{}dir.}

\KeywordTok{\textbackslash{}end}\NormalTok{\{}\ExtensionTok{document}\NormalTok{\}}
\end{Highlighting}
\end{Shaded}

Limit için \texttt{\textbackslash{}lim} komutu aşağıdaki şekilde
verilir.

\begin{Shaded}
\begin{Highlighting}[]
\FunctionTok{\textbackslash{}lim}\NormalTok{\_\{\textless{}değişken\textgreater{} }\FunctionTok{\textbackslash{}to}\NormalTok{ \textless{}değişken\textgreater{}\}}
\end{Highlighting}
\end{Shaded}

Buradaki \texttt{\textbackslash{}to} komutu \(\to\) üretir ve \(\infty\)
için \texttt{\textbackslash{}infty} komutu verilir.

\begin{Shaded}
\begin{Highlighting}[]
\BuiltInTok{\textbackslash{}documentclass}\NormalTok{\{}\ExtensionTok{article}\NormalTok{\}}
\BuiltInTok{\textbackslash{}usepackage}\NormalTok{[T1]\{}\ExtensionTok{fontenc}\NormalTok{\}}
\BuiltInTok{\textbackslash{}usepackage}\NormalTok{[turkish]\{}\ExtensionTok{babel}\NormalTok{\}}
\BuiltInTok{\textbackslash{}usepackage}\NormalTok{\{}\ExtensionTok{amsmath,amssymb,amsfonts}\NormalTok{\}}
\KeywordTok{\textbackslash{}begin}\NormalTok{\{}\ExtensionTok{document}\NormalTok{\}}

\SpecialStringTok{\textbackslash{}[}
\SpecialCharTok{\textbackslash{}lim}\SpecialStringTok{\_\{x}\SpecialCharTok{\textbackslash{}to}\SpecialStringTok{ 0\}}
\SpecialCharTok{\textbackslash{}frac}\SpecialStringTok{\{}\SpecialCharTok{\textbackslash{}sin}\SpecialStringTok{ x\}\{x\}=1 }\SpecialCharTok{\textbackslash{}qquad}
\SpecialCharTok{\textbackslash{}lim}\SpecialStringTok{\_\{n}\SpecialCharTok{\textbackslash{}to}\SpecialStringTok{ +}\SpecialCharTok{\textbackslash{}infty}\SpecialStringTok{\}f\_n=}\SpecialCharTok{\textbackslash{}delta}
\SpecialStringTok{\textbackslash{}]}

\KeywordTok{\textbackslash{}end}\NormalTok{\{}\ExtensionTok{document}\NormalTok{\}}
\end{Highlighting}
\end{Shaded}

\hypertarget{yux131ux11fux131n-simgeleri}{%
\section{Yığın Simgeleri}\label{yux131ux11fux131n-simgeleri}}

Matematikte bazen bir ifadenin altına ya da üstüne başka ifadeler yazmak
gerekebilir. Bunlar yığın simgeleri olarak adlandırılırlar.

{\LaTeX}'de aşağıdaki

\begin{Shaded}
\begin{Highlighting}[]
\FunctionTok{\textbackslash{}overset}\NormalTok{\{\textless{}birinci değişken\textgreater{}\}\{\textless{}ikinci değişken\textgreater{}\}}
\end{Highlighting}
\end{Shaded}

komutu birinci değişkendeki sembolü, normal boyda yazılan ikincinin
üzerine yazar. \texttt{\textbackslash{}underset} komutu ise alta yazar.

\begin{Shaded}
\begin{Highlighting}[]
\BuiltInTok{\textbackslash{}documentclass}\NormalTok{\{}\ExtensionTok{article}\NormalTok{\}}
\BuiltInTok{\textbackslash{}usepackage}\NormalTok{[T1]\{}\ExtensionTok{fontenc}\NormalTok{\}}
\BuiltInTok{\textbackslash{}usepackage}\NormalTok{[turkish]\{}\ExtensionTok{babel}\NormalTok{\}}
\BuiltInTok{\textbackslash{}usepackage}\NormalTok{\{}\ExtensionTok{amsmath,amssymb,amsfonts}\NormalTok{\}}
\KeywordTok{\textbackslash{}begin}\NormalTok{\{}\ExtensionTok{document}\NormalTok{\}}

\SpecialStringTok{\textbackslash{}[}
\SpecialCharTok{\textbackslash{}overset}\SpecialStringTok{\{R\}\{}\SpecialCharTok{\textbackslash{}sim}\SpecialStringTok{\}}
\SpecialStringTok{\textbackslash{}]}
\SpecialStringTok{\textbackslash{}[}
\SpecialCharTok{\textbackslash{}underset}\SpecialStringTok{\{R\}\{}\SpecialCharTok{\textbackslash{}sim}\SpecialStringTok{\}}
\SpecialStringTok{\textbackslash{}]}
\KeywordTok{\textbackslash{}end}\NormalTok{\{}\ExtensionTok{document}\NormalTok{\}}
\end{Highlighting}
\end{Shaded}

\texttt{\textbackslash{}overline} ve \texttt{\textbackslash{}underline}
komutları bir ifadenin üstüne veya altına yatay bir çizgi çekerler.

\begin{Shaded}
\begin{Highlighting}[]
\BuiltInTok{\textbackslash{}documentclass}\NormalTok{\{}\ExtensionTok{article}\NormalTok{\}}
\BuiltInTok{\textbackslash{}usepackage}\NormalTok{[T1]\{}\ExtensionTok{fontenc}\NormalTok{\}}
\BuiltInTok{\textbackslash{}usepackage}\NormalTok{[turkish]\{}\ExtensionTok{babel}\NormalTok{\}}
\BuiltInTok{\textbackslash{}usepackage}\NormalTok{\{}\ExtensionTok{amsmath,amssymb,amsfonts}\NormalTok{\}}
\KeywordTok{\textbackslash{}begin}\NormalTok{\{}\ExtensionTok{document}\NormalTok{\}}

\SpecialStringTok{\textbackslash{}[}
\SpecialCharTok{\textbackslash{}overline}\SpecialStringTok{\{x+y\}}
\SpecialStringTok{\textbackslash{}]}
\SpecialStringTok{\textbackslash{}[}
\SpecialCharTok{\textbackslash{}underline}\SpecialStringTok{\{x+y\}}
\SpecialStringTok{\textbackslash{}]}

\KeywordTok{\textbackslash{}end}\NormalTok{\{}\ExtensionTok{document}\NormalTok{\}}
\end{Highlighting}
\end{Shaded}

\texttt{\textbackslash{}overbrace} ve
\texttt{\textbackslash{}underbrace} komutları bir ifadenin üstüne veya
altına yatay bir çengel atarlar.

\begin{Shaded}
\begin{Highlighting}[]
\BuiltInTok{\textbackslash{}documentclass}\NormalTok{\{}\ExtensionTok{article}\NormalTok{\}}
\BuiltInTok{\textbackslash{}usepackage}\NormalTok{[T1]\{}\ExtensionTok{fontenc}\NormalTok{\}}
\BuiltInTok{\textbackslash{}usepackage}\NormalTok{[turkish]\{}\ExtensionTok{babel}\NormalTok{\}}
\BuiltInTok{\textbackslash{}usepackage}\NormalTok{\{}\ExtensionTok{amsmath,amssymb,amsfonts}\NormalTok{\}}
\KeywordTok{\textbackslash{}begin}\NormalTok{\{}\ExtensionTok{document}\NormalTok{\}}

\SpecialStringTok{\textbackslash{}[}
\SpecialCharTok{\textbackslash{}underbrace}\SpecialStringTok{\{1+2+}\SpecialCharTok{\textbackslash{}dots}\SpecialStringTok{+n\}\_\{\{\}=}
\SpecialCharTok{\textbackslash{}frac}\SpecialStringTok{\{n(n+1)\}\{2\}\}}
\SpecialStringTok{+(n+1)}
\SpecialStringTok{\textbackslash{}]}

\KeywordTok{\textbackslash{}end}\NormalTok{\{}\ExtensionTok{document}\NormalTok{\}}
\end{Highlighting}
\end{Shaded}

\texttt{\textbackslash{}overleftarrow} komutu ifadenin üstüne sola,
\texttt{\textbackslash{}overrightarrow} ise sağa bir ok çizer. Bu
komutlar vektörleri göstermek için kullanılabilir. Vektörler için
\texttt{\textbackslash{}vec} komutu da kullanılır.
\texttt{\textbackslash{}stackrel} komutu
\texttt{\textbackslash{}overset} gibi davranır.

\begin{Shaded}
\begin{Highlighting}[]
\BuiltInTok{\textbackslash{}documentclass}\NormalTok{\{}\ExtensionTok{article}\NormalTok{\}}
\BuiltInTok{\textbackslash{}usepackage}\NormalTok{[T1]\{}\ExtensionTok{fontenc}\NormalTok{\}}
\BuiltInTok{\textbackslash{}usepackage}\NormalTok{[turkish]\{}\ExtensionTok{babel}\NormalTok{\}}
\BuiltInTok{\textbackslash{}usepackage}\NormalTok{\{}\ExtensionTok{amsmath,amssymb,amsfonts}\NormalTok{\}}
\KeywordTok{\textbackslash{}begin}\NormalTok{\{}\ExtensionTok{document}\NormalTok{\}}

\SpecialStringTok{\textbackslash{}[}
\SpecialCharTok{\textbackslash{}overrightarrow}\SpecialStringTok{\{AB\} }\SpecialCharTok{\textbackslash{}quad}\SpecialStringTok{ }\SpecialCharTok{\textbackslash{}vec}\SpecialStringTok{\{a\}}
\SpecialStringTok{\textbackslash{}]}
\SpecialStringTok{\textbackslash{}[}
\SpecialCharTok{\textbackslash{}int}\SpecialStringTok{ f\_N(x) }\SpecialCharTok{\textbackslash{}stackrel}\SpecialStringTok{\{!\}\{=\} 1}
\SpecialStringTok{\textbackslash{}]}

\KeywordTok{\textbackslash{}end}\NormalTok{\{}\ExtensionTok{document}\NormalTok{\}}
\end{Highlighting}
\end{Shaded}

\hypertarget{matrisler}{%
\section{Matrisler}\label{matrisler}}

Temel matrisler \texttt{matrix} ortamında girilir. Bu ortamda elemanlar
otomatik ortalanır ve sütunlar normal bir tablo gibi dizilir. Her sütun
\texttt{\&} karakteriyle ayrılır ve alt satıra geçmek için
\texttt{\textbackslash{}\textbackslash{}} komutu verilir.

\begin{Shaded}
\begin{Highlighting}[]
\BuiltInTok{\textbackslash{}documentclass}\NormalTok{\{}\ExtensionTok{article}\NormalTok{\}}
\BuiltInTok{\textbackslash{}usepackage}\NormalTok{[T1]\{}\ExtensionTok{fontenc}\NormalTok{\}}
\BuiltInTok{\textbackslash{}usepackage}\NormalTok{[turkish]\{}\ExtensionTok{babel}\NormalTok{\}}
\BuiltInTok{\textbackslash{}usepackage}\NormalTok{\{}\ExtensionTok{amsmath,amssymb,amsfonts}\NormalTok{\}}
\KeywordTok{\textbackslash{}begin}\NormalTok{\{}\ExtensionTok{document}\NormalTok{\}}

\SpecialStringTok{\textbackslash{}[}
\KeywordTok{\textbackslash{}begin}\NormalTok{\{}\ExtensionTok{matrix}\NormalTok{\}}
\SpecialStringTok{a \& b \& c }\SpecialCharTok{\textbackslash{}\textbackslash{}}
\SpecialStringTok{d \& e \& f }\SpecialCharTok{\textbackslash{}\textbackslash{}}
\SpecialStringTok{g \& h \& i}
\KeywordTok{\textbackslash{}end}\NormalTok{\{}\ExtensionTok{matrix}\NormalTok{\}}
\SpecialStringTok{\textbackslash{}]}

\KeywordTok{\textbackslash{}end}\NormalTok{\{}\ExtensionTok{document}\NormalTok{\}}
\end{Highlighting}
\end{Shaded}

Çeşitli matrisler dizmek için \texttt{matrix} ortamının farklı sürümleri
kullanılır: \texttt{pmatrix}, \texttt{bmatrix}, \texttt{Bmatrix},
\texttt{vmatrix} ve \texttt{Vmatrix}. Bu ortamlar sırasıyla yuvarlak,
köşeli, çengelli, dikey çubuklu ve çift dikey çubuklu matrisler
oluşturur.

\begin{Shaded}
\begin{Highlighting}[]
\BuiltInTok{\textbackslash{}documentclass}\NormalTok{\{}\ExtensionTok{article}\NormalTok{\}}
\BuiltInTok{\textbackslash{}usepackage}\NormalTok{[T1]\{}\ExtensionTok{fontenc}\NormalTok{\}}
\BuiltInTok{\textbackslash{}usepackage}\NormalTok{[turkish]\{}\ExtensionTok{babel}\NormalTok{\}}
\BuiltInTok{\textbackslash{}usepackage}\NormalTok{\{}\ExtensionTok{amsmath,amssymb,amsfonts}\NormalTok{\}}
\KeywordTok{\textbackslash{}begin}\NormalTok{\{}\ExtensionTok{document}\NormalTok{\}}

\SpecialStringTok{\textbackslash{}[}
\KeywordTok{\textbackslash{}begin}\NormalTok{\{}\ExtensionTok{pmatrix}\NormalTok{\}}
\SpecialStringTok{1 \& 2 }\SpecialCharTok{\textbackslash{}\textbackslash{}}
\SpecialStringTok{3 \& 4}
\KeywordTok{\textbackslash{}end}\NormalTok{\{}\ExtensionTok{pmatrix}\NormalTok{\}}
\SpecialStringTok{\textbackslash{}]}
\SpecialStringTok{\textbackslash{}[}
\KeywordTok{\textbackslash{}begin}\NormalTok{\{}\ExtensionTok{bmatrix}\NormalTok{\}}
\SpecialStringTok{1 \& 2 }\SpecialCharTok{\textbackslash{}\textbackslash{}}
\SpecialStringTok{3 \& 4}
\KeywordTok{\textbackslash{}end}\NormalTok{\{}\ExtensionTok{bmatrix}\NormalTok{\}}
\SpecialStringTok{\textbackslash{}]}
\SpecialStringTok{\textbackslash{}[}
\SpecialStringTok{A=}
\KeywordTok{\textbackslash{}begin}\NormalTok{\{}\ExtensionTok{bmatrix}\NormalTok{\}}
\SpecialStringTok{x\_\{11\} \& x\_\{12\} \& }\SpecialCharTok{\textbackslash{}dots}\SpecialStringTok{ }\SpecialCharTok{\textbackslash{}\textbackslash{}}
\SpecialStringTok{x\_\{21\} \& x\_\{22\} \& }\SpecialCharTok{\textbackslash{}dots}\SpecialStringTok{ }\SpecialCharTok{\textbackslash{}\textbackslash{}}
\SpecialCharTok{\textbackslash{}vdots}\SpecialStringTok{ \& }\SpecialCharTok{\textbackslash{}vdots}\SpecialStringTok{ \& }\SpecialCharTok{\textbackslash{}ddots}
\KeywordTok{\textbackslash{}end}\NormalTok{\{}\ExtensionTok{bmatrix}\NormalTok{\}}
\SpecialStringTok{\textbackslash{}]}

\KeywordTok{\textbackslash{}end}\NormalTok{\{}\ExtensionTok{document}\NormalTok{\}}
\end{Highlighting}
\end{Shaded}

Küçük bir matris yazmak için \texttt{smallmatrix} ortamı kullanılır. Bu
matriste parantezler elle eklenmelidir. Ayrıca
\href{https://ftp.cc.uoc.gr/mirrors/CTAN/macros/latex/contrib/mathtools/mathtools.pdf}{mathtools}
paketi \texttt{psmallmatrix}, \texttt{bsmallmatrix} vb. ortamlar sağlar.

Bazı durumlarda, hizalamayı elle yapmak ve sütunlar veya satırlar
arasına çizgi çekmek istenebilir. Bu durumda \texttt{tabular} ortamının
matematik sürümü olan \texttt{array} ortamını kullanılmalıdır.

\begin{Shaded}
\begin{Highlighting}[]
\BuiltInTok{\textbackslash{}documentclass}\NormalTok{\{}\ExtensionTok{article}\NormalTok{\}}
\BuiltInTok{\textbackslash{}usepackage}\NormalTok{[T1]\{}\ExtensionTok{fontenc}\NormalTok{\}}
\BuiltInTok{\textbackslash{}usepackage}\NormalTok{[turkish]\{}\ExtensionTok{babel}\NormalTok{\}}
\BuiltInTok{\textbackslash{}usepackage}\NormalTok{\{}\ExtensionTok{amsmath,amssymb,amsfonts}\NormalTok{\}}
\KeywordTok{\textbackslash{}begin}\NormalTok{\{}\ExtensionTok{document}\NormalTok{\}}

\SpecialStringTok{\textbackslash{}[}
\SpecialCharTok{\textbackslash{}left}\SpecialStringTok{(}\KeywordTok{\textbackslash{}begin}\NormalTok{\{}\ExtensionTok{array}\NormalTok{\}}\SpecialStringTok{\{r|r\}}
\SpecialStringTok{ {-}1\&2}\SpecialCharTok{\textbackslash{}\textbackslash{}\textbackslash{}hline}
\SpecialStringTok{  3\&{-}4}
\KeywordTok{\textbackslash{}end}\NormalTok{\{}\ExtensionTok{array}\NormalTok{\}}\SpecialCharTok{\textbackslash{}right}\SpecialStringTok{)}
\SpecialStringTok{\textbackslash{}]}

\KeywordTok{\textbackslash{}end}\NormalTok{\{}\ExtensionTok{document}\NormalTok{\}}
\end{Highlighting}
\end{Shaded}

\hypertarget{yazux131-biuxe7em-ve-boyutlarux131}{%
\section{Yazı Biçem ve
Boyutları}\label{yazux131-biuxe7em-ve-boyutlarux131}}

Matematiksel ifadeleri dizerken bazen yazının biçemini ya da boyutunu
değiştirmek isteyebilirsiniz.

{\LaTeX}'de matematik kipindeki yazıların biçemleri aşağıdaki komutlar
kullanılarak değiştirilir.

\hypertarget{tbl-matyazi}{}
\begin{longtable}[]{@{}ll@{}}
\caption{\label{tbl-matyazi}Matematik Kipinde Yazı
Biçemleri}\tabularnewline
\toprule()
\textbf{Komut} & \textbf{Görünüm} \\
\midrule()
\endfirsthead
\toprule()
\textbf{Komut} & \textbf{Görünüm} \\
\midrule()
\endhead
\texttt{\textbackslash{}mathnormal\{ABC\ def\ 123\}} &
\(ABC def 123\) \\
\texttt{\textbackslash{}mathrm\{ABC\ def\ 123\}} &
\(\mathrm{ABC def 123}\) \\
\texttt{\textbackslash{}mathit\{ABC\ def\ 123\}} &
\(\mathit{ABC def 123}\) \\
\texttt{\textbackslash{}mathbf\{ABC\ def\ 123\}} &
\(\mathbf{ABC def 123}\) \\
\texttt{\textbackslash{}mathtt\{ABC\ def\ 123\}} &
\(\mathtt{ABC def 123}\) \\
\texttt{\textbackslash{}mathsf\{ABC\ def\ 123\}} &
\(\mathsf{ABC def 123}\) \\
\texttt{\textbackslash{}mathfrak\{ABC\ def\ 123\}} &
\(\mathfrak{ABC def 123}\) \\
\texttt{\textbackslash{}mathbb\{ABC\}} & \(\mathbb{ABC}\) \\
\texttt{\textbackslash{}mathcal\{ABC\}} & \(\mathcal{ABC}\) \\
\texttt{\textbackslash{}mathscr\{ABC\}} & \(\mathscr{ABC}\) \\
\bottomrule()
\end{longtable}

Son satırdaki komutun kullanılabilmesi için

\begin{Shaded}
\begin{Highlighting}[]
\BuiltInTok{\textbackslash{}usepackage}\NormalTok{\{}\ExtensionTok{mathrsfs}\NormalTok{\}}
\end{Highlighting}
\end{Shaded}

komutuyla \textbf{mathrsfs} paketi eklenmiş olmalıdır.

Bu komutlarla girilen ifadelerdeki boşluklar yine dikkate alınmaz ve
yine aksanlı harfler girilemez.

Matematik kipindeki bir ifadenin hem kalın hem de italik yazılması için
\texttt{\textbackslash{}boldsymbol} komutu kullanılmalıdır.

\begin{Shaded}
\begin{Highlighting}[]
\BuiltInTok{\textbackslash{}documentclass}\NormalTok{\{}\ExtensionTok{article}\NormalTok{\}}
\BuiltInTok{\textbackslash{}usepackage}\NormalTok{[T1]\{}\ExtensionTok{fontenc}\NormalTok{\}}
\BuiltInTok{\textbackslash{}usepackage}\NormalTok{[turkish]\{}\ExtensionTok{babel}\NormalTok{\}}
\BuiltInTok{\textbackslash{}usepackage}\NormalTok{\{}\ExtensionTok{amsmath,amssymb,amsfonts}\NormalTok{\}}
\BuiltInTok{\textbackslash{}usepackage}\NormalTok{\{}\ExtensionTok{mathrsfs}\NormalTok{\}}
\KeywordTok{\textbackslash{}begin}\NormalTok{\{}\ExtensionTok{document}\NormalTok{\}}

\SpecialStringTok{\textbackslash{}[}
\SpecialCharTok{\textbackslash{}mu}\SpecialStringTok{, M }\SpecialCharTok{\textbackslash{}qquad}\SpecialStringTok{ }\SpecialCharTok{\textbackslash{}mathbf}\SpecialStringTok{\{}\SpecialCharTok{\textbackslash{}mu}\SpecialStringTok{\}, }\SpecialCharTok{\textbackslash{}mathbf}\SpecialStringTok{\{M\}}\SpecialCharTok{\textbackslash{}qquad}
\SpecialCharTok{\textbackslash{}boldsymbol}\SpecialStringTok{\{}\SpecialCharTok{\textbackslash{}mu}\SpecialStringTok{\}, }\SpecialCharTok{\textbackslash{}boldsymbol}\SpecialStringTok{\{M\}}
\SpecialStringTok{\textbackslash{}]}

\KeywordTok{\textbackslash{}end}\NormalTok{\{}\ExtensionTok{document}\NormalTok{\}}
\end{Highlighting}
\end{Shaded}

Matematik kipindeki yazının boyutunu elle ayarlayabileceğiniz dört komut
vardır: \texttt{\textbackslash{}displaystyle},
\texttt{\textbackslash{}textstyle}, \texttt{\textbackslash{}scriptstyle}
ve \texttt{\textbackslash{}scriptscriptstyle}.
\texttt{\textbackslash{}textstyle} komutu ifadeyi normal metin boyutunda
dizer, \texttt{\textbackslash{}displaystyle} komutu ise ifadeyi ayrı
satırda sergilenir gibi büyük dizer.
\texttt{\textbackslash{}scriptstyle} ve
\texttt{\textbackslash{}scriptscriptstyle} komutları da normal metin
boyutundan küçük dizerler.

\begin{Shaded}
\begin{Highlighting}[]
\BuiltInTok{\textbackslash{}documentclass}\NormalTok{\{}\ExtensionTok{article}\NormalTok{\}}
\BuiltInTok{\textbackslash{}usepackage}\NormalTok{[T1]\{}\ExtensionTok{fontenc}\NormalTok{\}}
\BuiltInTok{\textbackslash{}usepackage}\NormalTok{[turkish]\{}\ExtensionTok{babel}\NormalTok{\}}
\BuiltInTok{\textbackslash{}usepackage}\NormalTok{\{}\ExtensionTok{amsmath,amssymb,amsfonts}\NormalTok{\}}
\BuiltInTok{\textbackslash{}usepackage}\NormalTok{\{}\ExtensionTok{mathrsfs}\NormalTok{\}}
\KeywordTok{\textbackslash{}begin}\NormalTok{\{}\ExtensionTok{document}\NormalTok{\}}

\SpecialStringTok{\textbackslash{}[}
\SpecialCharTok{\textbackslash{}sum}\SpecialStringTok{\_\{k=0\}\^{}n z\^{}k }\SpecialCharTok{\textbackslash{}qquad}\SpecialStringTok{ }\SpecialCharTok{\textbackslash{}textstyle\textbackslash{}sum}\SpecialStringTok{\_\{k=0\}\^{}n z\^{}k}
\SpecialStringTok{\textbackslash{}]}
\SpecialStringTok{$}\SpecialCharTok{\textbackslash{}displaystyle\textbackslash{}sum}\SpecialStringTok{\_\{k=0\}\^{}n z\^{}k$}\FunctionTok{\textbackslash{}qquad} 
\SpecialStringTok{$}\SpecialCharTok{\textbackslash{}sum}\SpecialStringTok{\_\{k=0\}\^{}n z\^{}k$} \FunctionTok{\textbackslash{}qquad}
\SpecialStringTok{$}\SpecialCharTok{\textbackslash{}scriptstyle\textbackslash{}sum}\SpecialStringTok{\_\{k=0\}\^{}n z\^{}k$}

\KeywordTok{\textbackslash{}end}\NormalTok{\{}\ExtensionTok{document}\NormalTok{\}}
\end{Highlighting}
\end{Shaded}

Kesirler dizilirken
\texttt{\{\textbackslash{}displaystyle\textbackslash{}frac\{…\}\{…\}\}}
ve \texttt{\{\textbackslash{}textstyle\textbackslash{}frac\{…\}\{…\}\}}
komutları yerine onların kısaltmaları olan
\texttt{\textbackslash{}dfrac} ve \texttt{\textbackslash{}tfrac}
komutları kullanılabilir. Aynı şey \texttt{\textbackslash{}binom} komutu
için de geçerlidir.

\begin{Shaded}
\begin{Highlighting}[]
\BuiltInTok{\textbackslash{}documentclass}\NormalTok{\{}\ExtensionTok{article}\NormalTok{\}}
\BuiltInTok{\textbackslash{}usepackage}\NormalTok{[T1]\{}\ExtensionTok{fontenc}\NormalTok{\}}
\BuiltInTok{\textbackslash{}usepackage}\NormalTok{[turkish]\{}\ExtensionTok{babel}\NormalTok{\}}
\BuiltInTok{\textbackslash{}usepackage}\NormalTok{\{}\ExtensionTok{amsmath,amssymb,amsfonts}\NormalTok{\}}
\BuiltInTok{\textbackslash{}usepackage}\NormalTok{\{}\ExtensionTok{mathrsfs}\NormalTok{\}}
\KeywordTok{\textbackslash{}begin}\NormalTok{\{}\ExtensionTok{document}\NormalTok{\}}

\SpecialStringTok{$}\SpecialCharTok{\textbackslash{}frac}\SpecialStringTok{\{1\}\{n\}}\SpecialCharTok{\textbackslash{}log}\SpecialStringTok{ x$} \FunctionTok{\textbackslash{}quad}
\SpecialStringTok{$}\SpecialCharTok{\textbackslash{}dfrac}\SpecialStringTok{\{1\}\{n\}}\SpecialCharTok{\textbackslash{}log}\SpecialStringTok{ x$} \FunctionTok{\textbackslash{}quad}
\SpecialStringTok{$}\SpecialCharTok{\textbackslash{}binom}\SpecialStringTok{\{n\}\{3\}$} \FunctionTok{\textbackslash{}quad}
\SpecialStringTok{$}\SpecialCharTok{\textbackslash{}dbinom}\SpecialStringTok{\{n\}\{3\}$}
\SpecialStringTok{\textbackslash{}[}
\SpecialCharTok{\textbackslash{}frac}\SpecialStringTok{\{1\}\{n\}}\SpecialCharTok{\textbackslash{}log}\SpecialStringTok{ x }\SpecialCharTok{\textbackslash{}quad}
\SpecialCharTok{\textbackslash{}tfrac}\SpecialStringTok{\{1\}\{n\}}\SpecialCharTok{\textbackslash{}log}\SpecialStringTok{ x }\SpecialCharTok{\textbackslash{}quad}
\SpecialCharTok{\textbackslash{}binom}\SpecialStringTok{\{n\}\{3\} }\SpecialCharTok{\textbackslash{}quad}
\SpecialCharTok{\textbackslash{}tbinom}\SpecialStringTok{\{n\}\{3\}}
\SpecialStringTok{\textbackslash{}]}

\KeywordTok{\textbackslash{}end}\NormalTok{\{}\ExtensionTok{document}\NormalTok{\}}
\end{Highlighting}
\end{Shaded}

\hypertarget{denklem-ortamlarux131}{%
\section{Denklem Ortamları}\label{denklem-ortamlarux131}}

Bir satıra sığmayacak kadar uzun bir formülü ya da birden çok satırdan
oluşan bir denklemi veya denklem sistemlerini hizalayıp dizmek için
{\LaTeX}'de çeşitli ortamlar kullanılır.

\texttt{multline} ortamı bir formülü hizalanmamış bir kaç satıra ayırır.

\begin{Shaded}
\begin{Highlighting}[]
\BuiltInTok{\textbackslash{}documentclass}\NormalTok{\{}\ExtensionTok{article}\NormalTok{\}}
\BuiltInTok{\textbackslash{}usepackage}\NormalTok{[T1]\{}\ExtensionTok{fontenc}\NormalTok{\}}
\BuiltInTok{\textbackslash{}usepackage}\NormalTok{[turkish]\{}\ExtensionTok{babel}\NormalTok{\}}
\BuiltInTok{\textbackslash{}usepackage}\NormalTok{\{}\ExtensionTok{amsmath,amssymb,amsfonts}\NormalTok{\}}
\KeywordTok{\textbackslash{}begin}\NormalTok{\{}\ExtensionTok{document}\NormalTok{\}}

\KeywordTok{\textbackslash{}begin}\NormalTok{\{}\ExtensionTok{multline}\NormalTok{\}}
\SpecialStringTok{f=a+b+c }\SpecialCharTok{\textbackslash{}\textbackslash{}}
\SpecialStringTok{+i+j+k+l }\SpecialCharTok{\textbackslash{}\textbackslash{}}
\SpecialStringTok{+x+y+z}
\KeywordTok{\textbackslash{}end}\NormalTok{\{}\ExtensionTok{multline}\NormalTok{\}}

\KeywordTok{\textbackslash{}end}\NormalTok{\{}\ExtensionTok{document}\NormalTok{\}}
\end{Highlighting}
\end{Shaded}

Bu ortamda ilk satır sola, son satır sağa ve kalanlar ortalı hizalanır.
Denklemin numarası da son satırın sağına yazılır. Dekleme numara
verilmesi istenmiyorsa ortam \texttt{multline*} şeklinde
kullanılmalıdır.

\texttt{split} ortamı denklemi dikey hizalanmış birden çok satırda
dizer.

\begin{Shaded}
\begin{Highlighting}[]
\BuiltInTok{\textbackslash{}documentclass}\NormalTok{\{}\ExtensionTok{article}\NormalTok{\}}
\BuiltInTok{\textbackslash{}usepackage}\NormalTok{[T1]\{}\ExtensionTok{fontenc}\NormalTok{\}}
\BuiltInTok{\textbackslash{}usepackage}\NormalTok{[turkish]\{}\ExtensionTok{babel}\NormalTok{\}}
\BuiltInTok{\textbackslash{}usepackage}\NormalTok{\{}\ExtensionTok{amsmath,amssymb,amsfonts}\NormalTok{\}}
\KeywordTok{\textbackslash{}begin}\NormalTok{\{}\ExtensionTok{document}\NormalTok{\}}

\KeywordTok{\textbackslash{}begin}\NormalTok{\{}\ExtensionTok{equation}\NormalTok{\}}
\KeywordTok{\textbackslash{}begin}\NormalTok{\{}\ExtensionTok{split}\NormalTok{\}}
\SpecialStringTok{ a\&= b+c{-}d}\SpecialCharTok{\textbackslash{}\textbackslash{}}
\SpecialStringTok{  \&= e{-}f}\SpecialCharTok{\textbackslash{}\textbackslash{}}
\SpecialStringTok{  \&= g}
\KeywordTok{\textbackslash{}end}\NormalTok{\{}\ExtensionTok{split}\NormalTok{\}}
\KeywordTok{\textbackslash{}end}\NormalTok{\{}\ExtensionTok{equation}\NormalTok{\}}

\KeywordTok{\textbackslash{}end}\NormalTok{\{}\ExtensionTok{document}\NormalTok{\}}
\end{Highlighting}
\end{Shaded}

Hizalama \texttt{\&} karakteriyle yapılır (genelde \texttt{=}
işaretinden hemen önce kullanılır). Ortam mutlaka formülün
numaralandırılmasından sorumlu ya da numara vermeyen başka bir matematik
ortamında kullanılması gerekir.

\texttt{gather} ortamı birden fazla formülü birlikte gruplandırır,
ortalar ve her birini ayrı bir satırda numaralandırır. Yine
\texttt{gather*} ortamı, aynı türden numaralandırılmamış formüller
üretir.

\begin{Shaded}
\begin{Highlighting}[]
\BuiltInTok{\textbackslash{}documentclass}\NormalTok{\{}\ExtensionTok{article}\NormalTok{\}}
\BuiltInTok{\textbackslash{}usepackage}\NormalTok{[T1]\{}\ExtensionTok{fontenc}\NormalTok{\}}
\BuiltInTok{\textbackslash{}usepackage}\NormalTok{[turkish]\{}\ExtensionTok{babel}\NormalTok{\}}
\BuiltInTok{\textbackslash{}usepackage}\NormalTok{\{}\ExtensionTok{amsmath,amssymb,amsfonts}\NormalTok{\}}
\KeywordTok{\textbackslash{}begin}\NormalTok{\{}\ExtensionTok{document}\NormalTok{\}}

\KeywordTok{\textbackslash{}begin}\NormalTok{\{}\ExtensionTok{gather}\NormalTok{\}}
\SpecialStringTok{a=b+c }\SpecialCharTok{\textbackslash{}\textbackslash{}}
\SpecialStringTok{V+F{-}S=2}
\KeywordTok{\textbackslash{}end}\NormalTok{\{}\ExtensionTok{gather}\NormalTok{\}}

\KeywordTok{\textbackslash{}end}\NormalTok{\{}\ExtensionTok{document}\NormalTok{\}}
\end{Highlighting}
\end{Shaded}

\texttt{align} ortamı, iki veya daha fazla satırdan oluşan bir denklemi
her bir satırı hizalı ve numaralı şekilde dizmek için kullanılır.
Hizalama aynı şekilde \texttt{\&} karakteriyle yapılır. Ortam yıldızlı
(\texttt{align*}) şekilde kullanılırsa hiçbir satır numaralandırılmaz.
\texttt{align} ortamı aynı zamanda birden fazla özerk formül dizisini
birleştirmek için de kullanışlıdır. Bu durumda, \texttt{\&} karakteri
konumuna bağlı olarak hizalama ve ayırıcı olmak üzere iki farklı işlev
üstlenir.

\begin{Shaded}
\begin{Highlighting}[]
\BuiltInTok{\textbackslash{}documentclass}\NormalTok{\{}\ExtensionTok{article}\NormalTok{\}}
\BuiltInTok{\textbackslash{}usepackage}\NormalTok{[T1]\{}\ExtensionTok{fontenc}\NormalTok{\}}
\BuiltInTok{\textbackslash{}usepackage}\NormalTok{[turkish]\{}\ExtensionTok{babel}\NormalTok{\}}
\BuiltInTok{\textbackslash{}usepackage}\NormalTok{\{}\ExtensionTok{amsmath,amssymb,amsfonts}\NormalTok{\}}
\KeywordTok{\textbackslash{}begin}\NormalTok{\{}\ExtensionTok{document}\NormalTok{\}}

\KeywordTok{\textbackslash{}begin}\NormalTok{\{}\ExtensionTok{align}\NormalTok{\}}
\SpecialStringTok{   a\& = b+c+d }\SpecialCharTok{\textbackslash{}\textbackslash{}}
\SpecialStringTok{   e\& = f }\SpecialCharTok{\textbackslash{}\textbackslash{}}
\SpecialStringTok{ x{-}1\& = y+z }
\KeywordTok{\textbackslash{}end}\NormalTok{\{}\ExtensionTok{align}\NormalTok{\}}
\KeywordTok{\textbackslash{}begin}\NormalTok{\{}\ExtensionTok{align*}\NormalTok{\}}
\SpecialStringTok{ a \&=b \& c\&=d \& e\&=f }\SpecialCharTok{\textbackslash{}\textbackslash{}}
\SpecialStringTok{ u \&=v \& w\&=x \& y\&=z}
\KeywordTok{\textbackslash{}end}\NormalTok{\{}\ExtensionTok{align*}\NormalTok{\}}
\KeywordTok{\textbackslash{}end}\NormalTok{\{}\ExtensionTok{document}\NormalTok{\}}
\end{Highlighting}
\end{Shaded}

\texttt{alignat} ortamı \texttt{align} ortamına benzer fakat sütun
sayısını belirten bir değişken alır (Bir satırda kullanılan \texttt{\&}
sayısının bir fazlasının yarısı sütun sayısını vermelidir\} ve
denklemler arasındaki yatay boşluğun kontrolünü sağlar. Eğer boşluk
komutlarından biri kullanılmazsa denklem sistemleri arasında boşluk
bırakılmaz (örnekte boşluk komutu olarak \texttt{\textbackslash{}qquad}
kullanılmıştır).

\begin{Shaded}
\begin{Highlighting}[]
\BuiltInTok{\textbackslash{}documentclass}\NormalTok{\{}\ExtensionTok{article}\NormalTok{\}}
\BuiltInTok{\textbackslash{}usepackage}\NormalTok{[T1]\{}\ExtensionTok{fontenc}\NormalTok{\}}
\BuiltInTok{\textbackslash{}usepackage}\NormalTok{[turkish]\{}\ExtensionTok{babel}\NormalTok{\}}
\BuiltInTok{\textbackslash{}usepackage}\NormalTok{\{}\ExtensionTok{amsmath,amssymb,amsfonts}\NormalTok{\}}
\KeywordTok{\textbackslash{}begin}\NormalTok{\{}\ExtensionTok{document}\NormalTok{\}}

\KeywordTok{\textbackslash{}begin}\NormalTok{\{}\ExtensionTok{alignat*}\NormalTok{\}\{3\}}
\SpecialStringTok{  a\&=b}\SpecialCharTok{\textbackslash{}qquad}\SpecialStringTok{ \& c\&=d}\SpecialCharTok{\textbackslash{}qquad}\SpecialStringTok{ \& e\&=f }\SpecialCharTok{\textbackslash{}\textbackslash{}}
\SpecialStringTok{  u\&=v}\SpecialCharTok{\textbackslash{}qquad}\SpecialStringTok{ \& w\&=x}\SpecialCharTok{\textbackslash{}qquad}\SpecialStringTok{ \& y\&=z }
\KeywordTok{\textbackslash{}end}\NormalTok{\{}\ExtensionTok{alignat*}\NormalTok{\}}

\KeywordTok{\textbackslash{}end}\NormalTok{\{}\ExtensionTok{document}\NormalTok{\}}
\end{Highlighting}
\end{Shaded}

\texttt{flalign} ortamı \texttt{align} ortamına benzer ancak ilk denklem
sistemini sola ve son denklem sistemini sağa yaslar.

\texttt{aligned} ortamı yine \texttt{align} ortamına benzer fakat başka
bir matematik ortamında kullanılması gerekir.

\begin{Shaded}
\begin{Highlighting}[]
\BuiltInTok{\textbackslash{}documentclass}\NormalTok{\{}\ExtensionTok{article}\NormalTok{\}}
\BuiltInTok{\textbackslash{}usepackage}\NormalTok{[T1]\{}\ExtensionTok{fontenc}\NormalTok{\}}
\BuiltInTok{\textbackslash{}usepackage}\NormalTok{[turkish]\{}\ExtensionTok{babel}\NormalTok{\}}
\BuiltInTok{\textbackslash{}usepackage}\NormalTok{\{}\ExtensionTok{amsmath,amssymb,amsfonts}\NormalTok{\}}
\KeywordTok{\textbackslash{}begin}\NormalTok{\{}\ExtensionTok{document}\NormalTok{\}}

\SpecialStringTok{\textbackslash{}[}
\SpecialCharTok{\textbackslash{}left}\SpecialStringTok{.}
\KeywordTok{\textbackslash{}begin}\NormalTok{\{}\ExtensionTok{aligned}\NormalTok{\}}
\SpecialStringTok{ a\&= b+1 }\SpecialCharTok{\textbackslash{}\textbackslash{}}
\SpecialStringTok{ a\&= 2b}
\KeywordTok{\textbackslash{}end}\NormalTok{\{}\ExtensionTok{aligned}\NormalTok{\}}
\SpecialCharTok{\textbackslash{}right\textbackslash{}\}}
\SpecialCharTok{\textbackslash{}quad}
\SpecialCharTok{\textbackslash{}text}\NormalTok{\{}\SpecialStringTok{$a=2$}\NormalTok{ ve }\SpecialStringTok{$b=1$}\NormalTok{.\}}
\SpecialStringTok{\textbackslash{}]}

\KeywordTok{\textbackslash{}end}\NormalTok{\{}\ExtensionTok{document}\NormalTok{\}}
\end{Highlighting}
\end{Shaded}

\texttt{cases} ortamı parçalı fonksiyonları dizmek için kullanışlıdır.
Ortamın içine yazılan denklemlerin solunda uygun boyda bir çengelli
parantez açar. Sütunlar sola yaslıdır. Ortamın başka bir matematik
ortamında kullanılması gerekir.

\begin{Shaded}
\begin{Highlighting}[]
\BuiltInTok{\textbackslash{}documentclass}\NormalTok{\{}\ExtensionTok{article}\NormalTok{\}}
\BuiltInTok{\textbackslash{}usepackage}\NormalTok{[T1]\{}\ExtensionTok{fontenc}\NormalTok{\}}
\BuiltInTok{\textbackslash{}usepackage}\NormalTok{[turkish]\{}\ExtensionTok{babel}\NormalTok{\}}
\BuiltInTok{\textbackslash{}usepackage}\NormalTok{\{}\ExtensionTok{amsmath,amssymb,amsfonts}\NormalTok{\}}
\KeywordTok{\textbackslash{}begin}\NormalTok{\{}\ExtensionTok{document}\NormalTok{\}}

\SpecialStringTok{\textbackslash{}[}
\SpecialStringTok{n!=}
\KeywordTok{\textbackslash{}begin}\NormalTok{\{}\ExtensionTok{cases}\NormalTok{\}}
\SpecialStringTok{1 \& }\SpecialCharTok{\textbackslash{}text}\NormalTok{\{}\SpecialStringTok{$n=0$}\NormalTok{ ise\}}\SpecialStringTok{ }\SpecialCharTok{\textbackslash{}\textbackslash{}}
\SpecialStringTok{n(n{-}1)!\& }\SpecialCharTok{\textbackslash{}text}\NormalTok{\{}\SpecialStringTok{$n\textbackslash{}ge 1$}\NormalTok{ ise\}}
\KeywordTok{\textbackslash{}end}\NormalTok{\{}\ExtensionTok{cases}\NormalTok{\}}
\SpecialStringTok{\textbackslash{}]}

\KeywordTok{\textbackslash{}end}\NormalTok{\{}\ExtensionTok{document}\NormalTok{\}}
\end{Highlighting}
\end{Shaded}

Her satıra numara veren bir ortamda bazı satırların numarasız olması
istenirse bu satırların sonuna \texttt{\textbackslash{}notag} ya da
\texttt{\textbackslash{}nonumber} komutları verilir.
\texttt{\textbackslash{}tag} komutuyla ise keyfi bir numara ya da işaret
yazılabilir.

\begin{Shaded}
\begin{Highlighting}[]
\BuiltInTok{\textbackslash{}documentclass}\NormalTok{\{}\ExtensionTok{article}\NormalTok{\}}
\BuiltInTok{\textbackslash{}usepackage}\NormalTok{[T1]\{}\ExtensionTok{fontenc}\NormalTok{\}}
\BuiltInTok{\textbackslash{}usepackage}\NormalTok{[turkish]\{}\ExtensionTok{babel}\NormalTok{\}}
\BuiltInTok{\textbackslash{}usepackage}\NormalTok{\{}\ExtensionTok{amsmath,amssymb,amsfonts}\NormalTok{\}}
\KeywordTok{\textbackslash{}begin}\NormalTok{\{}\ExtensionTok{document}\NormalTok{\}}

\KeywordTok{\textbackslash{}begin}\NormalTok{\{}\ExtensionTok{align}\NormalTok{\}}
\SpecialStringTok{ x\&=y}\SpecialCharTok{\textbackslash{}\textbackslash{}}
\SpecialStringTok{ z\&=y+1 }\SpecialCharTok{\textbackslash{}notag\textbackslash{}\textbackslash{}}
\SpecialStringTok{ w\&=3 }\SpecialCharTok{\textbackslash{}tag}\SpecialStringTok{\{*\}}
\KeywordTok{\textbackslash{}end}\NormalTok{\{}\ExtensionTok{align}\NormalTok{\}}

\KeywordTok{\textbackslash{}end}\NormalTok{\{}\ExtensionTok{document}\NormalTok{\}}
\end{Highlighting}
\end{Shaded}

Numaralı formüllere etiket yine \texttt{\textbackslash{}label} komutuyla
koyulur ve \texttt{\textbackslash{}eqref} komutuyla atıf yapılır.

\begin{Shaded}
\begin{Highlighting}[]
\BuiltInTok{\textbackslash{}documentclass}\NormalTok{\{}\ExtensionTok{article}\NormalTok{\}}
\BuiltInTok{\textbackslash{}usepackage}\NormalTok{[T1]\{}\ExtensionTok{fontenc}\NormalTok{\}}
\BuiltInTok{\textbackslash{}usepackage}\NormalTok{[turkish]\{}\ExtensionTok{babel}\NormalTok{\}}
\BuiltInTok{\textbackslash{}usepackage}\NormalTok{\{}\ExtensionTok{amsmath,amssymb,amsfonts}\NormalTok{\}}
\KeywordTok{\textbackslash{}begin}\NormalTok{\{}\ExtensionTok{document}\NormalTok{\}}

\KeywordTok{\textbackslash{}begin}\NormalTok{\{}\ExtensionTok{align}\NormalTok{\}}
\SpecialStringTok{ a \&= b+c }\SpecialCharTok{\textbackslash{}\textbackslash{}}
\SpecialStringTok{ c \&= d  }\SpecialCharTok{\textbackslash{}label}\SpecialStringTok{\{eq:cd\}}\SpecialCharTok{\textbackslash{}\textbackslash{}}
\SpecialStringTok{ e \&= f+g}
\KeywordTok{\textbackslash{}end}\NormalTok{\{}\ExtensionTok{align}\NormalTok{\}}
\NormalTok{Yukarıdaki }\KeywordTok{\textbackslash{}eqref}\NormalTok{\{}\ExtensionTok{eq:cd}\NormalTok{\}}
\NormalTok{formülü}\FunctionTok{\textbackslash{}dots}

\KeywordTok{\textbackslash{}end}\NormalTok{\{}\ExtensionTok{document}\NormalTok{\}}
\end{Highlighting}
\end{Shaded}

Aynı ortamda yer alan formüllerin 1, 2,\ldots{} yerine 1.a, 1.b,\ldots{}
biçiminde numaralandırılması için ortamın \texttt{subequations}
ortamının içine yazılması gerekir.

\begin{Shaded}
\begin{Highlighting}[]
\BuiltInTok{\textbackslash{}documentclass}\NormalTok{\{}\ExtensionTok{article}\NormalTok{\}}
\BuiltInTok{\textbackslash{}usepackage}\NormalTok{[T1]\{}\ExtensionTok{fontenc}\NormalTok{\}}
\BuiltInTok{\textbackslash{}usepackage}\NormalTok{[turkish]\{}\ExtensionTok{babel}\NormalTok{\}}
\BuiltInTok{\textbackslash{}usepackage}\NormalTok{\{}\ExtensionTok{amsmath,amssymb,amsfonts}\NormalTok{\}}
\KeywordTok{\textbackslash{}begin}\NormalTok{\{}\ExtensionTok{document}\NormalTok{\}}

\KeywordTok{\textbackslash{}begin}\NormalTok{\{}\ExtensionTok{subequations}\NormalTok{\}}
\KeywordTok{\textbackslash{}label}\NormalTok{\{}\ExtensionTok{eq:sub}\NormalTok{\}}
\KeywordTok{\textbackslash{}begin}\NormalTok{\{}\ExtensionTok{align}\NormalTok{\}}
\SpecialStringTok{ a \&= b+c }\SpecialCharTok{\textbackslash{}\textbackslash{}}
\SpecialStringTok{ c \&= d }\SpecialCharTok{\textbackslash{}label}\SpecialStringTok{\{eq:subb\} }\SpecialCharTok{\textbackslash{}\textbackslash{}}
\SpecialStringTok{ e \&= f+g}
\KeywordTok{\textbackslash{}end}\NormalTok{\{}\ExtensionTok{align}\NormalTok{\}}
\KeywordTok{\textbackslash{}end}\NormalTok{\{}\ExtensionTok{subequations}\NormalTok{\}}

\NormalTok{Formül\textasciitilde{}}\KeywordTok{\textbackslash{}eqref}\NormalTok{\{}\ExtensionTok{eq:sub}\NormalTok{\} ve}
\NormalTok{altformül\textasciitilde{}}\KeywordTok{\textbackslash{}eqref}\NormalTok{\{}\ExtensionTok{eq:subb}\NormalTok{\}}\FunctionTok{\textbackslash{}dots}

\KeywordTok{\textbackslash{}end}\NormalTok{\{}\ExtensionTok{document}\NormalTok{\}}
\end{Highlighting}
\end{Shaded}

\hypertarget{teorem-ve-benzeri-ortamlar}{%
\section{Teorem ve Benzeri Ortamlar}\label{teorem-ve-benzeri-ortamlar}}

Gerçek bir matematik kitabı karıştırdıysanız ``Teorem 2.1'', ``Sonuç
2.1.1'' gibi ifadelerle başlayan paragraflara rastlamış olmalısınız.
Bunlar metnin geri kalanından ayrılmış ve yanında sıralı sayılarla
etiketli paragraflardır. Bu, matematikte teoremler için yaygın olarak
kullanılır, ancak her şey için kullanılabilir.

{\LaTeX}, herhangi bir teorem benzeri bildirimi kolayca tanımlamanıza
izin veren bir komut sunar: \texttt{\textbackslash{}newtheorem}.

\hypertarget{temel-teoremler}{%
\subsection{Temel Teoremler}\label{temel-teoremler}}

Öncelikle sahanlığa

\begin{Shaded}
\begin{Highlighting}[]
\BuiltInTok{\textbackslash{}usepackage}\NormalTok{\{}\ExtensionTok{amsthm}\NormalTok{\}}
\end{Highlighting}
\end{Shaded}

komutuyla \texttt{amsthm} paketini ekleyiniz. En basit kullanım

\begin{Shaded}
\begin{Highlighting}[]
\FunctionTok{\textbackslash{}newtheorem}\NormalTok{\{\textless{}ad\textgreater{}\}\{\textless{}başlık\textgreater{}\}}
\end{Highlighting}
\end{Shaded}

komutunun sahanlığa verilmesidir. İlk değişken olan
\texttt{\textless{}ad\textgreater{}}, referans olarak kullanacağınız
addır, ikinci değişken \texttt{\textless{}başlık\textgreater{}} ise
{\LaTeX}'in her kullandığınızda yazdıracağı çıktıdır.
\texttt{\textless{}ad\textgreater{}} değişkeni aksanlı bir harf
içermemelidir. Örneğin

\begin{Shaded}
\begin{Highlighting}[]
\FunctionTok{\textbackslash{}newtheorem}\NormalTok{\{tanim\}\{Tanım\}}
\end{Highlighting}
\end{Shaded}

komutunu sahanlığa verdiğinizde \texttt{tanim} ortamını {\LaTeX}'e
tanıtmış olursunuz. Ayrıca kullanılan teoreme (bu örnekte Tanım) özel
bir ad vermek ya da not düşmek isteyebilirsiniz. Bu, ortam komutundan
sonra köşeli parantezler içinde belirtilebilir:

\begin{Shaded}
\begin{Highlighting}[]
\BuiltInTok{\textbackslash{}documentclass}\NormalTok{\{}\ExtensionTok{article}\NormalTok{\}}
\BuiltInTok{\textbackslash{}usepackage}\NormalTok{[T1]\{}\ExtensionTok{fontenc}\NormalTok{\}}
\BuiltInTok{\textbackslash{}usepackage}\NormalTok{[turkish]\{}\ExtensionTok{babel}\NormalTok{\}}
\BuiltInTok{\textbackslash{}usepackage}\NormalTok{\{}\ExtensionTok{amsmath,amssymb,amsfonts}\NormalTok{\}}
\BuiltInTok{\textbackslash{}usepackage}\NormalTok{\{}\ExtensionTok{amsthm}\NormalTok{\}}
\FunctionTok{\textbackslash{}newtheorem}\NormalTok{\{tanim\}\{Tanım\}}
\KeywordTok{\textbackslash{}begin}\NormalTok{\{}\ExtensionTok{document}\NormalTok{\}}

\KeywordTok{\textbackslash{}begin}\NormalTok{\{}\ExtensionTok{tanim}\NormalTok{\}}
\NormalTok{ İşte yeni bir tanım.}
\KeywordTok{\textbackslash{}end}\NormalTok{\{}\ExtensionTok{tanim}\NormalTok{\}}

\KeywordTok{\textbackslash{}begin}\NormalTok{\{}\ExtensionTok{tanim}\NormalTok{\}[Gauss]}
\NormalTok{ Gauss\textquotesingle{}un tanımı.}
\KeywordTok{\textbackslash{}end}\NormalTok{\{}\ExtensionTok{tanim}\NormalTok{\}}

\KeywordTok{\textbackslash{}end}\NormalTok{\{}\ExtensionTok{document}\NormalTok{\}}
\end{Highlighting}
\end{Shaded}

\hypertarget{sayauxe7lar}{%
\subsection{Sayaçlar}\label{sayauxe7lar}}

Sayaçlar, belge sınıfına göre varsayılan değerleri kullanır. Örneğin
\texttt{book} sınıfında bir teorem kullanıldığında ``Teorem 2.3''
(kitabın 2'inci bölümünde yer alan 3'üncü teorem), \texttt{article}
sınıfında bir teorem kullanıldığında ``Teorem 3'' (makaledeki 3'üncü
teorem) benzeri çıktılar alınır. Varsayılan ayarları değiştirmek için
sayacın takip etmesi istenilen bölüm seviyesi (chapter, section gibi)
belirtilebilir:

\begin{Shaded}
\begin{Highlighting}[]
\FunctionTok{\textbackslash{}newtheorem}\NormalTok{\{\textless{}ad\textgreater{}\}\{\textless{}başlık\textgreater{}\}[\textless{}sayaç\textgreater{}]}
\end{Highlighting}
\end{Shaded}

Örneğin \texttt{article} sınıfında sahanlığa

\begin{Shaded}
\begin{Highlighting}[]
\FunctionTok{\textbackslash{}newtheorem}\NormalTok{\{teorem\}\{Teorem\}[section]}
\end{Highlighting}
\end{Shaded}

komutunu verdiğinizde teoreminiz \texttt{\textbackslash{}section} başlık
seviyesinin numarasına göre numara alır.

\begin{Shaded}
\begin{Highlighting}[]
\BuiltInTok{\textbackslash{}documentclass}\NormalTok{\{}\ExtensionTok{article}\NormalTok{\}}
\BuiltInTok{\textbackslash{}usepackage}\NormalTok{[T1]\{}\ExtensionTok{fontenc}\NormalTok{\}}
\BuiltInTok{\textbackslash{}usepackage}\NormalTok{[turkish]\{}\ExtensionTok{babel}\NormalTok{\}}
\BuiltInTok{\textbackslash{}usepackage}\NormalTok{\{}\ExtensionTok{amsmath,amssymb,amsfonts}\NormalTok{\}}
\BuiltInTok{\textbackslash{}usepackage}\NormalTok{\{}\ExtensionTok{amsthm}\NormalTok{\}}
\FunctionTok{\textbackslash{}newtheorem}\NormalTok{\{teorem\}\{Teorem\}[section]}
\KeywordTok{\textbackslash{}begin}\NormalTok{\{}\ExtensionTok{document}\NormalTok{\}}

\KeywordTok{\textbackslash{}section}\NormalTok{\{Teoremler\}}

\KeywordTok{\textbackslash{}begin}\NormalTok{\{}\ExtensionTok{teorem}\NormalTok{\}}
\NormalTok{ İşte bölüm numarasını takip eden teorem.}
\KeywordTok{\textbackslash{}end}\NormalTok{\{}\ExtensionTok{teorem}\NormalTok{\}}



\KeywordTok{\textbackslash{}end}\NormalTok{\{}\ExtensionTok{document}\NormalTok{\}}
\end{Highlighting}
\end{Shaded}

Varsayılan olarak, her teorem kendi sayacını kullanır. Bununla birlikte,
benzer teoremlerin (örneğin Teoremler, Lemmalar ve Sonuçlar) bir sayacı
paylaşması yaygındır. Bu durumda, sonraki teoremleri şöyle tanımlayın:

\begin{Shaded}
\begin{Highlighting}[]
\FunctionTok{\textbackslash{}newtheorem}\NormalTok{\{\textless{}ad\textgreater{}\}[\textless{}sayaç\textgreater{}]\{\textless{}başlık\textgreater{}\}}
\end{Highlighting}
\end{Shaded}

Burada \texttt{\textless{}sayaç\textgreater{}} kullanılacak olan sayacın
adıdır. Genelde ana teorem adı olur. Örneğin sahanlıkta

\begin{Shaded}
\begin{Highlighting}[]
\FunctionTok{\textbackslash{}newtheorem}\NormalTok{\{lemma\}[teorem]\{Lemma\}}
\end{Highlighting}
\end{Shaded}

tanımlamasını yaparsanız (Bu komutu verebilmek için önceden
\texttt{teorem} ortamı tanımlanmış olmalıdır) artık Lemma'lar
Teorem'lerle aynı sayacı kullanacaktır.

\begin{Shaded}
\begin{Highlighting}[]
\BuiltInTok{\textbackslash{}documentclass}\NormalTok{\{}\ExtensionTok{article}\NormalTok{\}}
\BuiltInTok{\textbackslash{}usepackage}\NormalTok{[T1]\{}\ExtensionTok{fontenc}\NormalTok{\}}
\BuiltInTok{\textbackslash{}usepackage}\NormalTok{[turkish]\{}\ExtensionTok{babel}\NormalTok{\}}
\BuiltInTok{\textbackslash{}usepackage}\NormalTok{\{}\ExtensionTok{amsmath,amssymb,amsfonts}\NormalTok{\}}
\BuiltInTok{\textbackslash{}usepackage}\NormalTok{\{}\ExtensionTok{amsthm}\NormalTok{\}}
\FunctionTok{\textbackslash{}newtheorem}\NormalTok{\{teorem\}\{Teorem\}[section]}
\FunctionTok{\textbackslash{}newtheorem}\NormalTok{\{lemma\}[teorem]\{Lemma\}}
\KeywordTok{\textbackslash{}begin}\NormalTok{\{}\ExtensionTok{document}\NormalTok{\}}

\KeywordTok{\textbackslash{}section}\NormalTok{\{Teoremler\}}

\KeywordTok{\textbackslash{}begin}\NormalTok{\{}\ExtensionTok{teorem}\NormalTok{\}}
\NormalTok{ İşte bölüm numarasını takip eden teorem.}
\KeywordTok{\textbackslash{}end}\NormalTok{\{}\ExtensionTok{teorem}\NormalTok{\}}

\KeywordTok{\textbackslash{}begin}\NormalTok{\{}\ExtensionTok{lemma}\NormalTok{\}}
\NormalTok{ İşte teoremle aynı sayacı paylaşan lemma.}
\KeywordTok{\textbackslash{}end}\NormalTok{\{}\ExtensionTok{lemma}\NormalTok{\}}

\KeywordTok{\textbackslash{}end}\NormalTok{\{}\ExtensionTok{document}\NormalTok{\}}
\end{Highlighting}
\end{Shaded}

\texttt{\textbackslash{}newtheorem} komutu en fazla bir tane isteğe
bağlı değişken içerebilir. Ayrıca komut
\texttt{\textbackslash{}newtheorem*} şekilde kullanılırsa numara
verilmeyen bir teorem tanımlamış olur.

\hypertarget{kanux131tlar}{%
\subsection{Kanıtlar}\label{kanux131tlar}}

Bir teoremin kanıtı \texttt{proof} ortamında yazılır. Genel kullanım
şöyledir:

\begin{Shaded}
\begin{Highlighting}[]
\BuiltInTok{\textbackslash{}documentclass}\NormalTok{\{}\ExtensionTok{article}\NormalTok{\}}
\BuiltInTok{\textbackslash{}usepackage}\NormalTok{[T1]\{}\ExtensionTok{fontenc}\NormalTok{\}}
\BuiltInTok{\textbackslash{}usepackage}\NormalTok{[turkish]\{}\ExtensionTok{babel}\NormalTok{\}}
\BuiltInTok{\textbackslash{}usepackage}\NormalTok{\{}\ExtensionTok{amsmath,amssymb,amsfonts}\NormalTok{\}}
\BuiltInTok{\textbackslash{}usepackage}\NormalTok{\{}\ExtensionTok{amsthm}\NormalTok{\}}
\FunctionTok{\textbackslash{}newtheorem}\NormalTok{\{teorem\}\{Teorem\}[section]}
\KeywordTok{\textbackslash{}begin}\NormalTok{\{}\ExtensionTok{document}\NormalTok{\}}

\KeywordTok{\textbackslash{}section}\NormalTok{\{Teoremler\}}

\KeywordTok{\textbackslash{}begin}\NormalTok{\{}\ExtensionTok{teorem}\NormalTok{\}}
\NormalTok{ İşte teoremim.}
\KeywordTok{\textbackslash{}end}\NormalTok{\{}\ExtensionTok{teorem}\NormalTok{\}}

\KeywordTok{\textbackslash{}begin}\NormalTok{\{}\ExtensionTok{proof}\NormalTok{\}}
\NormalTok{ İşte kanıtım.}
\KeywordTok{\textbackslash{}end}\NormalTok{\{}\ExtensionTok{proof}\NormalTok{\}}

\KeywordTok{\textbackslash{}end}\NormalTok{\{}\ExtensionTok{document}\NormalTok{\}}
\end{Highlighting}
\end{Shaded}

Bu ortamı kullandığınızda en sona kanıtın bittiği anlamında bir kare
(QED adıyla da bilinir) ekler ve Türkçe dil paketi ekli belgelerde
``Kanıt'' adını yazar. Bu ad, \texttt{\textbackslash{}proofname}
komutunda saklı olup istenirse değiştirilebilir:

\begin{Shaded}
\begin{Highlighting}[]
\FunctionTok{\textbackslash{}renewcommand}\NormalTok{\{}\ExtensionTok{\textbackslash{}proofname}\NormalTok{\}\{İspat\}}
\end{Highlighting}
\end{Shaded}

Bu komuttan sonra kullanılan tüm \texttt{proof} ortamlarında artık
``Kanıt'' yerine ``İspat'' yazar.

Ayrıca kanıtı tek seferliğine elle adlandırmak isterseniz, kendi adınızı
köşeli parantezler içine yazabilirsiniz:

Kanıtın sonunu bildiren \(\square\) işareti bazen son satırda yalnız
kalırsa, \texttt{\textbackslash{}qedhere} komutuyla onu doğru yere
oturtabilirsiniz:

Özel bir QED sembolü kullanmak isterseniz
\texttt{\textbackslash{}qedsymbol} komutunu yeniden tanımlayabilirsiniz.

Eğer sembolü gizlemek isterseniz \texttt{\textbackslash{}renewcommand}
komutunun son değişkenini boş bırakmanız yeterli olur.

\begin{Shaded}
\begin{Highlighting}[]
\BuiltInTok{\textbackslash{}documentclass}\NormalTok{\{}\ExtensionTok{article}\NormalTok{\}}
\BuiltInTok{\textbackslash{}usepackage}\NormalTok{[T1]\{}\ExtensionTok{fontenc}\NormalTok{\}}
\BuiltInTok{\textbackslash{}usepackage}\NormalTok{[turkish]\{}\ExtensionTok{babel}\NormalTok{\}}
\BuiltInTok{\textbackslash{}usepackage}\NormalTok{\{}\ExtensionTok{amsmath,amssymb,amsfonts}\NormalTok{\}}
\BuiltInTok{\textbackslash{}usepackage}\NormalTok{\{}\ExtensionTok{amsthm}\NormalTok{\}}
\FunctionTok{\textbackslash{}newtheorem}\NormalTok{\{teorem\}\{Teorem\}[section]}
\KeywordTok{\textbackslash{}begin}\NormalTok{\{}\ExtensionTok{document}\NormalTok{\}}

\KeywordTok{\textbackslash{}section}\NormalTok{\{Teoremler\}}

\KeywordTok{\textbackslash{}begin}\NormalTok{\{}\ExtensionTok{teorem}\NormalTok{\}}
\NormalTok{ İşte teoremim.}
\KeywordTok{\textbackslash{}end}\NormalTok{\{}\ExtensionTok{teorem}\NormalTok{\}}

\KeywordTok{\textbackslash{}begin}\NormalTok{\{}\ExtensionTok{proof}\NormalTok{\}}
\NormalTok{ İşte kanıtım.}
\KeywordTok{\textbackslash{}end}\NormalTok{\{}\ExtensionTok{proof}\NormalTok{\}}

\KeywordTok{\textbackslash{}begin}\NormalTok{\{}\ExtensionTok{proof}\NormalTok{\}[İspat]}
\NormalTok{ İşte diğer ispatım.}
\KeywordTok{\textbackslash{}end}\NormalTok{\{}\ExtensionTok{proof}\NormalTok{\}}

\KeywordTok{\textbackslash{}begin}\NormalTok{\{}\ExtensionTok{proof}\NormalTok{\}}
\NormalTok{ Sadeleştirme yapılırsa }\SpecialStringTok{\textbackslash{}[E=mc\^{}2 }\SpecialCharTok{\textbackslash{}qedhere}\SpecialStringTok{\textbackslash{}]}
\KeywordTok{\textbackslash{}end}\NormalTok{\{}\ExtensionTok{proof}\NormalTok{\}}

\FunctionTok{\textbackslash{}renewcommand}\NormalTok{\{}\ExtensionTok{\textbackslash{}qedsymbol}\NormalTok{\}\{}\SpecialStringTok{$}\SpecialCharTok{\textbackslash{}blacksquare}\SpecialStringTok{$}\NormalTok{\}}
\KeywordTok{\textbackslash{}begin}\NormalTok{\{}\ExtensionTok{proof}\NormalTok{\}}
\NormalTok{ Siyah kare.}
\KeywordTok{\textbackslash{}end}\NormalTok{\{}\ExtensionTok{proof}\NormalTok{\}}

\KeywordTok{\textbackslash{}end}\NormalTok{\{}\ExtensionTok{document}\NormalTok{\}}
\end{Highlighting}
\end{Shaded}

\hypertarget{teorem-stilleri}{%
\subsection{Teorem Stilleri}\label{teorem-stilleri}}

Teorem stilleri \texttt{\textbackslash{}theoremstyle} komutuyla
değiştirilir. Bu komut, \texttt{\textbackslash{}newtheorem} komutu
kullanarak tanımlanan ortamların çıktısını değiştirme olanağını verir.

\begin{Shaded}
\begin{Highlighting}[]
\FunctionTok{\textbackslash{}theoremstyle}\NormalTok{\{\textless{}stil adı\textgreater{}\}}
\end{Highlighting}
\end{Shaded}

Buradaki \texttt{\textless{}stil\ adı\textgreater{}} kullanmak
istediğiniz stildir. Bu komuttan sonra tanımlanmış tüm teoremler bu
stili kullanacaktır. {\LaTeX}'de önceden tanımlanmış stiller
aşağıdakilerdir:

\hypertarget{tbl-teoremstil}{}
\begin{longtable}[]{@{}
  >{\raggedright\arraybackslash}p{(\columnwidth - 4\tabcolsep) * \real{0.4000}}
  >{\raggedright\arraybackslash}p{(\columnwidth - 4\tabcolsep) * \real{0.3000}}
  >{\raggedright\arraybackslash}p{(\columnwidth - 4\tabcolsep) * \real{0.3000}}@{}}
\caption{\label{tbl-teoremstil}Teorem Stilleri}\tabularnewline
\toprule()
\begin{minipage}[b]{\linewidth}\raggedright
\textbf{Stil Adı}
\end{minipage} & \begin{minipage}[b]{\linewidth}\raggedright
\textbf{Açıklama}
\end{minipage} & \begin{minipage}[b]{\linewidth}\raggedright
\textbf{Görünüm}
\end{minipage} \\
\midrule()
\endfirsthead
\toprule()
\begin{minipage}[b]{\linewidth}\raggedright
\textbf{Stil Adı}
\end{minipage} & \begin{minipage}[b]{\linewidth}\raggedright
\textbf{Açıklama}
\end{minipage} & \begin{minipage}[b]{\linewidth}\raggedright
\textbf{Görünüm}
\end{minipage} \\
\midrule()
\endhead
\texttt{plain} & Teoremler, lemmalar, önermeler vb. için kullanılır
(varsayılan) & Başlık düz ve kalın, gövde metni vurgulu \\
\texttt{definition} & Tanımlar ve örnekler için kullanılır & Başlık düz
ve kalın, gövde metni düz \\
\texttt{remark} & Açıklamalar ve notlar için kullanılır & Başlık
vurgulu, gövde metni düz \\
\bottomrule()
\end{longtable}

Örneğin sahanlığa

\begin{Shaded}
\begin{Highlighting}[]
\FunctionTok{\textbackslash{}theoremstyle}\NormalTok{\{remark\}}
\FunctionTok{\textbackslash{}newtheorem}\NormalTok{\{notum\}\{Not\}}
\end{Highlighting}
\end{Shaded}

komutlarını verip \texttt{notum} ortamını kullandığınızda başlık
vurgulu, gövde metni düz olacaktır.

\begin{Shaded}
\begin{Highlighting}[]
\BuiltInTok{\textbackslash{}documentclass}\NormalTok{\{}\ExtensionTok{article}\NormalTok{\}}
\BuiltInTok{\textbackslash{}usepackage}\NormalTok{[T1]\{}\ExtensionTok{fontenc}\NormalTok{\}}
\BuiltInTok{\textbackslash{}usepackage}\NormalTok{[turkish]\{}\ExtensionTok{babel}\NormalTok{\}}
\BuiltInTok{\textbackslash{}usepackage}\NormalTok{\{}\ExtensionTok{amsmath,amssymb,amsfonts}\NormalTok{\}}
\BuiltInTok{\textbackslash{}usepackage}\NormalTok{\{}\ExtensionTok{amsthm}\NormalTok{\}}
\FunctionTok{\textbackslash{}theoremstyle}\NormalTok{\{remark\}}
\FunctionTok{\textbackslash{}newtheorem}\NormalTok{\{notum\}\{Not\}}
\KeywordTok{\textbackslash{}begin}\NormalTok{\{}\ExtensionTok{document}\NormalTok{\}}

\KeywordTok{\textbackslash{}section}\NormalTok{\{Stiller\}}

\KeywordTok{\textbackslash{}begin}\NormalTok{\{}\ExtensionTok{notum}\NormalTok{\}}
\NormalTok{ Buraya not aldım.}
\KeywordTok{\textbackslash{}end}\NormalTok{\{}\ExtensionTok{notum}\NormalTok{\}}



\KeywordTok{\textbackslash{}end}\NormalTok{\{}\ExtensionTok{document}\NormalTok{\}}
\end{Highlighting}
\end{Shaded}

\hypertarget{uxf6zel-stiller}{%
\subsubsection{Özel Stiller}\label{uxf6zel-stiller}}

Kendi stilinizi tanımlamak için \texttt{\textbackslash{}newtheoremstyle}
komutunu kullanabilirsiniz:

\begin{Shaded}
\begin{Highlighting}[]
\FunctionTok{\textbackslash{}newtheoremstyle}\NormalTok{\{\textless{}stil adı\textgreater{}\}}\CommentTok{\% kullanılacak stilin adı}
\NormalTok{\{\textless{}üst boşluk\textgreater{}\}}\CommentTok{\% teoremin üstünde bırakılacak boşluk. Örn:3pt.}
\NormalTok{\{\textless{}alt boşluk\textgreater{}\}}\CommentTok{\% teoremin altında bırakılacak boşluk. Örn:3pt.}
\NormalTok{\{\textless{}gövde yazı tipi\textgreater{}\}}\CommentTok{\% teorem gövdesinde kullanılacak yazı tipi.}
 \CommentTok{\%Örn:\textbackslash{}normalfont, \textbackslash{}itshape...}
\NormalTok{\{\textless{}girinti\textgreater{}\}}\CommentTok{\% Paragraf girintisi ölçüsü. Örn:0pt}
\NormalTok{\{\textless{}başlık yazı tipi\textgreater{}\}}\CommentTok{\% teorem başlık yazı tipi.}
 \CommentTok{\%Örn:\textbackslash{}sffamily,\textbackslash{}bfseries}
\NormalTok{\{\textless{}noktalama\textgreater{}\}}\CommentTok{\% başlıktan sonraki noktalama.}
 \CommentTok{\%Noktalama istenmezse boşluk bırakılabilir. Örn:\textbackslash{}; }
\NormalTok{\{\textless{}boşluk\textgreater{}\}}\CommentTok{\% başlıktan sonraki boşluk. Örn:0.25em}
\NormalTok{\{\textless{}manuel başlık\textgreater{}\}}\CommentTok{\% Elle başlık belirtilir.}
\end{Highlighting}
\end{Shaded}

Boş bırakılan herhangi bir değişken olursa varsayılan değerler alınır.
Son satırdaki \texttt{\textless{}manuel\ başlık\textgreater{}} değişkeni
\texttt{\textbackslash{}thmname}, \texttt{\textbackslash{}thmnumber} ve
\texttt{\textbackslash{}thmnote} komutlarıyla biçimlendirilir. Birinci
komut başlığı, ikinci komut numarayı, üçüncüsü ise notu biçimlendirmek
içindir.

Not değişkeni her zaman isteğe bağlıdır, ancak başlık oluşturulurken
\texttt{\textbackslash{}thmnote} komutuyla belirtilmezse varsayılan
olarak görünmeyecektir.

\begin{Shaded}
\begin{Highlighting}[]
\BuiltInTok{\textbackslash{}documentclass}\NormalTok{\{}\ExtensionTok{article}\NormalTok{\}}
\BuiltInTok{\textbackslash{}usepackage}\NormalTok{[T1]\{}\ExtensionTok{fontenc}\NormalTok{\}}
\BuiltInTok{\textbackslash{}usepackage}\NormalTok{[turkish]\{}\ExtensionTok{babel}\NormalTok{\}}
\BuiltInTok{\textbackslash{}usepackage}\NormalTok{\{}\ExtensionTok{amsmath,amssymb,amsfonts}\NormalTok{\}}
\BuiltInTok{\textbackslash{}usepackage}\NormalTok{\{}\ExtensionTok{amsthm}\NormalTok{\}}
\FunctionTok{\textbackslash{}newtheoremstyle}\NormalTok{\{benimstilim\}}
\NormalTok{\{5pt\}}\CommentTok{\% }
\NormalTok{\{5pt\}}\CommentTok{\% }
\NormalTok{\{}\FunctionTok{\textbackslash{}normalfont}\NormalTok{\}}\CommentTok{\% }
\NormalTok{\{\} }\CommentTok{\%}
\NormalTok{\{}\FunctionTok{\textbackslash{}bfseries\textbackslash{}sffamily}\NormalTok{\}}\CommentTok{\%}
\NormalTok{\{}\FunctionTok{\textbackslash{};}\NormalTok{\}}\CommentTok{\% }
\NormalTok{\{0.25em\}}\CommentTok{\% }
\NormalTok{\{}\FunctionTok{\textbackslash{}thmname}\NormalTok{\{\#1\} }\FunctionTok{\textbackslash{}thmnumber}\NormalTok{\{\#2\}}\FunctionTok{\textbackslash{}thmnote}\NormalTok{\{{-}{-}{-}\#3\}.\}}\CommentTok{\%}
\FunctionTok{\textbackslash{}theoremstyle}\NormalTok{\{benimstilim\}}\CommentTok{\%}
\FunctionTok{\textbackslash{}newtheorem}\NormalTok{\{sonuc\}\{Sonuç\}}\CommentTok{\%}

\KeywordTok{\textbackslash{}begin}\NormalTok{\{}\ExtensionTok{document}\NormalTok{\}}

\KeywordTok{\textbackslash{}section}\NormalTok{\{Özel Stiller\}}

\KeywordTok{\textbackslash{}begin}\NormalTok{\{}\ExtensionTok{sonuc}\NormalTok{\}[Özel]}
\NormalTok{ Sonucu görüyor olmalısınız.}
\KeywordTok{\textbackslash{}end}\NormalTok{\{}\ExtensionTok{sonuc}\NormalTok{\}}



\KeywordTok{\textbackslash{}end}\NormalTok{\{}\ExtensionTok{document}\NormalTok{\}}
\end{Highlighting}
\end{Shaded}

\bookmarksetup{startatroot}

\hypertarget{yuxfczer-gezer-nesneler}{%
\chapter{Yüzer Gezer Nesneler}\label{yuxfczer-gezer-nesneler}}

\hypertarget{tablolar}{%
\section{Tablolar}\label{tablolar}}

\hypertarget{ux15fekiller}{%
\section{Şekiller}\label{ux15fekiller}}

\bookmarksetup{startatroot}

\hypertarget{uxf6zel-sayfalar}{%
\chapter{Özel Sayfalar}\label{uxf6zel-sayfalar}}

Bu bölümde bir kitaptaki özel sayfalar olan kaynakça ve dizinden
bahsedeceğiz.

\hypertarget{kaynakuxe7a}{%
\section{Kaynakça}\label{kaynakuxe7a}}

\hypertarget{buxfctuxfcnleux15fik-kaynakuxe7a}{%
\subsection{Bütünleşik
Kaynakça}\label{buxfctuxfcnleux15fik-kaynakuxe7a}}

{\LaTeX}'de kaynakça oluşturmanın bir yolu, kaynakçayı kaynak dosyanızın
(\texttt{.tex} uzantılı ana dosyanız) içindeki bir ortamda
hazırlamaktır. Kullanacağınız ortam \texttt{thebibliography} ortamıdır.

\begin{Shaded}
\begin{Highlighting}[]
\KeywordTok{\textbackslash{}begin}\NormalTok{\{}\ExtensionTok{thebibliography}\NormalTok{\}\{\textless{}sayı\textgreater{}\}}

\KeywordTok{\textbackslash{}end}\NormalTok{\{}\ExtensionTok{thebibliography}\NormalTok{\}}
\end{Highlighting}
\end{Shaded}

Ortam komutundaki \texttt{\textless{}sayı\textgreater{}} değişkeni,
kaynağın etiketi veya etiket girilmediği takdirde verilen sıra
numarasının kaç karakter uzunluğunda olacağını belirtir. Örneğin ortam

\begin{Shaded}
\begin{Highlighting}[]
\KeywordTok{\textbackslash{}begin}\NormalTok{\{}\ExtensionTok{thebibliography}\NormalTok{\}\{9\}}
        
\KeywordTok{\textbackslash{}end}\NormalTok{\{}\ExtensionTok{thebibliography}\NormalTok{\}}
\end{Highlighting}
\end{Shaded}

şeklinde oluşturulursa etiket veya etiket girilmediği takdirde verilen
sıra numarası için bir karakter uzunluğunda yer ayrılması gerektiği ve
toplamda bu ortama en fazla dokuz adet kaynak girileceği belirtilmiş
olur. Eğer dokuzdan fazla kaynak girilecekse, 99 kaynağa kadar izin
veren ``45'' gibi iki basamaklı bir sayı girilebilir.

Ortama her kaynak \texttt{\textbackslash{}bibitem} komutuyla eklenir ve
komuttan sonra kaynağı tanımlayıcı bilgiler girilir. Bu bilgiler
girilirken biçim elle oluşturulur.

\begin{Shaded}
\begin{Highlighting}[]
\FunctionTok{\textbackslash{}bibitem}\NormalTok{[\textless{}etiket\textgreater{}]\{\textless{}anahtar\textgreater{}\}}
\end{Highlighting}
\end{Shaded}

Komutun zorunlu değişkeni olan
\texttt{\textless{}anahtar\textgreater{}}, ileride kaynağa atıf yapmak
için kullanacağınız bir tanımlayıcıdır ve her kaynak için benzersiz
olmalıdır. Genelde akılda kolay kalması için yazarın soyadı ve yayın
yılı olacak şekilde düzenlenir.

Zorunlu olmayan \texttt{\textless{}etiket\textgreater{}} değişkeni
girilmediği takdirde kaynağın önüne köşeli parantezler içinde kaynağın
sıra numarası yazdırılır.

Kaynağın sıra numarasının köşeli parantezler içinde yazılması istenmezse
aşağıdaki komutlarla değişiklik yapılabilir.

\begin{Shaded}
\begin{Highlighting}[]
\FunctionTok{\textbackslash{}makeatletter}
\FunctionTok{\textbackslash{}renewcommand}\NormalTok{\{}\ExtensionTok{\textbackslash{}@biblabel}\NormalTok{\}[1]\{}\FunctionTok{\textbackslash{}textbf}\NormalTok{\{\#1.\}\}}
\FunctionTok{\textbackslash{}makeatother}
\end{Highlighting}
\end{Shaded}

Bu komut verilirse sıra numaraları parantezsiz, kalın ve ardında nokta
olacak şekilde yazılır.

Ortam genelde \texttt{\textbackslash{}end\{document\}} komutundan hemen
önce oluşturulur ve ortamın oluşturulduğu yerde {\LaTeX}, \texttt{book}
ve \texttt{report} sınıflarında eğer Türkçe dil paketi eklenmişse
``Kaynakça'', \texttt{article} sınıfında ise ``Kaynaklar'' ismini ve
ardından kaynakları yazdırır.

Kaynaklardan herhangi birine atıf \texttt{\textbackslash{}cite}
komutuyla yapılır.

\begin{Shaded}
\begin{Highlighting}[]
\KeywordTok{\textbackslash{}cite}\NormalTok{[\textless{}seçenekler\textgreater{}]\{}\ExtensionTok{\textless{}anahtar\textgreater{}}\NormalTok{\}}
\end{Highlighting}
\end{Shaded}

Komutun zorunlu değişkeni olan
\texttt{\textless{}anahtar\textgreater{}}, atıf yapılmak istenen
kaynağın \texttt{\textbackslash{}bibitem} komutundaki zorunlu
değişkenidir. İsteğe bağlı \texttt{\textless{}seçenekler\textgreater{}}
değişkeninde ise sayfa numarası, bölüm numarası gibi fazladan
vurgulanmak istenen bilgiler girilebilir.

Atıf yapılan yerde kaynağın etiketi ya da etiket girilmediği takdirde
sıra numarası köşeli parantez içinde yazdırılır. Eğer fazladan yapılan
vurgu varsa, bu, etiket ya da sıra numarasının devamında virgülden sonra
yazdırılır.

Aynı yerde birden fazla kaynağa atıf yapılacaksa atıf yapılacak
kaynakların anahtarları aralarına virgül koyularak
\texttt{\textbackslash{}cite\{\textless{}anahtar1\textgreater{},\textless{}anahtar2\textgreater{},\textless{}anahtar3\textgreater{}\}}
şeklinde yazılır.

\begin{Shaded}
\begin{Highlighting}[]
\BuiltInTok{\textbackslash{}documentclass}\NormalTok{\{}\ExtensionTok{article}\NormalTok{\}}
\BuiltInTok{\textbackslash{}usepackage}\NormalTok{[T1]\{}\ExtensionTok{fontenc}\NormalTok{\}}
\BuiltInTok{\textbackslash{}usepackage}\NormalTok{[turkish]\{}\ExtensionTok{babel}\NormalTok{\}}
\FunctionTok{\textbackslash{}title}\NormalTok{\{}\FunctionTok{\textbackslash{}LaTeX}\NormalTok{\textquotesingle{}de Kaynakça Yönetimi 1: Bütünleşik Kaynakça\}}
\FunctionTok{\textbackslash{}author}\NormalTok{\{Zafer Acar\}}

\KeywordTok{\textbackslash{}begin}\NormalTok{\{}\ExtensionTok{document}\NormalTok{\}}
\FunctionTok{\textbackslash{}maketitle}

\NormalTok{WYSIWYG editörleri yerine, }\FunctionTok{\textbackslash{}TeX}\NormalTok{/}\FunctionTok{\textbackslash{}LaTeX}\NormalTok{\{\}  }\KeywordTok{\textbackslash{}cite}\NormalTok{\{}\ExtensionTok{lamport94}\NormalTok{\}}
\NormalTok{dizgi sistemini kullanmaya başlayın. Görüldüğü gibi kaynakça oluşturmak }
\NormalTok{ve atıf yapmak oldukça kolaydır.}

\NormalTok{Ali Nesin, }\KeywordTok{\textbackslash{}cite}\NormalTok{[s.\textasciitilde{}47]\{}\ExtensionTok{nesin07}\NormalTok{\}\textquotesingle{}de pokerin matematiğini anlatıyor.}

\NormalTok{İki kaynağa birden atıf }\KeywordTok{\textbackslash{}cite}\NormalTok{\{}\ExtensionTok{lamport94,nesin07}\NormalTok{\} şeklinde yapılır.}

\KeywordTok{\textbackslash{}begin}\NormalTok{\{}\ExtensionTok{thebibliography}\NormalTok{\}\{9\}}
\FunctionTok{\textbackslash{}bibitem}\NormalTok{\{lamport94\}}
\NormalTok{Leslie Lamport,}
\FunctionTok{\textbackslash{}textit}\NormalTok{\{}\FunctionTok{\textbackslash{}LaTeX}\NormalTok{: a document preparation system\}, Addison Wesley,}
\NormalTok{Massachusetts, 2nd edition, 1994.}

\FunctionTok{\textbackslash{}bibitem}\NormalTok{[N]\{nesin07\} Ali Nesin, }\FunctionTok{\textbackslash{}textit}\NormalTok{\{Matematik ve Oyun\}, }
\NormalTok{ Nesin Yayıncılık, 2007.}
\KeywordTok{\textbackslash{}end}\NormalTok{\{}\ExtensionTok{thebibliography}\NormalTok{\}}
\KeywordTok{\textbackslash{}end}\NormalTok{\{}\ExtensionTok{document}\NormalTok{\}}
\end{Highlighting}
\end{Shaded}

\hypertarget{kaynakuxe7anux131n-ayrux131-dosyada-hazux131rlanmasux131}{%
\subsection{Kaynakçanın ayrı dosyada
hazırlanması}\label{kaynakuxe7anux131n-ayrux131-dosyada-hazux131rlanmasux131}}

Kaynakçayı ayrı bir dosyada hazırlayıp {\TeX} dağıtımlarıyla hazır
olarak gelen \textsc{Bib}{\TeX} programıyla yazdırabiliriz.

Bu yöntemde kaynakça, uzantısı \texttt{.bib} olan ayrı bir dosyada
hazırlanır. Bu dosya basit bir metin dosyası olup, metin editörü ya da
{\LaTeX} editörü kullanılarak oluşturulabilir, düzenlenebilir. Ayrıca
\href{https://www.mendeley.com/?interaction_required=true}{Mendeley} ya
da \href{https://www.jabref.org/}{Jabref} gibi akademik referans
düzenleme programlarından da yararlanılabilir.

Bu yöntemin önemli avantajları vardır:

\begin{enumerate}
\def\labelenumi{\arabic{enumi}.}
\tightlist
\item
  Biçimlendirme otomatik yapılır. Eğer çalışmanızı yayımlayacak dergi ya
  da yayınevi kaynakçayı farklı bir formatta isterse her kaynağı tek tek
  elle biçimlendirmek zorunda kalmazsınız. Basit bir komut işinizi
  görür.
\item
  Dosyayı bir kere oluşturur ve sonra başka çalışmalarda
  kullanabilirsiniz.
\item
  \href{https://scholar.google.com.tr/}{Google Akademik},
  \href{https://books.google.com.tr/}{Google Kitaplar} ve
  \href{http://www.dergipark.org.tr/tr/}{DergiPark} gibi platformlardan
  kullandığınız kaynakların \textsc{Bib}{\TeX} kodunu çekebilirsiniz
  (bkz. Şekil @ref(fig:fig-google)).
\item
  Yukarıda da bahsettiğimiz gibi Mendeley ve Jabref gibi akademik atıf
  düzenleme programlarını kullanarak kaynakların \textsc{Bib}{\TeX}
  kodunu oluşturabilir, düzenleyebilirsiniz.
\end{enumerate}

\begin{figure}

\begin{minipage}[t]{0.33\linewidth}

{\centering 

\raisebox{-\height}{

\includegraphics{./images/galinti1.png}

}

}

\subcaption{\label{fig-g1}Alıntı yap}
\end{minipage}%
%
\begin{minipage}[t]{0.33\linewidth}

{\centering 

\raisebox{-\height}{

\includegraphics{./images/galinti2.png}

}

}

\subcaption{\label{fig-g2}\textsc{Bib}{\TeX}}
\end{minipage}%
%
\begin{minipage}[t]{0.33\linewidth}

{\centering 

\raisebox{-\height}{

\includegraphics{./images/galinti3.png}

}

}

\subcaption{\label{fig-g3}Kodu kopyala}
\end{minipage}%

\caption{\label{fig-google}Google Akademik alıntı yapma}

\end{figure}

\hypertarget{dosyanux131n-hazux131rlanmasux131}{%
\subsubsection{Dosyanın
hazırlanması}\label{dosyanux131n-hazux131rlanmasux131}}

Aşağıda \texttt{.bib} uzantılı bir dosya örneği gösterilmiştir.

\begin{Shaded}
\begin{Highlighting}[]
\VariableTok{@book}\NormalTok{\{}\OtherTok{lang13}\NormalTok{,}
    \DataTypeTok{title}\NormalTok{=\{Algebraic number theory\},}
    \DataTypeTok{author}\NormalTok{=\{Lang, Serge\},}
    \DataTypeTok{volume}\NormalTok{=\{110\},}
    \DataTypeTok{year}\NormalTok{=\{2013\},}
    \DataTypeTok{publisher}\NormalTok{=\{Springer Science }\CharTok{\textbackslash{}\&}\NormalTok{ Business Media\},}
\NormalTok{    \}}
\VariableTok{@article}\NormalTok{\{}\OtherTok{lamport78}\NormalTok{,}
    \DataTypeTok{title}\NormalTok{=\{Time, clocks, and the ordering of events in a}
\NormalTok{        distributed system\},}
    \DataTypeTok{author}\NormalTok{=\{Lamport, Leslie\},}
    \DataTypeTok{journal}\NormalTok{=\{Communications of the ACM\},}
    \DataTypeTok{volume}\NormalTok{=\{21\},}
    \DataTypeTok{number}\NormalTok{=\{7\},}
    \DataTypeTok{pages}\NormalTok{=\{558{-}{-}565\},}
    \DataTypeTok{year}\NormalTok{=\{1978\},}
    \DataTypeTok{publisher}\NormalTok{=\{ACM\},}
\NormalTok{\}}
\VariableTok{@manual}\NormalTok{\{}\OtherTok{Oetiker06}\NormalTok{,}
    \DataTypeTok{author}\NormalTok{ = \{Oetiker, Tobias and Partl, Hubert and Hyna, Irene}
\NormalTok{        and Schlegl, Elisabeth\},}
    \DataTypeTok{title}\NormalTok{  = \{İnce bir \{}\CharTok{\textbackslash{}LaTeXe}\NormalTok{\} Elkitabı veya, 116 dakikada}
\NormalTok{        \{}\CharTok{\textbackslash{}LaTeXe}\NormalTok{\}\},}
    \DataTypeTok{note}\NormalTok{   = \{Türkçesi: Bekir Karaoğlu\},}
    \DataTypeTok{url}\NormalTok{    = \{http://ftp.ntua.gr/mirror/ctan/info/lshort/turkish/}
\NormalTok{        lshort{-}tr.pdf\},}
    \DataTypeTok{year}\NormalTok{   = \{2006\},}
\NormalTok{\}}
\end{Highlighting}
\end{Shaded}

Bu dosyada Serge Lang'a ait bir kitap (\texttt{@book}), Leslie Lamport'a
ait bir makale (\texttt{@article}) ve {\LaTeX} için bir teknik kılavuz
(\texttt{@manual}) vardır.

Her kaynağın ilk olarak \texttt{@} işaretiyle türü belirtilir.
Yukarıdakilere ek olarak rapor için \texttt{@report}, tez için
\texttt{@thesis}, çevrimiçi kaynaklar için \texttt{@online} kullanılır.
Bunların dışındaki birçok türe {\LaTeX} editörlerinin menü
çubuğuklarında bulunan ``Kaynakça (Bibliography)'' menüsünden
ulaşılabilir.

İlk girdi (\texttt{lang13}, \texttt{lamport78}, \texttt{Oetiker06})
kaynağa atıf yapmak için kullanılan anahtardır. Sonrasında gelenler de
tahmin edilebileceği gibi başlık (\texttt{title}), yazar
(\texttt{author}), yayıncı (\texttt{publisher}), yıl (\texttt{year}),
dergi (\texttt{journal}), cilt (\texttt{volume})\ldots{} gibi kaynağı
tanımlayan bilgilerdir. Bu tanımlamaların her biri eşittir işaretinden
sonra iki çengelli parantez arasında yapılır (çift tırnak da
kullanılabilir) ve her tanımlama (sonuncusu olsa dahi) virgülle ayrılır.

Yazar adı ya

\begin{Shaded}
\begin{Highlighting}[]
\CommentTok{author=\{Adı Soyadı\}}
\end{Highlighting}
\end{Shaded}

ya da

\begin{Shaded}
\begin{Highlighting}[]
\CommentTok{author=\{Soyadı, Adı\}}
\end{Highlighting}
\end{Shaded}

şeklinde girilmelidir ve birden fazla yazar varsa yazarlar yukarıdaki
yazımdan dolayı virgülle değil \texttt{and} ile ayrılmalıdır. Yazarları
ayırmak için virgül kullanırsanız yüksek ihtimalle {\LaTeX}, yazarların
adları ve soyadlarını karıştıracaktır.

Bir diğer önemli nokta özel kelimeleri yazmak için kullanılan komutları
ve aksanlı harfleri iki çengelli parantez içinde yazmaktır. Örneğin
``â'' için \texttt{\{\textbackslash{}\^{}a\}} yazılmalıdır. Genel olarak
sorun yaşanan karakterleri iki çengelli parantez içine yazmak gerekir.

Her tür için zorunlu olarak belirtilmesi gereken bilgiler ve isteğe
bağlı bilgiler vardır. Bunların ne olduklarını tahmin etmek zor
değildir. Bu konuda editörden de yararlanabilirsiniz. Örneğin,
\texttt{.bib} uzantılı dosyayı açıp editörde ``Kaynakça \(\rightarrow\)
Tez'' yolunu izlerseniz aşağıdaki listeyi yazdıracaktır.

\begin{Shaded}
\begin{Highlighting}[]
\VariableTok{@thesis}\NormalTok{\{}\OtherTok{ID}\NormalTok{,}
    \DataTypeTok{author}\NormalTok{ = \{author\},}
    \DataTypeTok{title}\NormalTok{ = \{title\},}
    \DataTypeTok{type}\NormalTok{ = \{type\},}
    \DataTypeTok{institution}\NormalTok{ = \{institution\},}
    \DataTypeTok{date}\NormalTok{ = \{date\},}
    \DataTypeTok{OPTsubtitle}\NormalTok{ = \{subtitle\},}
    \DataTypeTok{OPTtitleaddon}\NormalTok{ = \{titleaddon\},}
    \DataTypeTok{OPTlanguage}\NormalTok{ = \{language\},}
    \DataTypeTok{OPTnote}\NormalTok{ = \{note\},}
    \DataTypeTok{OPTlocation}\NormalTok{ = \{location\},}
    \DataTypeTok{OPTmonth}\NormalTok{ = \{month\},}
    \DataTypeTok{OPTisbn}\NormalTok{ = \{isbn\},}
    \DataTypeTok{OPTchapter}\NormalTok{ = \{chapter\},}
    \DataTypeTok{OPTpages}\NormalTok{ = \{pages\},}
    \DataTypeTok{OPTpagetotal}\NormalTok{ = \{pagetotal\},}
    \DataTypeTok{OPTaddendum}\NormalTok{ = \{addendum\},}
    \DataTypeTok{OPTpubstate}\NormalTok{ = \{pubstate\},}
    \DataTypeTok{OPTdoi}\NormalTok{ = \{doi\},}
    \DataTypeTok{OPTeprint}\NormalTok{ = \{eprint\},}
    \DataTypeTok{OPTeprintclass}\NormalTok{ = \{eprintclass\},}
    \DataTypeTok{OPTeprinttype}\NormalTok{ = \{eprinttype\},}
    \DataTypeTok{OPTurl}\NormalTok{ = \{url\},}
    \DataTypeTok{OPTurldate}\NormalTok{ = \{urldate\},}
\NormalTok{\}}
\end{Highlighting}
\end{Shaded}

Görüldüğü gibi ilk altı satır zorunlu, OPT ile başlayanlar isteğe
bağlıdır. İsteğe bağlı olanlardan belirtmek istediklerinizin başında
bulunan OPT'yi silip tanımlamayı yapabilirsiniz.

Editörden yararlanmanın diğer bir yolu ``Kaynakça \(\rightarrow\)
Kaynakça Kaydı Ekle\ldots{}'' yolunu izlemektir. Bu yolu izlediğinizde
aşağıdaki pencere açılır (örnek TeXstudio editörüne aittir).

\includegraphics{./images/tex-studio.png}

Pencerenin solunda kaydı eklemek istediğiniz dosyayı ve ortada kayıt
türünü belirtir, sağda da kaynağın bilgilerini girersiniz. Zorunlu
bilgiler en üstte yer alan kalın yazılmış olanlardır.

\hypertarget{kaynakuxe7anux131n-yazdux131rux131lmasux131}{%
\subsubsection{Kaynakçanın
yazdırılması}\label{kaynakuxe7anux131n-yazdux131rux131lmasux131}}

Kaynakçayı yazdırmak için \textsc{Bib}{\TeX}'i kullanacağız.
\textsc{Bib}{\TeX}'in {\LaTeX}'le standart olarak geldiğini ifade
etmiştik. Dolayısıyla bu programı kullanmak için ek bir şey yapmanız
gerekmez.

Oluşturulan \texttt{.bib} uzantılı dosya
\texttt{\textbackslash{}bibliography} komutuyla içeri aktarılır,
\texttt{\textbackslash{}bibliographystyle} komutuyla da kullanılacak
biçim belirtilir.

\begin{Shaded}
\begin{Highlighting}[]
\BuiltInTok{\textbackslash{}bibliographystyle}\NormalTok{\{}\ExtensionTok{\textless{}biçim\textgreater{}}\NormalTok{\}}
\BuiltInTok{\textbackslash{}bibliography}\NormalTok{\{}\ExtensionTok{\textless{}dosya\textgreater{}}\NormalTok{\}}
\end{Highlighting}
\end{Shaded}

Burada yer alan \texttt{\textless{}dosya\textgreater{}} uzantısının
belirtilmesine gerek yoktur. Dosyanın \texttt{kaynakca.bib} olduğunu
varsayarak, komut \texttt{\textbackslash{}bibliography\{kaynakca\}}
şeklinde verilir. Kullanılabilecek biçimler \texttt{abbrv},
\texttt{acm}, \texttt{alpha}, \texttt{apalike}, \texttt{ieeetr},
\texttt{plain}, \texttt{siam} ve \texttt{unsrt}'dir. Biçimlerin nasıl
çıktı verdiklerini görmek için
\href{https://tr.overleaf.com/learn/latex/Bibtex_bibliography_styles}{şuraya}
bakabilirsiniz.

Atıf, bütünleşik kaynakçada olduğu gibi \texttt{\textbackslash{}cite}
komutuyla yapılır fakat bütünleşik kaynakçadan farklı olarak atıf
yapılmayan kaynaklar yazdırılmaz. Bazı kaynakların bu kuraldan ayrı
tutulması istenirse \texttt{\textbackslash{}nocite} komutu, değişkenine
kaynağın anahtarı yazılarak \texttt{\textbackslash{}bibliography}
komutundan önce verilmelidir.

\begin{Shaded}
\begin{Highlighting}[]
\KeywordTok{\textbackslash{}nocite}\NormalTok{\{}\ExtensionTok{\textless{}anahtar\textgreater{}}\NormalTok{\}}
\end{Highlighting}
\end{Shaded}

Eğer tüm kaynakların bu kuraldan ayrı tutulması isteniyorsa komut
\texttt{\textbackslash{}nocite\{*\}} şeklinde verilmelidir.

Kaynakçanın belgeye yazılması için kaynak dosyanın derlenip,
\textsc{Bib}{\TeX} programının çalıştırılması ve ardından dosyanın en az
iki kere daha derlenmesi gerekir. \textsc{Bib}{\TeX} programı, editörde
``Araçlar \(\rightarrow\) Kaynakça'' yoluyla çalıştırılır (klavye kısa
yolu F8). Aynı şey, uçbirimde sırasıyla

\begin{Shaded}
\begin{Highlighting}[]
\ExtensionTok{pdflatex}\NormalTok{ kaynakdosya}
\ExtensionTok{bibtex}\NormalTok{ kaynakdosya}
\ExtensionTok{pdflatex}\NormalTok{ kaynakdosya}
\ExtensionTok{pdflatex}\NormalTok{ kaynakdosya}
\end{Highlighting}
\end{Shaded}

komutları çalıştırılarak yapılabilir.

Aşağıda kaynak dosya örneği verilmiştir. Bu dosyayı derleyebilmeniz için
içeriği yukarıda verilen \texttt{kaynakca.bib} dosyasının bu dosyayla
aynı dizinde olması gerektiğini unutmayınız.

\begin{Shaded}
\begin{Highlighting}[]
\BuiltInTok{\textbackslash{}documentclass}\NormalTok{[10pt,a4paper]\{}\ExtensionTok{article}\NormalTok{\}}
\BuiltInTok{\textbackslash{}usepackage}\NormalTok{[T1]\{}\ExtensionTok{fontenc}\NormalTok{\}}
\BuiltInTok{\textbackslash{}usepackage}\NormalTok{[turkish]\{}\ExtensionTok{babel}\NormalTok{\}}
\BuiltInTok{\textbackslash{}usepackage}\NormalTok{\{}\ExtensionTok{dtk{-}logos}\NormalTok{\} }\CommentTok{\% \textbackslash{}BibTeX komutu için...}
\FunctionTok{\textbackslash{}title}\NormalTok{\{Kaynakça Yönetimi 2: }\FunctionTok{\textbackslash{}BibTeX}\NormalTok{\}}
\FunctionTok{\textbackslash{}author}\NormalTok{\{Zafer Acar\}}
\KeywordTok{\textbackslash{}begin}\NormalTok{\{}\ExtensionTok{document}\NormalTok{\}}
\FunctionTok{\textbackslash{}maketitle}
\NormalTok{Lang\textquotesingle{}ın kitabı }\KeywordTok{\textbackslash{}cite}\NormalTok{\{}\ExtensionTok{lang13}\NormalTok{\}, Lamport\textquotesingle{}un makalesi  }\KeywordTok{\textbackslash{}cite}\NormalTok{\{}\ExtensionTok{lamport78}\NormalTok{\} }
\NormalTok{ve }\FunctionTok{\textbackslash{}LaTeX}\NormalTok{\{\} için Türkçe kaynak }\KeywordTok{\textbackslash{}cite}\NormalTok{\{}\ExtensionTok{Oetiker06}\NormalTok{\} }\FunctionTok{\textbackslash{}dots}

\BuiltInTok{\textbackslash{}bibliographystyle}\NormalTok{\{}\ExtensionTok{siam}\NormalTok{\}}
\BuiltInTok{\textbackslash{}bibliography}\NormalTok{\{}\ExtensionTok{kaynakca}\NormalTok{\}}
\KeywordTok{\textbackslash{}end}\NormalTok{\{}\ExtensionTok{document}\NormalTok{\}}
\end{Highlighting}
\end{Shaded}

\hypertarget{dizin}{%
\section{Dizin}\label{dizin}}

Bilimsel bir yapıtta bulunması gereken \emph{dizin} ya da diğer adıyla
\emph{indeks}, bir yapıtın kişi, konu, yer adı vb. bakımından
içindekileri yer numarasıyla belirten ve yapıtın arkasında yer alan
alfabetik listedir.

{\LaTeX}'de dizin oluşturabilmek için sahanlığa

\begin{Shaded}
\begin{Highlighting}[]
\BuiltInTok{\textbackslash{}usepackage}\NormalTok{\{}\ExtensionTok{makeidx}\NormalTok{\}}
\FunctionTok{\textbackslash{}makeindex}
\end{Highlighting}
\end{Shaded}

komutları girilir. Birinci komut dizin için gerekli olan
\href{http://ftp.ntua.gr/mirror/ctan/macros/latex/base/makeindx.pdf}{\textbf{makeidx}}
paketini çağırır, ikinci komut ise dizinleme komutlarını etkinleştirir.

Dizinde gösterilmek istenen madde, \texttt{\textbackslash{}index}
komutunun değişkeni olarak girilir:

\begin{Shaded}
\begin{Highlighting}[]
\FunctionTok{\textbackslash{}index}\NormalTok{\{\textless{}madde\textgreater{}\}}
\end{Highlighting}
\end{Shaded}

Dizin maddesi girme örnekleri aşağıda gösterilmiştir.

\includegraphics{./images/dizina.png}

{\LaTeX}, kaynak dosyanızı derlediğinizde bu dizin maddelerini sayfa
numaralarıyla birlikte, kaynak dosyayla adı aynı fakat uzantısı
\texttt{.idx} olan bir dosyaya kaydeder (bu dosyaya \emph{ham dosya}
denir). Bu dosyanın \texttt{makeindex} programından geçirilmesi gerekir.
Bu editörde ``Araçlar \(\rightarrow\) Dizin'' yoluyla yapılır. Aynı şey
uçbirimde,

\begin{Shaded}
\begin{Highlighting}[]
\ExtensionTok{makeindex}\NormalTok{ kaynakdosya}
\end{Highlighting}
\end{Shaded}

komutu girilerek yapılabilir. Dosya tekrar derlendiğinde sıralanmış
dizin belgeye yazılır. Bunun için dizinin yazılması istenen yere
\texttt{\textbackslash{}printindex} komutu verilir. Bu genelde, belgenin
en sonunda \texttt{\textbackslash{}end\{document\}} komutundan hemen
öncedir. Komutun girildiği yere {\LaTeX}, Türkçe dil paketi ekli
belgelerde ``Dizin'' başlığını oluşturur ve belgede
\texttt{\textbackslash{}index} komutuyla eklenmiş maddeleri sırayla
dizer.

Program, ham dosyayı işleyip dizin maddelerini abece sırasına göre dizer
ve \texttt{.ind} uzantılı bir dosyaya aktarır. Ancak, Türkçe aksanlı
harflerle başlayan kelimeler doğru yerde yazılmazlar. Bu harflerin doğru
yere yazılması için \texttt{.ind} uzantılı dosyanın metin editörüyle
açılarak elle düzenlenmesi gerekir. Ardından kaynak dosya derlenir. Elle
düzeltmeden sonra tekrar \texttt{makeindex} programı çalıştırılırsa
\texttt{.ind} uzantılı dosya tekrar oluşturulacağı için elle yapılan
düzeltmeler bozulur. O yüzden düzeltme en son yapılmalıdır.

\begin{tcolorbox}[enhanced jigsaw, opacitybacktitle=0.6, coltitle=black, leftrule=.75mm, rightrule=.15mm, toprule=.15mm, bottomtitle=1mm, titlerule=0mm, colbacktitle=quarto-callout-note-color!10!white, breakable, arc=.35mm, opacityback=0, colframe=quarto-callout-note-color-frame, toptitle=1mm, title=\textcolor{quarto-callout-note-color}{\faInfo}\hspace{0.5em}{Not}, bottomrule=.15mm, left=2mm, colback=white]
Aksanlı harflerle başlayan kelimelerin doğru yerde yazılmaları için
aksanlı madde girme komutundan faydalanılabilir. Örneğin ``çekiç''
kelimesinin peşine \texttt{\textbackslash{}index\{czekiç@çekiç\}} komutu
verilirse bu kelime doğru yerde dizilecektir. Burada yapılan sorun
yaratan ``ç'' harfi yerine ``cz'' yazılmasıdır.
\end{tcolorbox}

\hypertarget{uxe7oklu-dizin}{%
\subsection{Çoklu Dizin}\label{uxe7oklu-dizin}}

Birden fazla dizin oluşturmak isterseniz (örneğin biri \emph{normal
dizin} diğeri de \emph{simgeler dizini})
\href{http://ftp.cc.uoc.gr/mirrors/CTAN/macros/latex/contrib/index/index.pdf}{\textbf{index}}
paketini kullanabilirsiniz. Her dizin paket eklendikten ve
\texttt{\textbackslash{}makeindex} komutu sahanlıkta verildikten sonra
\texttt{\textbackslash{}newindex} komutuyla tanıtılır.

\begin{Shaded}
\begin{Highlighting}[]
\BuiltInTok{\textbackslash{}usepackage}\NormalTok{\{}\ExtensionTok{index}\NormalTok{\}}
\FunctionTok{\textbackslash{}makeindex}
\FunctionTok{\textbackslash{}newindex}\NormalTok{\{normal\}\{ndx\}\{nnd\}\{Normal Dizin\}}
\FunctionTok{\textbackslash{}newindex}\NormalTok{\{simge\}\{sdx\}\{snd\}\{Simgeler Dizini\}}
\end{Highlighting}
\end{Shaded}

Komutun dört değişkeni vardır. Bunlar sırasıyla, dizin adı (örnekte
\texttt{normal} ve \texttt{simge}), oluşturulacak ham dosyanın uzantısı
(örnekte \texttt{.ndx} ve \texttt{.sdx}), \texttt{makeindex} tarafından
ham dosyanın işlenmesiyle oluşturulan dosyanın uzantısı (örnekte
\texttt{.nnd} ve \texttt{.snd}) ve son olarak
\texttt{\textbackslash{}printindex} komutuyla yazdırılacak başlıktır
(örnekte ``Normal Dizin'' ve ``Simgeler Dizini''). Buradaki uzantılar
varsayılan \texttt{.idx} ve \texttt{.ind} uzantılardan farklı olmalıdır.

Ardından bir kelime ya da simgeyi dizine eklemek için, eklemek istenilen
dizine göre \texttt{\textbackslash{}index} komutu köşeli parantezler
içinde dizin adı belirtilerek kullanılır.

\begin{Shaded}
\begin{Highlighting}[]
\FunctionTok{\textbackslash{}index}\NormalTok{[normal]\{kuvvet\}}
\FunctionTok{\textbackslash{}index}\NormalTok{[simge]\{F@}\SpecialStringTok{$}\SpecialCharTok{\textbackslash{}vec}\SpecialStringTok{\{F\}$}\NormalTok{\}}
\end{Highlighting}
\end{Shaded}

Birinci komut, ``kuvvet'' kelimesini normal dizine, ikinci komut
\(\vec{F}\) simgesini simgeler dizinine ekler.

Belge derlendikten sonra iki tane \texttt{\textbackslash{}makeindex}
komutu uçbirimde,

\begin{Shaded}
\begin{Highlighting}[]
\ExtensionTok{makeindex}\NormalTok{ kaynakdosya.ndx }\AttributeTok{{-}o}\NormalTok{ kaynakdosya.nnd }
\ExtensionTok{makeindex}\NormalTok{ kaynakdosya.sdx }\AttributeTok{{-}o}\NormalTok{ kaynakdosya.snd }
\end{Highlighting}
\end{Shaded}

şeklinde verilir. Belgenizde dizinlerin yazılması istenen yere de

\begin{Shaded}
\begin{Highlighting}[]
\FunctionTok{\textbackslash{}printindex}\NormalTok{[normal]}
\FunctionTok{\textbackslash{}printindex}\NormalTok{[simge]}
\end{Highlighting}
\end{Shaded}

komutları verilir. Ardından belge tekrar derlenerek dizinler yazdırılır.

Çoklu dizin için diğer bir seçenek
\href{https://www.ctan.org/pkg/multind}{\textbf{multind}} paketini
kullanmaktır. Görece index paketine göre daha pratiktir. Sahanlığa

\begin{Shaded}
\begin{Highlighting}[]
\BuiltInTok{\textbackslash{}usepackage}\NormalTok{\{}\ExtensionTok{multind}\NormalTok{\}}
\FunctionTok{\textbackslash{}makeindex}\NormalTok{\{normal\}}
\FunctionTok{\textbackslash{}makeindex}\NormalTok{\{simge\}}
\end{Highlighting}
\end{Shaded}

komutlarıyla \texttt{normal} ve \texttt{simge} adında iki dizin
tanımlanır. Yine dizine yazılması istenen maddeler
\texttt{\textbackslash{}index} komutundan önce çengelli parantezler
içinde dizin adı belirtilerek girilir.

\begin{Shaded}
\begin{Highlighting}[]
\FunctionTok{\textbackslash{}index}\NormalTok{\{normal\}\{kuvvet\}}
\FunctionTok{\textbackslash{}index}\NormalTok{\{simge\}\{F@}\SpecialStringTok{$}\SpecialCharTok{\textbackslash{}vec}\SpecialStringTok{\{F\}$}\NormalTok{\}}
\end{Highlighting}
\end{Shaded}

Bu defa \texttt{makeindex} programı uçbirimde

\begin{Shaded}
\begin{Highlighting}[]
\ExtensionTok{makeindex}\NormalTok{ normal}
\ExtensionTok{makeindex}\NormalTok{ simge}
\end{Highlighting}
\end{Shaded}

komutlarıyla çalıştırılır. Yine \texttt{\textbackslash{}printindex}
komutları dizinlerin eklenmesi istenen yere

\begin{Shaded}
\begin{Highlighting}[]
\FunctionTok{\textbackslash{}printindex}\NormalTok{\{normal\}\{Normal Dizin\}}
\FunctionTok{\textbackslash{}printindex}\NormalTok{\{simge\}\{Simgeler Dizini\}}
\end{Highlighting}
\end{Shaded}

şeklinde verilir.

\hypertarget{dizinin-iuxe7indekiler-tablosuna-yazux131lmasux131}{%
\subsection{Dizinin İçindekiler tablosuna
yazılması}\label{dizinin-iuxe7indekiler-tablosuna-yazux131lmasux131}}

Dizini İçindekiler tablosuna yazmak için
\texttt{\textbackslash{}printindex} komutunun peşine \texttt{book} ve
\texttt{report} sınıflarında \texttt{\textbackslash{}addcontentsline}
komutu,

\begin{Shaded}
\begin{Highlighting}[]
\FunctionTok{\textbackslash{}addcontentsline}\NormalTok{\{toc\}\{chapter\}\{}\FunctionTok{\textbackslash{}indexname}\NormalTok{\}}
\end{Highlighting}
\end{Shaded}

şeklinde, \texttt{article} sınıfında ise

\begin{Shaded}
\begin{Highlighting}[]
\FunctionTok{\textbackslash{}addcontentsline}\NormalTok{\{toc\}\{section\}\{}\FunctionTok{\textbackslash{}indexname}\NormalTok{\}}
\end{Highlighting}
\end{Shaded}

şeklinde verilmelidir. Çoklu dizin oluşturulmuşsa, \texttt{book} ve
\texttt{report} sınıflarında

\begin{Shaded}
\begin{Highlighting}[]
\FunctionTok{\textbackslash{}printindex}\NormalTok{\{normal\}\{Normal Dizin\}}
\FunctionTok{\textbackslash{}addcontentsline}\NormalTok{\{toc\}\{chapter\}\{Normal Dizin\}}
\FunctionTok{\textbackslash{}printindex}\NormalTok{\{simge\}\{Simgeler Dizini\}}
\FunctionTok{\textbackslash{}addcontentsline}\NormalTok{\{toc\}\{chapter\}\{Simgeler Dizini\}}
\end{Highlighting}
\end{Shaded}

şeklinde, \texttt{article} sınıfında ise komutlardaki \texttt{chapter}
yazan yere \texttt{section} yazarak verilmelidir.

\bookmarksetup{startatroot}

\hypertarget{uxf6zelleux15ftirmeler}{%
\chapter{Özelleştirmeler}\label{uxf6zelleux15ftirmeler}}

\hypertarget{varsayux131lan-sayfa-duxfczenini-deux11fiux15ftirme}{%
\section{Varsayılan Sayfa Düzenini
Değiştirme}\label{varsayux131lan-sayfa-duxfczenini-deux11fiux15ftirme}}

{\LaTeX}'de varsayılan kağıt boyutunun \texttt{letterpaper} olduğunu
Bölün @ref(belgesinifi)'de ifade etmiştik. Ayrıca aynı yazıda başka bir
kağıt boyutunun nasıl seçileceğine de yer vermiştik. Şimdi ise hem
sayfamızın kenar boşluklarının nasıl ayarlanacağından hem de ön tanımlı
olmayan, tamamen keyfi bir sayfa boyutunun nasıl belirleneceğinden
bahsedelim.

Bu tür sayfa düzenlemeleri için {\LaTeX}'de
\href{http://ftp.cc.uoc.gr/mirrors/CTAN/macros/latex/contrib/geometry/geometry.pdf}{\textbf{geometry}}
paketi kullanılır. Öncelikle paketi

\begin{Shaded}
\begin{Highlighting}[]
\BuiltInTok{\textbackslash{}usepackage}\NormalTok{\{}\ExtensionTok{geometry}\NormalTok{\}}
\end{Highlighting}
\end{Shaded}

komutuyla sahanlığa ekleyin. Ardından paket seçeneklerinde aşağıdaki
tanımlamalarla sayfanın düzenini değiştirebilirsiniz:

\begin{longtable}[]{@{}ll@{}}
\toprule()
Tanım & Değer \\
\midrule()
\endhead
\texttt{top} & üst boșluk \\
\texttt{bottom} & alt boșluk \\
\texttt{left} & sol boșluk \\
\texttt{right} & sağ boșluk \\
\texttt{paperwidht} & sayfa genișliği \\
\texttt{paperheight} & sayfa yüksekliği \\
\bottomrule()
\end{longtable}

Örneğin sahanlıkta

\begin{Shaded}
\begin{Highlighting}[]
\BuiltInTok{\textbackslash{}usepackage}\NormalTok{[paperwidth=175mm,paperheight=255mm,top=2cm,bottom=2cm,}
\NormalTok{left=2.5cm,right=2.5cm]\{}\ExtensionTok{geometry}\NormalTok{\}}
\end{Highlighting}
\end{Shaded}

komutunu verdiğinizde boyutu \(175\times 255\) mm, üst ve alt boşluğu
\(2\) cm, sol ve sağ boşluğu \(2.5\) cm olan bir sayfa düzeni
oluşturursunuz. Dilerseniz \texttt{paperwidth} ve\texttt{paperheight}
tanımlamalarını yerine, örneğin \texttt{a4paper} yazarak sadece kenar
boşluklarla ilgili tanımlamaları yapabilirsiniz.

\begin{tcolorbox}[enhanced jigsaw, opacitybacktitle=0.6, coltitle=black, leftrule=.75mm, rightrule=.15mm, toprule=.15mm, bottomtitle=1mm, titlerule=0mm, colbacktitle=quarto-callout-note-color!10!white, breakable, arc=.35mm, opacityback=0, colframe=quarto-callout-note-color-frame, toptitle=1mm, title=\textcolor{quarto-callout-note-color}{\faInfo}\hspace{0.5em}{Not}, bottomrule=.15mm, left=2mm, colback=white]
{\LaTeX}'de milimetre (mm) ve santimetre (cm) dışında inç (in), punto
(pt), em ve ex gibi ölçü birimleri de vardır. Bunlara ileride
değinilecektir. Ayrıca yine ileride daha ayrıntılı sayfa düzeni
oluşturmaktan da bahsedeceğiz. Bu aşamada bu kadarı yeterli olacaktır.
\end{tcolorbox}

\bookmarksetup{startatroot}

\hypertarget{kaynakuxe7a-1}{%
\chapter*{Kaynakça}\label{kaynakuxe7a-1}}
\addcontentsline{toc}{chapter}{Kaynakça}

\hypertarget{refs}{}
\begin{CSLReferences}{1}{0}
\leavevmode\vadjust pre{\hypertarget{ref-Oetiker}{}}%
Oetiker, Tobias, Hubert Partl, Irene Hyna, and Elisabeth Schlegl. 2006.
\emph{{İnce bir LaTeX2e Elkitabı veya, 116 dakikada LaTeX2e}}.

\end{CSLReferences}



\end{document}
