% Options for packages loaded elsewhere
\PassOptionsToPackage{unicode}{hyperref}
\PassOptionsToPackage{hyphens}{url}
%
\documentclass[
  10pt,
]{scrbook}
\usepackage{amsmath,amssymb}
\usepackage{lmodern}
\usepackage{iftex}
\ifPDFTeX
  \usepackage[T1]{fontenc}
  \usepackage[utf8]{inputenc}
  \usepackage{textcomp} % provide euro and other symbols
\else % if luatex or xetex
  \usepackage{unicode-math}
  \defaultfontfeatures{Scale=MatchLowercase}
  \defaultfontfeatures[\rmfamily]{Ligatures=TeX,Scale=1}
  \setmainfont[]{Source Serif Pro}
  \setsansfont[]{Source Sans Pro}
  \setmonofont[Scale=0.8]{Source Code Pro}
\fi
% Use upquote if available, for straight quotes in verbatim environments
\IfFileExists{upquote.sty}{\usepackage{upquote}}{}
\IfFileExists{microtype.sty}{% use microtype if available
  \usepackage[]{microtype}
  \UseMicrotypeSet[protrusion]{basicmath} % disable protrusion for tt fonts
}{}
\makeatletter
\@ifundefined{KOMAClassName}{% if non-KOMA class
  \IfFileExists{parskip.sty}{%
    \usepackage{parskip}
  }{% else
    \setlength{\parindent}{0pt}
    \setlength{\parskip}{6pt plus 2pt minus 1pt}}
}{% if KOMA class
  \KOMAoptions{parskip=half}}
\makeatother
\usepackage{xcolor}
\IfFileExists{xurl.sty}{\usepackage{xurl}}{} % add URL line breaks if available
\IfFileExists{bookmark.sty}{\usepackage{bookmark}}{\usepackage{hyperref}}
\hypersetup{
  pdftitle={LaTeX Dersleri},
  pdfauthor={Zafer Acar},
  hidelinks,
  pdfcreator={LaTeX via pandoc}}
\urlstyle{same} % disable monospaced font for URLs
\usepackage{color}
\usepackage{fancyvrb}
\newcommand{\VerbBar}{|}
\newcommand{\VERB}{\Verb[commandchars=\\\{\}]}
\DefineVerbatimEnvironment{Highlighting}{Verbatim}{commandchars=\\\{\}}
% Add ',fontsize=\small' for more characters per line
\usepackage{framed}
\definecolor{shadecolor}{RGB}{248,248,248}
\newenvironment{Shaded}{\begin{snugshade}}{\end{snugshade}}
\newcommand{\AlertTok}[1]{\textcolor[rgb]{0.94,0.16,0.16}{#1}}
\newcommand{\AnnotationTok}[1]{\textcolor[rgb]{0.56,0.35,0.01}{\textbf{\textit{#1}}}}
\newcommand{\AttributeTok}[1]{\textcolor[rgb]{0.77,0.63,0.00}{#1}}
\newcommand{\BaseNTok}[1]{\textcolor[rgb]{0.00,0.00,0.81}{#1}}
\newcommand{\BuiltInTok}[1]{#1}
\newcommand{\CharTok}[1]{\textcolor[rgb]{0.31,0.60,0.02}{#1}}
\newcommand{\CommentTok}[1]{\textcolor[rgb]{0.56,0.35,0.01}{\textit{#1}}}
\newcommand{\CommentVarTok}[1]{\textcolor[rgb]{0.56,0.35,0.01}{\textbf{\textit{#1}}}}
\newcommand{\ConstantTok}[1]{\textcolor[rgb]{0.00,0.00,0.00}{#1}}
\newcommand{\ControlFlowTok}[1]{\textcolor[rgb]{0.13,0.29,0.53}{\textbf{#1}}}
\newcommand{\DataTypeTok}[1]{\textcolor[rgb]{0.13,0.29,0.53}{#1}}
\newcommand{\DecValTok}[1]{\textcolor[rgb]{0.00,0.00,0.81}{#1}}
\newcommand{\DocumentationTok}[1]{\textcolor[rgb]{0.56,0.35,0.01}{\textbf{\textit{#1}}}}
\newcommand{\ErrorTok}[1]{\textcolor[rgb]{0.64,0.00,0.00}{\textbf{#1}}}
\newcommand{\ExtensionTok}[1]{#1}
\newcommand{\FloatTok}[1]{\textcolor[rgb]{0.00,0.00,0.81}{#1}}
\newcommand{\FunctionTok}[1]{\textcolor[rgb]{0.00,0.00,0.00}{#1}}
\newcommand{\ImportTok}[1]{#1}
\newcommand{\InformationTok}[1]{\textcolor[rgb]{0.56,0.35,0.01}{\textbf{\textit{#1}}}}
\newcommand{\KeywordTok}[1]{\textcolor[rgb]{0.13,0.29,0.53}{\textbf{#1}}}
\newcommand{\NormalTok}[1]{#1}
\newcommand{\OperatorTok}[1]{\textcolor[rgb]{0.81,0.36,0.00}{\textbf{#1}}}
\newcommand{\OtherTok}[1]{\textcolor[rgb]{0.56,0.35,0.01}{#1}}
\newcommand{\PreprocessorTok}[1]{\textcolor[rgb]{0.56,0.35,0.01}{\textit{#1}}}
\newcommand{\RegionMarkerTok}[1]{#1}
\newcommand{\SpecialCharTok}[1]{\textcolor[rgb]{0.00,0.00,0.00}{#1}}
\newcommand{\SpecialStringTok}[1]{\textcolor[rgb]{0.31,0.60,0.02}{#1}}
\newcommand{\StringTok}[1]{\textcolor[rgb]{0.31,0.60,0.02}{#1}}
\newcommand{\VariableTok}[1]{\textcolor[rgb]{0.00,0.00,0.00}{#1}}
\newcommand{\VerbatimStringTok}[1]{\textcolor[rgb]{0.31,0.60,0.02}{#1}}
\newcommand{\WarningTok}[1]{\textcolor[rgb]{0.56,0.35,0.01}{\textbf{\textit{#1}}}}
\usepackage{longtable,booktabs,array}
\usepackage{calc} % for calculating minipage widths
% Correct order of tables after \paragraph or \subparagraph
\usepackage{etoolbox}
\makeatletter
\patchcmd\longtable{\par}{\if@noskipsec\mbox{}\fi\par}{}{}
\makeatother
% Allow footnotes in longtable head/foot
\IfFileExists{footnotehyper.sty}{\usepackage{footnotehyper}}{\usepackage{footnote}}
\makesavenoteenv{longtable}
\usepackage{graphicx}
\makeatletter
\def\maxwidth{\ifdim\Gin@nat@width>\linewidth\linewidth\else\Gin@nat@width\fi}
\def\maxheight{\ifdim\Gin@nat@height>\textheight\textheight\else\Gin@nat@height\fi}
\makeatother
% Scale images if necessary, so that they will not overflow the page
% margins by default, and it is still possible to overwrite the defaults
% using explicit options in \includegraphics[width, height, ...]{}
\setkeys{Gin}{width=\maxwidth,height=\maxheight,keepaspectratio}
% Set default figure placement to htbp
\makeatletter
\def\fps@figure{htbp}
\makeatother
\setlength{\emergencystretch}{3em} % prevent overfull lines
\providecommand{\tightlist}{%
  \setlength{\itemsep}{0pt}\setlength{\parskip}{0pt}}
\setcounter{secnumdepth}{5}
\usepackage[turkish,shorthands=:!]{babel}
\usepackage{amsmath,amsfonts,amssymb,amsthm}

\makeatletter
\def\thm@space@setup{%
  \thm@preskip=8pt plus 2pt minus 4pt
  \thm@postskip=\thm@preskip
}
\makeatother

\renewcommand{\textfraction}{0.05}
\renewcommand{\topfraction}{0.8}
\renewcommand{\bottomfraction}{0.8}
\renewcommand{\floatpagefraction}{0.75}

\usepackage{booktabs}
\usepackage{longtable}
\usepackage[bf]{caption}

\usepackage{framed,color}

\renewcommand{\href}[2]{#2\footnote{\url{#1}}}


\usepackage{makeidx}
\makeindex


\frontmatter
\ifLuaTeX
  \usepackage{selnolig}  % disable illegal ligatures
\fi
\usepackage[]{natbib}
\bibliographystyle{apalike}

\title{LaTeX Dersleri}
\author{Zafer Acar}
\date{2022-01-17}

\begin{document}
\maketitle



{
\setcounter{tocdepth}{2}
\tableofcontents
}
\listoffigures
\listoftables
\hypertarget{uxf6nsuxf6z}{%
\chapter*{Önsöz}\label{uxf6nsuxf6z}}


\begin{Shaded}
\begin{Highlighting}[]
\BuiltInTok{\textbackslash{}documentclass}\NormalTok{\{}\ExtensionTok{article}\NormalTok{\}}

\KeywordTok{\textbackslash{}begin}\NormalTok{\{}\ExtensionTok{document}\NormalTok{\}}
\SpecialStringTok{$x\^{}2+}\SpecialCharTok{\textbackslash{}sin}\SpecialStringTok{ x $}

\KeywordTok{\textbackslash{}end}\NormalTok{\{}\ExtensionTok{document}\NormalTok{\}}
\end{Highlighting}
\end{Shaded}

\citep{talbot2012}

\mainmatter

\hypertarget{genel}{%
\chapter{Genel}\label{genel}}

\hypertarget{latex-nedir}{%
\section{LaTeX nedir?}\label{latex-nedir}}

LaTeX, TeX'in kullanımını kolaylaştırmak için 1984 yılında Leslie
Lamport tarafından tasarlanmış bir makro pakettir. TeX ise, 1978'den
itibaren \href{https://www-cs-faculty.stanford.edu/~knuth/}{Donald
Knuth} tarafından belgelerin bilgisayarda dizilmesi
için geliştirdiği bir dizgi sistemidir.

Günümüzde Microsoft Word gibi programlar kullanıcının çok güzel belgeler
üretmesini sağlasa da, LaTeX kullanımı, okunaklı bir belge için
vazgeçilmez olan çok sayıda profesyonel olasılık sunmaktadır. Aşağıda
birkaç önemli örnek veriyoruz.

\begin{quote}
Aşağıdaki örnek ve açıklamalar
\href{http://www.rtznet.nl/zink/latex.php?lang=nl}{http://www.rtznet.nl/zink/index.php?lang=nl},
\url{https://nitens.org/w/latex/} ve
\url{https://ctan.org/tex} adreslerinden derlenmiştir.
\end{quote}

\hypertarget{matematiksel-ifadeler}{%
\section{Matematiksel ifadeler}\label{matematiksel-ifadeler}}

LaTeX'le yazılmış bir matematiksel metinle hiç karşılaşmadıysanız,
aşağıdaki örnek (Rogers, \emph{Recursive Functions}), LaTeX'in bu konuda
rakipsiz olduğunu göstermeye yetecektir.

\hypertarget{programlanabilirlik}{%
\section{Programlanabilirlik}\label{programlanabilirlik}}

Bir kelime işlemciyle çalışmak, `gördüğünüzü alırsınız' (WYSIWYG)
ilkesine göre çalışır. Bu avantajlar sunar, ancak kitaplar gibi büyük
metinler üzerinde çalışırken gereksiz şekilde karmaşık ve zaman alıcı
hale gelir. Öte yandan LaTeX, LaTeX'in tamamen formatlanmış bir PDF
belgesine çevirdiği yapı ve düzeni belirtmek için özel kodlarla
oluşturulmuş düz metin kullanır. LaTeX'i kullanmak için çok sayıda komut
ve kod bilmeniz gerekmesine rağmen, bu yaklaşım aynı zamanda kapsamlı
otomasyon imkanı da sunar.

\hypertarget{hizalama-ve-tireleme}{%
\section{Hizalama ve tireleme}\label{hizalama-ve-tireleme}}

Bir metnin genel görünümü ve okunabilirliği, metnin nasıl
hizalandığından ve kesildiğinden büyük ölçüde etkilenir. LaTeX, tüm
paragraf için hizalamayı ve kesmeleri optimize eden son derece gelişmiş
TeX algoritmalarını kullanır. Kelime işlemciler ve diğer programlar,
satır başına çalıştıkları için oldukça yetersiz kalırlar. Bu, diğer
şeylerin yanı sıra düzensiz aralıklara ve birçok kısa çizgiye sebep
olur. Sonuçları görmeniz için Microsoft Word 2008 (Mac), Adobe InDesign
CS4 ve LaTeX'le dizilmiş bir metni ekliyoruz:

Sonuç, LaTeX'in diğer programların her ikisinden de üstün olduğunu
açıkça gösterir: iki kat daha az tireleme kullanır ve yine de sözcük
aralığındaki varyasyon, Word veya InDesign'dan belirgin şekilde daha
azdır. LaTeX'te çok büyük sözcük aralığı içeren satırlar oluşmaz.

\hypertarget{bitiux15fik-harfler}{%
\section{Bitişik harfler}\label{bitiux15fik-harfler}}

Birkaç harf yan yana geldiklerinde çarpışır. İyi bilinen örnekler, f'nin
i'nin ucuna veya l'nin tepesine çarptığı `fl' ve `fi'
kombinasyonlarıdır. Bu nedenle birçok yazı tipi bu birleşimler için özel
karakterler içerir. LaTeX'in kendisi, bir yazı tipinin hangi bitişik
harfleri desteklediğini arar ve bunları uygun şekilde kullanır. Kelime
işlemciler ve diğer programlar bunu yapmaz, bu nedenle kullanıcının bu
kombinasyonları değiştirmek için kendisinin araması gerekir.

\hypertarget{geruxe7ek-kuxfcuxe7uxfck-buxfcyuxfck-harfler-ve-baux15flux131klar}{%
\section{Gerçek küçük büyük harfler ve başlıklar}\label{geruxe7ek-kuxfcuxe7uxfck-buxfcyuxfck-harfler-ve-baux15flux131klar}}

Çoğu kelime işlemcisi, büyük harflerin boyutunu ayarlayarak sahte küçük
büyük harfler oluşturur. Profesyonel yazı tipleri, herhangi bir ciddi
dizgi sisteminin uygun bağlamda kullanabilmesi gereken küçük büyük
harfler ve tam boyutlu büyük harfler için farklı glif kümeleri içerir.
Özellikle, gerçek küçük büyük harfler, büyük harflerin yeniden
boyutlandırılmış versiyonlarından daha fazlasıdır: nispeten daha ağır
bir kontura sahiptirler ve aynı yazı tipinin küçük harfli
karakterleriyle görsel olarak uyumlu olacak şekilde tasarlanmıştır. Bazı
OpenType yazı tiplerinin özel ``başlık'' alternatifleri vardır, bunlar
büyük boyutlarda ve daha açık kontura sahip tümüyle büyük harfli tipler
için tasarlanmıştır.

\hypertarget{uxe7ux131kux131ntux131-karakter-aralux131ux11fux131}{%
\section{Çıkıntı (karakter aralığı)}\label{uxe7ux131kux131ntux131-karakter-aralux131ux11fux131}}

Çıkıntı, şekiller buna neden olduğunda harflerin birbirine daha yakın
veya daha fazla aralıklı olmasıdır. Bu, okunabilirliği artıran çok daha
dengeli bir aralık üretir. T, V, W ve A gibi harfler bundan yararlanır.
Profesyonel yazı tipi setleri, tüm harf kombinasyonlarının doğru
çıkıntısını gösteren kapsamlı bir çıkıntı tablosu içerir. Word, diğer
çoğu kelime işlemci programı gibi, bu tabloları yok sayarken, LaTeX
bunları varsayılan olarak kullanır.

\hypertarget{nadir-tarihi-ve-bitiux15fik-harfler}{%
\section{Nadir, tarihi ve bitişik harfler}\label{nadir-tarihi-ve-bitiux15fik-harfler}}

LaTeX ile profesyonel yazı tiplerinin ekstra olanaklarını da otomatik
olarak kullanabilirsiniz. Uzun s gibi tarihsel harf formlarının yanı
sıra ekstra bitişik harfler (örneğin `st' kombinasyonu) LaTeX'te
otomatik olarak kullanılabilir. Bazen manuel olarak eklemek mümkün olsa
da, Word bu seçeneklere erişim sağlamaz.

LaTeX, genelde WYSIWYG editörleriyle karşılaştırılır. WYSIWYG, Microsoft
Word, Libreoffice Writer gibi kelime işlemcilere ya da Adobe Indesign
gibi programlara verilen genel bir isimdir. Hepsinin ortak özelliği,
girdi ile çıktının aynı anda ve birlikte görünmesidir.

LaTeX'de ise işler farklı yürür. Girdi ve çıktı ekranı farklıdır ve
çıktıyı görmek için girdinin derleme işleminden geçmesi gerekir. Ayrıca
birçok şey için WYSIWYG editörlerinde olmayan yapılar vardır. Şimdi, bu
yapıların ne oldukları ve ne işe yaradıklarını açıklayalım.

\hypertarget{komutlar}{%
\section{Komutlar}\label{komutlar}}

LaTeX komutları bir geribölü (\texttt{\textbackslash{}}) işaretiyle başlar ve ya sadece
harflerden ya da bir tane harf olmayan karakterden oluşurlar. Komut
yazıldıktan sonra ya boşluk, ya bir sayı ya da harf olmayan bir karakter
gelebilir.

Çoğu komut, zorunlu değişken alır. Bu zorunlu değişken komut adından
sonra çengelli parantezler içine yazılır. Zorunlu değişken alan
komutlar, zorunlu olmayan (isteğe bağlı) değişkenler de alabilir, bunlar
da komut adından sonra gelen köşeli parantezler içine yazılırlar. Eğer
değişkenler birden fazlaysa aralarına virgül koyularak ayrılır.

\begin{Shaded}
\begin{Highlighting}[]
\NormalTok{\textbackslash{}:}
\NormalTok{\textbackslash{}LaTeX}
\NormalTok{\textbackslash{}item[...]}
\NormalTok{\textbackslash{}emph\{...\}}
\NormalTok{\textbackslash{}documentclass[...]\{...\}}
\NormalTok{\textbackslash{}subfloat[...][...]\{...\}}
\NormalTok{\textbackslash{}raisebox\{...\}[...][...]\{...\}}
\NormalTok{\textbackslash{}multicolumn\{...\}\{...\}\{...\}}
\NormalTok{\{\textbackslash{}bfseries ...\}}
\end{Highlighting}
\end{Shaded}

Fikir vermesi açısından yukarıda dokuz adet komut örneği verilmiştir.
Birinci komut bir tane harf olmayan karakterden oluşan bir komuttur.
İkincisi, değişkeni olmayan bir komuttur. Bazı harflerin büyük
bazılarınınsa küçük olması komutların büyük-küçük harfe duyarlı olduğunu
gösterir. Dokuzuncu komut ise bildirim şeklinde verilmiştir.

\hypertarget{paketler}{%
\section{Paketler}\label{paketler}}

LaTeX'de bazı özelliklerin (renkli yazmak, şekil eklemek vb.)
kullanılabilmesi için kaynak dosyaya bazı paketlerin eklenmesi gerekir.
Bu, \texttt{\textbackslash{}usepackage} komutuyla yapılır. Bu komutun zorunlu değişkenine
paket adı, zorunlu olmayan kısmına ise paket seçenekleri yazılır:

\begin{Shaded}
\begin{Highlighting}[]
\NormalTok{\textbackslash{}usepackage[\textless{}seçenekler\textgreater{}]\{\textless{}paket adı\textgreater{}\}}
\end{Highlighting}
\end{Shaded}

Bu komutla paketin kaynak dosyaya eklenmesi TeX dağıtımıyla sisteminize
kurulmuş olan paketin belgeye çağrılarak işe koşulması demektir.

\hypertarget{ortamlar}{%
\section{Ortamlar}\label{ortamlar}}

LaTeX'de ortamlar önemli bir yer tutar. Örneğin \texttt{document} bir ortamdır.
Ortamları birden fazla ögeye uygulanan komutlar olarak düşünebiliriz.

Bir ortam \texttt{\textbackslash{}begin} komutuyla başlayıp \texttt{\textbackslash{}end} komutuyla biter. Her iki
komutun zorunlu değişkeni ortamın adıdır:

\begin{Shaded}
\begin{Highlighting}[]
\NormalTok{\textbackslash{}begin\{\textless{}ortam adı\textgreater{}\}}
\NormalTok{ ...}
\NormalTok{\textbackslash{}end\{\textless{}ortam adı\textgreater{}\}}
\end{Highlighting}
\end{Shaded}

\hypertarget{gruplar}{%
\section{Gruplar}\label{gruplar}}

Gruplar, ortam benzeri yapılardır. Grup \texttt{\textbackslash{}begingroup} komutuyla başlar
ve \texttt{\textbackslash{}endgroup} komutuyla biter. Grubun içinde kullanılan bir bildirim
sadece gruba uygulanır.

\hypertarget{boux15fluklar}{%
\section{Boşluklar}\label{boux15fluklar}}

LaTeX'de belgenizin metnini oluştururken ister klavyedeki Space, ister
Tab tuşu ile boşluk bırakın, bu boşluklar LaTeX tarafından bir karakter
boşluk olarak algılanır. Arka arkaya çok sayıda boşluk bırakılsa da
LaTeX bunu tek bir boşluk olarak algılar.

Bütün bir satırın boş bırakılması LaTeX tarafından paragraf başı olarak
algılanır. Arka arkaya boş bırakılan çok sayıda boş satır LaTeX
tarafından tek bir boş satır yani paragraf başı olarak algılanır.

\begin{Shaded}
\begin{Highlighting}[]
\NormalTok{ İster bir boşluk, isterseniz de çok         sayıda boşluk bırakın. }
\NormalTok{İkisi de bir boşluk gibi işlem görür. }

\NormalTok{Boş bir satır yeni paragraf demektir, burada olduğu gibi.}
\end{Highlighting}
\end{Shaded}

\href{https://www.texdizgi.com/wp-content/uploads/2020/04/LaTeX_Sunu.pdf}{LaTeX\_Sunu\\
}

Komutlardan sonra gelen boşlukları LaTeX dikkate almaz. Komuttan sonra
gerçekten bir boşluk bırakmak için, ya \texttt{\{\}} ve ardından boşluk girilir
ya da komut adından sonra özel bir boşluk komutu kullanılır.

\begin{Shaded}
\begin{Highlighting}[]
\NormalTok{\textbackslash{}LaTeX  boşluk yok.\textbackslash{}\textbackslash{}}
\NormalTok{\textbackslash{}LaTeX\{\} boşluk var.\textbackslash{}\textbackslash{}}
\NormalTok{\textbackslash{}LaTeX\textbackslash{} boşluk komutuyla  boşluk.}
\end{Highlighting}
\end{Shaded}

\href{https://www.texdizgi.com/wp-content/uploads/2020/04/LaTeX_Sunu-1.pdf}{LaTeX\_Sunu-1\\
}

\hypertarget{uxf6zel-amauxe7lux131-karakterler}{%
\section{Özel amaçlı karakterler}\label{uxf6zel-amauxe7lux131-karakterler}}

Aşağıdaki karakterlerin herbiri LaTeX'de özel bir amaç için kullanılır.
Dolayısıyla bu karakterleri doğrudan kullanmak istenmeyen sonuçlara yol
açabilir.

\begin{Shaded}
\begin{Highlighting}[]
\NormalTok{\#  $  \%   \&   \{   \}   \textasciitilde{}  \^{}  \_ \textbackslash{}}
\end{Highlighting}
\end{Shaded}

Bu karakterleri çıktıda elde etmek isterseniz, sondaki hariç, başına bir
geribölü koymanız gerekir. Sondaki için, yani bir geribölü sembolü elde
etmek içinse \texttt{\textbackslash{}textbackslash} komutunu kullanabilirsiniz. Eğer \texttt{\textbackslash{}\textbackslash{}}
komutunu verirseniz yeni bir satır başlatmış olursunuz.

Bu karakterlerden örneğin yüzde (\texttt{\%}) karakteri kaynak dosyanızda yorum
ya da açıklama yazmaya yarar. Bu sembolden sonra yazılanları LaTeX
dikkate almaz ve çıktıda görünmez.

\begin{Shaded}
\begin{Highlighting}[]
\NormalTok{\textbackslash{}\%50 zam geldi. }
\NormalTok{\% Buraya yorum yazabilirsiniz.}
\end{Highlighting}
\end{Shaded}

\href{https://www.texdizgi.com/wp-content/uploads/2020/04/LaTeX_Sunu-2.pdf}{LaTeX\_Sunu-2\\
}

:::

Diğer karakterlerden örneğin (\texttt{\$}) nin matematik kipini açma ve
kapatmaya yaradığından
\href{https://www.texdizgi.com/latexde-matematiksel-ifadeleri-dizmek-1-giris/}{bahsetmiştik}. (\texttt{\&}) karekteri tablo ve benzeri yapılarda dikey
hizalama yapmak için veya sütun ayracı olarak kullanılır. Çengelli
parantezlerden zaten yeterince bahsettik. (\texttt{\#}) karakteri yeni komutlar
tanımlamakta kullanılır. Tilda (\texttt{\textasciitilde{}}) ise genişlemeyen bir boşluk
yaratmak için kullanılır. (\texttt{\^{}}) ve (\texttt{\_}) karakterleri de matematikte üst
ve alt indis yazmak için kullanılır. Her birinin kullanımlarından yeri
geldiğinde tekrar bahsedeceğiz.

\backmatter

  \bibliography{book.bib,packages.bib}

\printindex

\end{document}
