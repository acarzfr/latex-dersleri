% Options for packages loaded elsewhere
\PassOptionsToPackage{unicode}{hyperref}
\PassOptionsToPackage{hyphens}{url}
%
\documentclass[
  10pt,
]{scrbook}
\usepackage{amsmath,amssymb}
\usepackage{lmodern}
\usepackage{iftex}
\ifPDFTeX
  \usepackage[T1]{fontenc}
  \usepackage[utf8]{inputenc}
  \usepackage{textcomp} % provide euro and other symbols
\else % if luatex or xetex
  \usepackage{unicode-math}
  \defaultfontfeatures{Scale=MatchLowercase}
  \defaultfontfeatures[\rmfamily]{Ligatures=TeX,Scale=1}
  \setmainfont[]{Source Serif Pro}
  \setsansfont[]{Source Sans Pro}
  \setmonofont[Scale=0.8]{Source Code Pro}
\fi
% Use upquote if available, for straight quotes in verbatim environments
\IfFileExists{upquote.sty}{\usepackage{upquote}}{}
\IfFileExists{microtype.sty}{% use microtype if available
  \usepackage[]{microtype}
  \UseMicrotypeSet[protrusion]{basicmath} % disable protrusion for tt fonts
}{}
\makeatletter
\@ifundefined{KOMAClassName}{% if non-KOMA class
  \IfFileExists{parskip.sty}{%
    \usepackage{parskip}
  }{% else
    \setlength{\parindent}{0pt}
    \setlength{\parskip}{6pt plus 2pt minus 1pt}}
}{% if KOMA class
  \KOMAoptions{parskip=half}}
\makeatother
\usepackage{xcolor}
\IfFileExists{xurl.sty}{\usepackage{xurl}}{} % add URL line breaks if available
\IfFileExists{bookmark.sty}{\usepackage{bookmark}}{\usepackage{hyperref}}
\hypersetup{
  pdftitle={LaTeX Dersleri},
  pdfauthor={Zafer Acar},
  hidelinks,
  pdfcreator={LaTeX via pandoc}}
\urlstyle{same} % disable monospaced font for URLs
\usepackage{color}
\usepackage{fancyvrb}
\newcommand{\VerbBar}{|}
\newcommand{\VERB}{\Verb[commandchars=\\\{\}]}
\DefineVerbatimEnvironment{Highlighting}{Verbatim}{commandchars=\\\{\}}
% Add ',fontsize=\small' for more characters per line
\usepackage{framed}
\definecolor{shadecolor}{RGB}{248,248,248}
\newenvironment{Shaded}{\begin{snugshade}}{\end{snugshade}}
\newcommand{\AlertTok}[1]{\textcolor[rgb]{0.94,0.16,0.16}{#1}}
\newcommand{\AnnotationTok}[1]{\textcolor[rgb]{0.56,0.35,0.01}{\textbf{\textit{#1}}}}
\newcommand{\AttributeTok}[1]{\textcolor[rgb]{0.77,0.63,0.00}{#1}}
\newcommand{\BaseNTok}[1]{\textcolor[rgb]{0.00,0.00,0.81}{#1}}
\newcommand{\BuiltInTok}[1]{#1}
\newcommand{\CharTok}[1]{\textcolor[rgb]{0.31,0.60,0.02}{#1}}
\newcommand{\CommentTok}[1]{\textcolor[rgb]{0.56,0.35,0.01}{\textit{#1}}}
\newcommand{\CommentVarTok}[1]{\textcolor[rgb]{0.56,0.35,0.01}{\textbf{\textit{#1}}}}
\newcommand{\ConstantTok}[1]{\textcolor[rgb]{0.00,0.00,0.00}{#1}}
\newcommand{\ControlFlowTok}[1]{\textcolor[rgb]{0.13,0.29,0.53}{\textbf{#1}}}
\newcommand{\DataTypeTok}[1]{\textcolor[rgb]{0.13,0.29,0.53}{#1}}
\newcommand{\DecValTok}[1]{\textcolor[rgb]{0.00,0.00,0.81}{#1}}
\newcommand{\DocumentationTok}[1]{\textcolor[rgb]{0.56,0.35,0.01}{\textbf{\textit{#1}}}}
\newcommand{\ErrorTok}[1]{\textcolor[rgb]{0.64,0.00,0.00}{\textbf{#1}}}
\newcommand{\ExtensionTok}[1]{#1}
\newcommand{\FloatTok}[1]{\textcolor[rgb]{0.00,0.00,0.81}{#1}}
\newcommand{\FunctionTok}[1]{\textcolor[rgb]{0.00,0.00,0.00}{#1}}
\newcommand{\ImportTok}[1]{#1}
\newcommand{\InformationTok}[1]{\textcolor[rgb]{0.56,0.35,0.01}{\textbf{\textit{#1}}}}
\newcommand{\KeywordTok}[1]{\textcolor[rgb]{0.13,0.29,0.53}{\textbf{#1}}}
\newcommand{\NormalTok}[1]{#1}
\newcommand{\OperatorTok}[1]{\textcolor[rgb]{0.81,0.36,0.00}{\textbf{#1}}}
\newcommand{\OtherTok}[1]{\textcolor[rgb]{0.56,0.35,0.01}{#1}}
\newcommand{\PreprocessorTok}[1]{\textcolor[rgb]{0.56,0.35,0.01}{\textit{#1}}}
\newcommand{\RegionMarkerTok}[1]{#1}
\newcommand{\SpecialCharTok}[1]{\textcolor[rgb]{0.00,0.00,0.00}{#1}}
\newcommand{\SpecialStringTok}[1]{\textcolor[rgb]{0.31,0.60,0.02}{#1}}
\newcommand{\StringTok}[1]{\textcolor[rgb]{0.31,0.60,0.02}{#1}}
\newcommand{\VariableTok}[1]{\textcolor[rgb]{0.00,0.00,0.00}{#1}}
\newcommand{\VerbatimStringTok}[1]{\textcolor[rgb]{0.31,0.60,0.02}{#1}}
\newcommand{\WarningTok}[1]{\textcolor[rgb]{0.56,0.35,0.01}{\textbf{\textit{#1}}}}
\usepackage{longtable,booktabs,array}
\usepackage{calc} % for calculating minipage widths
% Correct order of tables after \paragraph or \subparagraph
\usepackage{etoolbox}
\makeatletter
\patchcmd\longtable{\par}{\if@noskipsec\mbox{}\fi\par}{}{}
\makeatother
% Allow footnotes in longtable head/foot
\IfFileExists{footnotehyper.sty}{\usepackage{footnotehyper}}{\usepackage{footnote}}
\makesavenoteenv{longtable}
\usepackage{graphicx}
\makeatletter
\def\maxwidth{\ifdim\Gin@nat@width>\linewidth\linewidth\else\Gin@nat@width\fi}
\def\maxheight{\ifdim\Gin@nat@height>\textheight\textheight\else\Gin@nat@height\fi}
\makeatother
% Scale images if necessary, so that they will not overflow the page
% margins by default, and it is still possible to overwrite the defaults
% using explicit options in \includegraphics[width, height, ...]{}
\setkeys{Gin}{width=\maxwidth,height=\maxheight,keepaspectratio}
% Set default figure placement to htbp
\makeatletter
\def\fps@figure{htbp}
\makeatother
\setlength{\emergencystretch}{3em} % prevent overfull lines
\providecommand{\tightlist}{%
  \setlength{\itemsep}{0pt}\setlength{\parskip}{0pt}}
\setcounter{secnumdepth}{5}
\usepackage[turkish,shorthands=:!]{babel}
\usepackage{amsmath,amsfonts,amssymb,amsthm}

\makeatletter
\def\thm@space@setup{%
  \thm@preskip=8pt plus 2pt minus 4pt
  \thm@postskip=\thm@preskip
}
\makeatother

\renewcommand{\textfraction}{0.05}
\renewcommand{\topfraction}{0.8}
\renewcommand{\bottomfraction}{0.8}
\renewcommand{\floatpagefraction}{0.75}

\usepackage{booktabs}
\usepackage{longtable}
\usepackage[bf]{caption}

\usepackage{framed,color}

\renewcommand{\href}[2]{#2\footnote{\url{#1}}}


\usepackage{makeidx}
\makeindex


\frontmatter
\ifLuaTeX
  \usepackage{selnolig}  % disable illegal ligatures
\fi
\usepackage[]{natbib}
\bibliographystyle{apalike}

\title{LaTeX Dersleri}
\author{Zafer Acar}
\date{2022-01-21}

\begin{document}
\maketitle



{
\setcounter{tocdepth}{2}
\tableofcontents
}
\listoffigures
\listoftables
\hypertarget{uxf6nsuxf6z}{%
\chapter*{Önsöz}\label{uxf6nsuxf6z}}


\mainmatter

\hypertarget{genel}{%
\chapter{Genel}\label{genel}}

Bu bölümde LaTeX kullanımıyla ilgili genel bilgilerden bahsedeceğiz.

\hypertarget{latex-nedir}{%
\section{LaTeX Nedir?}\label{latex-nedir}}

Önce TeX'le başlayalım. TeX, 1978'den
itibaren \href{https://www-cs-faculty.stanford.edu/~knuth/}{Donald Knuth} tarafından belgelerin bilgisayarda dizilmesi
için geliştirdiği bir dizgi sistemidir.
LaTeX ise TeX'in kullanımını kolaylaştırmak için 1984 yılında \href{http://www.lamport.org/}{Leslie
Lamport} tarafından tasarlanmış bir makro pakettir.

LaTeX, genelde WYSIWYG\footnote{WYSIWYG, İngilizce'de ``What You See Is What You Get'' teriminin baş harflerinden oluşan bir bilgisayar terimidir. Türkçesi \emph{Ne Görüyorsan Onu Alırsın} demek olup ekranda görülene çok benzer bir çıktı alınacağı ortamları tanımlar.} editörleriyle karşılaştırılır. WYSIWYG, Microsoft
Word, Libreoffice Writer gibi kelime işlemcilere ya da Adobe Indesign
gibi programlara verilen genel bir isimdir. Hepsinin ortak özelliği,
girdi ile çıktının aynı anda ve birlikte görünmesidir.

Bir metnin genel görünümü ve okunabilirliği, metnin nasıl
hizalandığından ve kesildiğinden büyük ölçüde etkilenir. LaTeX, tüm
paragraf için hizalamayı ve kesmeleri optimize eden son derece gelişmiş
TeX algoritmalarını kullanır. Kelime işlemciler ve diğer programlar,
satır başına çalıştıkları için oldukça yetersiz kalırlar. Bu, diğer
şeylerin yanı sıra düzensiz aralıklara ve birçok kısa çizgiye sebep
olur. Sonuçları görmeniz için Microsoft Word 2008 (Mac), Adobe InDesign
CS4 ve LaTeX'le dizilmiş bir metni \href{http://www.rtznet.nl/zink/comparison.pdf}{şuradan} inceleyebilirsiniz.

Sonuç, LaTeX'in diğer programların her ikisinden de üstün olduğunu
açıkça gösterir: iki kat daha az tireleme kullanır ve yine de sözcük
aralığındaki varyasyon, Word veya InDesign'dan belirgin şekilde daha
azdır. LaTeX'te çok büyük sözcük aralığı içeren satırlar oluşmaz.

LaTeX'de girdi ve çıktı ekranı farklıdır ve
çıktıyı görmek için girdinin derleme işleminden geçmesi gerekir. Ayrıca
birçok şey için WYSIWYG editörlerinde olmayan yapılar vardır. Şimdi, bu
yapıların ne oldukları ve ne işe yaradıklarını açıklayalım.

\hypertarget{uxf6nemli-yapux131lar}{%
\section{Önemli Yapılar}\label{uxf6nemli-yapux131lar}}

\hypertarget{komutlar}{%
\subsection{Komutlar}\label{komutlar}}

LaTeX komutları bir geribölü (\texttt{\textbackslash{}}) işaretiyle başlar ve ya sadece
harflerden ya da bir tane harf olmayan karakterden oluşurlar. Komut
yazıldıktan sonra ya boşluk, ya bir sayı ya da harf olmayan bir karakter
gelebilir.

Çoğu komut, zorunlu değişken alır. Bu zorunlu değişken komut adından
sonra çengelli parantezler içine yazılır. Zorunlu değişken alan
komutlar, zorunlu olmayan (isteğe bağlı) değişkenler de alabilir, bunlar
da komut adından sonra gelen köşeli parantezler içine yazılırlar. Eğer
değişkenler birden fazlaysa aralarına virgül koyularak ayrılır.

\begin{Shaded}
\begin{Highlighting}[numbers=left,,]
\NormalTok{\textbackslash{}}\SpecialCharTok{:}
\NormalTok{\textbackslash{}LaTeX}
\NormalTok{\textbackslash{}item[...]}
\NormalTok{\textbackslash{}emph\{...\}}
\NormalTok{\textbackslash{}documentclass[...]\{...\}}
\NormalTok{\textbackslash{}subfloat[...][...]\{...\}}
\NormalTok{\textbackslash{}raisebox\{...\}[...][...]\{...\}}
\NormalTok{\textbackslash{}multicolumn\{...\}\{...\}\{...\}}
\NormalTok{\{\textbackslash{}bfseries ...\}}
\end{Highlighting}
\end{Shaded}

Fikir vermesi açısından yukarıda dokuz adet komut örneği verilmiştir.
Birinci komut bir tane harf olmayan karakterden oluşan bir komuttur.
İkincisi, değişkeni olmayan bir komuttur. Bazı harflerin büyük
bazılarınınsa küçük olması komutların büyük-küçük harfe duyarlı olduğunu
gösterir. Dokuzuncu komut ise bildirim şeklinde verilmiştir.

\hypertarget{paketler}{%
\subsection{Paketler}\label{paketler}}

LaTeX'de bazı özelliklerin (renkli yazmak, şekil eklemek vb.)
kullanılabilmesi için kaynak dosyaya bazı paketlerin eklenmesi gerekir.
Bu, \texttt{\textbackslash{}usepackage} komutuyla yapılır. Bu komutun zorunlu değişkenine
paket adı, zorunlu olmayan kısmına ise paket seçenekleri yazılır:

\begin{Shaded}
\begin{Highlighting}[]
\NormalTok{\textbackslash{}usepackage[}\SpecialCharTok{\textless{}}\NormalTok{seçenekler}\SpecialCharTok{\textgreater{}}\NormalTok{]\{}\SpecialCharTok{\textless{}}\NormalTok{paket adı}\SpecialCharTok{\textgreater{}}\NormalTok{\}}
\end{Highlighting}
\end{Shaded}

Bu komutla paketin kaynak dosyaya eklenmesi TeX dağıtımıyla sisteminize
kurulmuş olan paketin belgeye çağrılarak işe koşulması demektir.

\hypertarget{ortamlar}{%
\subsection{Ortamlar}\label{ortamlar}}

LaTeX'de ortamlar önemli bir yer tutar. Örneğin \texttt{document} bir ortamdır.
Ortamları birden fazla ögeye uygulanan komutlar olarak düşünebiliriz.

Bir ortam \texttt{\textbackslash{}begin} komutuyla başlayıp \texttt{\textbackslash{}end} komutuyla biter. Her iki
komutun zorunlu değişkeni ortamın adıdır:

\begin{Shaded}
\begin{Highlighting}[]
\NormalTok{\textbackslash{}begin\{}\SpecialCharTok{\textless{}}\NormalTok{ortam adı}\SpecialCharTok{\textgreater{}}\NormalTok{\}}
\NormalTok{ ...}
\NormalTok{\textbackslash{}end\{}\SpecialCharTok{\textless{}}\NormalTok{ortam adı}\SpecialCharTok{\textgreater{}}\NormalTok{\}}
\end{Highlighting}
\end{Shaded}

\hypertarget{gruplar}{%
\subsection{Gruplar}\label{gruplar}}

Gruplar, ortam benzeri yapılardır. Grup \texttt{\textbackslash{}begingroup} komutuyla başlar
ve \texttt{\textbackslash{}endgroup} komutuyla biter. Grubun içinde kullanılan bir bildirim
sadece gruba uygulanır.

\hypertarget{boux15fluklar}{%
\subsection{Boşluklar}\label{boux15fluklar}}

LaTeX'de belgenizin metnini oluştururken ister klavyedeki Space, ister
Tab tuşu ile boşluk bırakın, bu boşluklar LaTeX tarafından bir karakter
boşluk olarak algılanır. Arka arkaya çok sayıda boşluk bırakılsa da
LaTeX bunu tek bir boşluk olarak algılar.

Bütün bir satırın boş bırakılması LaTeX tarafından paragraf başı olarak
algılanır. Arka arkaya boş bırakılan çok sayıda boş satır LaTeX
tarafından tek bir boş satır yani paragraf başı olarak algılanır.

\begin{Shaded}
\begin{Highlighting}[]
\NormalTok{ İster bir boşluk, isterseniz de çok         sayıda boşluk bırakın. }
\NormalTok{İkisi de bir boşluk gibi işlem görür. }

\NormalTok{Boş bir satır yeni paragraf demektir, burada olduğu gibi.}
\end{Highlighting}
\end{Shaded}

\href{https://github.com/acarzfr/latex-dersleri/blob/main/examples/ex1.pdf}{Çıktı}

Komutlardan sonra gelen boşlukları LaTeX dikkate almaz. Komuttan sonra
gerçekten bir boşluk bırakmak için, ya \texttt{\{\}} ve ardından boşluk girilir
ya da komut adından sonra özel bir boşluk komutu kullanılır.

\begin{Shaded}
\begin{Highlighting}[]
\NormalTok{\textbackslash{}LaTeX  boşluk yok.\textbackslash{}\textbackslash{}}
\NormalTok{\textbackslash{}LaTeX\{\} boşluk var.\textbackslash{}\textbackslash{}}
\NormalTok{\textbackslash{}LaTeX\textbackslash{} boşluk komutuyla  boşluk.}
\end{Highlighting}
\end{Shaded}

\href{https://github.com/acarzfr/latex-dersleri/blob/main/examples/ex2.pdf}{Çıktı}

\hypertarget{uxf6zel-amauxe7lux131-karakterler}{%
\subsection{Özel amaçlı karakterler}\label{uxf6zel-amauxe7lux131-karakterler}}

Aşağıdaki karakterlerin herbiri LaTeX'de özel bir amaç için kullanılır.
Dolayısıyla bu karakterleri doğrudan kullanmak istenmeyen sonuçlara yol
açabilir.

\begin{Shaded}
\begin{Highlighting}[]
\CommentTok{\#  $  \%   \&   \{   \}   \textasciitilde{}  \^{}  \_ \textbackslash{}}
\end{Highlighting}
\end{Shaded}

Bu karakterleri çıktıda elde etmek isterseniz, sondaki hariç, başına bir
geribölü koymanız gerekir. Sondaki için, yani bir geribölü sembolü elde
etmek içinse \texttt{\textbackslash{}textbackslash} komutunu kullanabilirsiniz. Eğer \texttt{\textbackslash{}\textbackslash{}}
komutunu verirseniz yeni bir satır başlatmış olursunuz.

Bu karakterlerden örneğin yüzde (\texttt{\%}) karakteri kaynak dosyanızda yorum
ya da açıklama yazmaya yarar. Bu sembolden sonra yazılanları LaTeX
dikkate almaz ve çıktıda görünmez.

Diğer karakterlerden örneğin (\texttt{\$}) nin matematik kipini açma ve
kapatmaya yarar. (\texttt{\&}) karekteri tablo ve benzeri yapılarda dikey
hizalama yapmak için veya sütun ayracı olarak kullanılır. Çengelli
parantezlerden zaten yeterince bahsettik. (\texttt{\#}) karakteri yeni komutlar
tanımlamakta kullanılır. Tilda (\texttt{\textasciitilde{}}) ise genişlemeyen bir boşluk
yaratmak için kullanılır. (\texttt{\^{}}) ve (\texttt{\_}) karakterleri de matematikte üst
ve alt indis yazmak için kullanılır. Her birinin kullanımlarından yeri
geldiğinde tekrar bahsedeceğiz.

\hypertarget{kurulum}{%
\section{Kurulum}\label{kurulum}}

LaTeX'i kurmak için ilk olarak bir TeX dağıtımı edinmeniz gerekir.
Dağıtımlar, dizgi sistemini ve LaTeX'de belge oluşturabilmek için
gereken paketleri içerir.

İkinci ihtiyaç duyacağınız şey bir LaTeX editörüdür. Edindiğiniz TeX
dağıtımları genelde bir LaTeX editörüyle birlikte gelir. Tabi editör
kişisel bir tercihtir ve bir LaTeX editörü yerine basit bir metin
editörü kullanabilirsiniz. Ancak farklı işletim sistemleri için birçok
iyi LaTeX editörü vardır ve bunların kod vurgulama, otomatik tamamlama,
otomatik belge oluşturma gibi LaTeX'e özgü işlevleri vardır. Dolayısıyla
LaTeX'de yeniyseniz bir editör kullanmanızı tavsiye ederiz.

\hypertarget{gnulinux}{%
\subsection{GNU/Linux}\label{gnulinux}}

Linux sistemlere \href{https://miktex.org/download}{MiKTeX} ya da \href{http://www.tug.org/texlive/}{TeX
Live} kurulabilir. MiKTeX'in indirme sayfasında Ubuntu, Mint,
Debian, Fedora, CentOS ve openSUSE gibi Linux dağıtımlarında nasıl
kurulacağı anlatılmıştır. TeX Live ise tüm popüler Linux dağıtımlarının
depolarında mevcut olup, paket yöneticisi ya da komut satırı yardımıyla
kurulabilir. Örneğin Ubuntu, Debian, Mint, Pardus gibi \texttt{.deb} uzantılı
paketlerin kullanıldığı dağıtımlarda

\begin{Shaded}
\begin{Highlighting}[]
\NormalTok{sudo apt}\SpecialCharTok{{-}}\NormalTok{get install texlive}\SpecialCharTok{{-}}\NormalTok{base}
\end{Highlighting}
\end{Shaded}

komutuyla temel kurulum,

\begin{Shaded}
\begin{Highlighting}[]
\NormalTok{sudo apt}\SpecialCharTok{{-}}\NormalTok{get install texlive}\SpecialCharTok{{-}}\NormalTok{full}
\end{Highlighting}
\end{Shaded}

komutuyla da tam kurulum yapılır.

\hypertarget{mac-os}{%
\subsection{Mac OS}\label{mac-os}}

Mac OS kullanıcıları için iki seçenek mevcuttur:
\href{https://miktex.org/download}{MiKTeX} ya da
\href{http://www.tug.org/mactex/}{MacTeX}. MiKTeX kurulumu için \texttt{.dmg} uzantılı, MacTeX içinse
\texttt{.pkg} uzantılı dosya indirilir ve standart kurulum yapılır.

\hypertarget{windows}{%
\subsection{Windows}\label{windows}}

Windows için aşağıdaki dağıtımlardan birini kurabilirsiniz.

\begin{itemize}
\tightlist
\item
  \href{https://miktex.org/download}{MiKTeX}
\item
  \href{http://www.tug.org/texlive/}{TeX Live}
\item
  \href{https://tug.org/protext/}{proTeXt}
\end{itemize}

MiKTeX veya TeX Live dağıtımını kurarsanız sisteminize
\href{https://www.tug.org/texworks/}{TeXworks} editörü de kurulur. proTeXt dağıtımı MiKTeX tabanlı bir
dağıtım olup, tüm paketleri içerir ve beraberinde
\href{https://texstudio.org/}{TeXstudio} editörüyle gelir.

\hypertarget{latex-edituxf6rleri}{%
\subsection{LaTeX Editörleri}\label{latex-edituxf6rleri}}

Hangi editörü kullanacağınıza birkaç deneme yaptıktan sonra karar
verebilirsiniz. \href{https://beebom.com/best-latex-editors/}{Burada} en çok
beğenilen editörler listelenmiş.

Her LaTeX editöründe olan özelliklerin (otomatik kod tamamlama vb.) yanı
sıra kullanıcı dostu arayüzü, yüzde yüze yakın Türkçe desteği, ücretsiz
oluşu ve her üç sistemde de çalışabilmesinden dolayı
\href{https://texstudio.org/}{TeXstudio}'yu tavsiye ediyoruz. Karar sizin.

\hypertarget{uxe7evrimiuxe7i-edituxf6rler}{%
\subsection{Çevrimiçi Editörler}\label{uxe7evrimiuxe7i-edituxf6rler}}

LaTeX'i hiçbir kurulum yapmadan çevrimiçi de kullanabilirsiniz. Aşağıda
üç tanesi listelenmiştir.

\begin{itemize}
\tightlist
\item
  \href{https://www.overleaf.com/}{Overleaf}
\item
  \href{https://papeeria.com/}{Papeeria}
\item
  \href{https://latexbase.com/}{LaTeX Base}
\end{itemize}

En popüler olanı Overleaf olup, sayfasında beğenebileceğiniz binlerce
\href{https://www.overleaf.com/latex/templates}{şablon} ve LaTeX kullanımına yönelik
\href{https://www.overleaf.com/learn}{anlatımlar} bulunur.

\hypertarget{tipik}{%
\section{Tipik Bir Belge Yazımı}\label{tipik}}

LaTeX'in varsayılan dosya uzantısı \texttt{.tex}'tir. Bu basit bir metin
dosyası olup, LaTeX editörleriyle oluşturulup düzenlenebileceği gibi
basit bir metin editörüyle de düzenlenebilir.

Bir belge hazırlamaya başlamak için verilecek ilk komut

\begin{Shaded}
\begin{Highlighting}[]
\NormalTok{\textbackslash{}documentclass[...]\{...\}}
\end{Highlighting}
\end{Shaded}

olup, çengelli parantezler arasına oluşturmak istediğiniz belgenin
sınıfı yazılır. Köşeli parantezlerin içine de isteğe bağlı bazı
değişkenler yazılabilir. Eğer bu kısım boş bırakılırsa LaTeX varsayılan
değerleri alacaktır. Bu komutun ardından sırasıyla \texttt{\textbackslash{}begin\{document\}} ve
\texttt{\textbackslash{}end\{document\}} komutları verilerek belge ortamı oluşturulur.
\texttt{\textbackslash{}end\{document\}} komutuyla LaTeX'e belgenin bittiği söylenmiş olur ve
LaTeX bu komuttan sonra girilenleri dikkate almaz.

\texttt{\textbackslash{}documentclass} komutuyla \texttt{\textbackslash{}begin\{document\}} komutu arasına \emph{sahanlık}
denir. Sahanlık, belgenin ayarlarının yapıldığı kısımdır ve bu kısım
çıktıda görünmez. \texttt{\textbackslash{}begin\{document\}} ile \texttt{\textbackslash{}end\{document\}} arasına da
\emph{gövde} denir. İçerik burada oluşturulur.

Aşağıda asgari bir LaTeX kaynak dosyası gösterilmiştir. \texttt{\textbackslash{}documentclass}
komutunun değişkeni olan \texttt{article}, belgenin makale olacağını belirtir.

\begin{Shaded}
\begin{Highlighting}[]
\NormalTok{\textbackslash{}documentclass\{article\}}
\NormalTok{\textbackslash{}begin\{document\}}
\NormalTok{İşte ilk belgem.}
\NormalTok{\textbackslash{}end\{document\}}
\end{Highlighting}
\end{Shaded}

Bu noktadan sonra örnek kaynak dosyayı LaTeX editörünüzünde oluşturup
önceden oluşturduğunuz bir dizine kaydedin. Kaydederken dosya adında
boşluk ve Türkçe karakter kullanmayın. Örneğin kaynak dosyanız
\texttt{belge1.tex} olsun.

İkinci aşama kaynak dosyanın derlenmesidir. Derleme işlemi için LaTeX
editörlerinde genelde araç çubuğunda oklar bulunur. Oka tıklandığında
dosya derlenir ve sonuç, çıktı ekranında görünür.

Eğer metin editörü kullanıyorsanız derlemeyi uçbirimde (terminal,
konsol,\ldots) yapmanız gerekir. Derleme için uçbirim kaynak dosyanın
olduğu dizinde açılıp

\begin{Shaded}
\begin{Highlighting}[]
\NormalTok{pdflatex belge1}
\end{Highlighting}
\end{Shaded}

komutu verilmelidir.

Derleme işleminden sonra kaynak dosyanızın olduğu dizinde \texttt{belge1.tex}
ve \texttt{belge1.pdf} dosyalarının yanında yine \texttt{belge1} ile başlayan farklı
uzantılara sahip dosyalar olacaktır. Bu dosyaların ne olduklarına
ilerleyen yazılarda değinilecektir ancak dileyen okur \citep[s. 13-14]{Oetiker}'e bakabilir.

\hypertarget{belge-sux131nux131flarux131-ve-seuxe7enekleri}{%
\section{Belge Sınıfları ve Seçenekleri}\label{belge-sux131nux131flarux131-ve-seuxe7enekleri}}

Bölüm \ref{tipik}'de \texttt{\textbackslash{}documentclass} komutunun zorunlu değişkeninin belge
sınıfı olduğunu ve köşeli paratezler içine de seçeneklerin
yazılacağından bahsetmiştik. Bu yazıda bunların neler olabileceklerinden
bahsedelim.

Başka sınıflar olmakla birlikte LaTeX'de varsayılan olarak kullanılan
beş belge sınıfı vardır.

\begin{longtable}[]{@{}ll@{}}
\toprule
\endhead
\texttt{article} & Makale \\
\texttt{report} & Makaleden daha hacimli belgeler için kullanılır. Rapor, tez gibi \\
\texttt{book} & Kitap \\
\texttt{letter} & Mektup \\
\texttt{beamer} & Sunu \\
\bottomrule
\end{longtable}

Bu beş sınıftan \texttt{article}, \texttt{report} ve \texttt{book} için kullanılabilecek
seçenekler aşağıdadır.

\begin{longtable}[]{@{}ll@{}}
\toprule
\endhead
\textbf{10pt, 11pt, 12pt} & Belge ana yazı büyüklüğü. \\
\textbf{a4paper, a5paper, letterpaper,\ldots{}} & Kağıt boyutu. \\
\textbf{fleqn} & Formülleri ortada yazmak yerine, sola bitişik yazar. \\
\textbf{leqno} & Formül numaralarını sağ yerine sol tarafa koyar. \\
\textbf{titlepage, notitlepage} & Belge başlığını attıktan sonra yeni bir sayfa açıp açmayacağını belirler. \\
\textbf{onecolumn, twocolumn} & Belgenin tek sütun veya çift sütun dizileceğini belirtir. \\
\textbf{twoside, oneside} & Belgenin kağıdın hep tek tarafına mı yoksa iki tarafına mı basılacağını belirtir. \\
\textbf{landscape} & Belgeyi enine tutulmuş kağıda basılmak üzere hazırlar. \\
\textbf{openright, openany} & Belgede bölümleri hep sağ sayfalardan veya ilk gelen boş sayfadan başlatır. \\
\textbf{draft, final} & Belgeyi sırasıyla \emph{taslak} ve \emph{son} şeklinde hazırlar. \textbf{draft} seçilirse, sağ taraftan fırlamış olan satırlar kalın siyah bir çizgiyle işaretlenir. \\
\bottomrule
\end{longtable}

Bu seçeneklerin her birinin kullanılabilirliği belge sınıfına göre
farklılık gösterir. Aşağıdaki tabloda hangi seçeneğin hangi sınıf için
varsayılan olduğu ve kullanılabilir olup olmadığı gösterilmiştir.

\begin{longtable}[]{@{}llll@{}}
\toprule
\endhead
\textbf{Seçenek} & \texttt{book} & \texttt{report} & \texttt{article} \\
\textbf{10pt} & 1 & 1 & 1 \\
\textbf{letterpaper} & 1 & 1 & 1 \\
\textbf{oneside} & 1/2 & 1 & 1 \\
\textbf{twoside} & 1 & 1/2 & 1/2 \\
\textbf{openany} & 1/2 & 1 & 0 \\
\textbf{openright} & 1 & 1/2 & 0 \\
\textbf{titlepage} & 1 & 1 & 1/2 \\
\textbf{final} & 1 & 1 & 1 \\
\bottomrule
\end{longtable}

1: varsayılan 1/2: kullanılabilir 0:kullanılamaz

Örneğin belgeye

\begin{Shaded}
\begin{Highlighting}[]
\NormalTok{\textbackslash{}documentclass[a4paper,12pt]\{article\}}
\end{Highlighting}
\end{Shaded}

komutuyla başlarsak LaTeX'e kağıt boyutu A4, ana yazı büyüklüğü 12 punto
olan bir makale yazacağımızı bildirmiş oluruz.

Başka bir örnek

\begin{Shaded}
\begin{Highlighting}[]
\NormalTok{\textbackslash{}documentclass[a5paper,11pt,twocolumn]\{book\}}
\end{Highlighting}
\end{Shaded}

olsun. Bu örnekte kağıt boyutu A5, ana yazı büyüklüğü 11 punto olan bir
kitap yazacağımızı ve kitabın iki sütun olarak dizilmesini söyledik.

\hypertarget{tuxfcrkuxe7e-dil-ayarlarux131-ve-uxe7oklu-dil-kullanux131mux131}{%
\section{Türkçe Dil Ayarları ve Çoklu dil Kullanımı}\label{tuxfcrkuxe7e-dil-ayarlarux131-ve-uxe7oklu-dil-kullanux131mux131}}

LaTeX'de Türkçe belgeler oluşturmak için öncelikle sahanlığa

\begin{Shaded}
\begin{Highlighting}[]
\NormalTok{\textbackslash{}usepackage[T1]\{fontenc\}}
\NormalTok{\textbackslash{}usepackage[turkish]\{babel\}}
\end{Highlighting}
\end{Shaded}

komutlarının verilmesi gerekir.

\texttt{T1} seçenekli \texttt{fontenc} paketi yazıtipi kodlamasıyla ilgili bir paket
olup, hecelemenin doğru şekilde yapılmasını sağlar. Bir çok Avrupa
dilinde de \texttt{T1} seçeneğiyle kullanılır. \texttt{turkish} seçenekli \texttt{babel}
paketi de Chapter, Table, Contents,\ldots{} gibi isimlerin Türkçeleşmesi
(Bölüm, Tablo, İçindekiler,\ldots) içindir.

\begin{quote}
Yakın zamana kadar ö, ş, ç,\ldots{} gibi Türkçe karakterlerin
kullanılabilmesi için sahanlığa \texttt{\textbackslash{}usepackage{[}utf8{]}\{inputenc\}} ya da
\texttt{\textbackslash{}usepackage{[}latin5{]}\{inputenc\}} komutlarından birinin verilmesi
gerekiyordu. Bu paket (\texttt{inputenc}) girdi kodlamasını yöneten bir
pakettir. Son güncellemelerle birlikte bu paketin kullanılma
zorunluluğu ortadan kalkmıştır.
\end{quote}

Aşağıda Türkçe asgari bir LaTeX kaynak dosyası örneği verilmiştir.

\begin{Shaded}
\begin{Highlighting}[]
\NormalTok{\textbackslash{}documentclass\{article\}}
\NormalTok{\textbackslash{}usepackage[T1]\{fontenc\}}
\NormalTok{\textbackslash{}usepackage[turkish]\{babel\}}
\NormalTok{\textbackslash{}begin\{document\}}
\NormalTok{İşte  Türkçe ilk belgem.}
\NormalTok{\textbackslash{}end\{document\}}
\end{Highlighting}
\end{Shaded}

Türkçe dışında ikinci bir dil kullanmak isterseniz, örneğin İngilizce,
\texttt{babel} paketinin seçeneğini

\begin{Shaded}
\begin{Highlighting}[]
\NormalTok{\textbackslash{}usepackage[english,turkish]\{babel\}}
\end{Highlighting}
\end{Shaded}

şeklinde değiştirmeniz gerekir. Burada etkin olan dil Türkçedir.
İngilizceyi etkin hale getirmek için \texttt{\textbackslash{}selectlanguage\{english\}} komutu
kullanılır. Tekrar Türkçeye geçmek için de benzer şekilde
\texttt{\textbackslash{}selectlanguage\{turkish\}} komutu kullanılır.

Bir kelime ya da cümle gibi kısa metinler kullanılacaksa
\texttt{\textbackslash{}foreignlanguage} komutu kullanılabilir:

\begin{Shaded}
\begin{Highlighting}[]
\NormalTok{\textbackslash{}foreignlanguage\{}\SpecialCharTok{\textless{}}\NormalTok{dil}\SpecialCharTok{\textgreater{}}\NormalTok{\}\{}\SpecialCharTok{\textless{}}\NormalTok{metin}\SpecialCharTok{\textgreater{}}\NormalTok{\}}
\end{Highlighting}
\end{Shaded}

Uzun metinler içinse diğer bir seçenek \texttt{otherlanguage} ortamıdır.

\begin{Shaded}
\begin{Highlighting}[]
\NormalTok{\textbackslash{}begin\{otherlanguage\}\{}\SpecialCharTok{\textless{}}\NormalTok{dil}\SpecialCharTok{\textgreater{}}\NormalTok{\}}
\NormalTok{...}
\NormalTok{\textbackslash{}end\{otherlanguage\}}
\end{Highlighting}
\end{Shaded}

Bu ortamın isimleri değiştirmeyen, örneğin, dil seçeneği İngilizce
olmasına rağmen belgeye bir tablo eklediğinizde ``Table'' yerine yine
``Tablo'' adını yazan yıldızlı sürümü de (\texttt{otherlanguage*}) vardır.

\hypertarget{heceleme}{%
\section{Heceleme}\label{heceleme}}

Bazen tüm bu ayarlamalara rağmen LaTeX bazı kelimeleri doğru
heceleyemeyebilir. Böyle durumlarda hecelemeyi elle yapmak gerekir.
Yanlış hecelenen kelimenin bölünebileceği yerler \texttt{\textbackslash{}-} komutuyla
gösterilir:

\begin{Shaded}
\begin{Highlighting}[]
\NormalTok{He\textbackslash{}}\SpecialCharTok{{-}}\NormalTok{ce\textbackslash{}}\SpecialCharTok{{-}}\NormalTok{le\textbackslash{}}\SpecialCharTok{{-}}\NormalTok{me}
\end{Highlighting}
\end{Shaded}

Bu sadece ilgili kelimenin tireyle ayrıldığı yerde doğru hecelenmesini
sağlar. Aynı kelime belgenin başka bir yerinde yine yalnış
hecelenebilir. Bunun yerine \texttt{\textbackslash{}begin\{document\}} komutundan sonra
\texttt{\textbackslash{}hyphenation} komutuyla hece yerleri tire (\texttt{-}) işaretiyle gösterilmiş
olan kelime listesi oluşturulursa belgenin tamamına bu kural uygulanmış
olur. Örneğin

\begin{Shaded}
\begin{Highlighting}[]
\NormalTok{\textbackslash{}hyphenation\{He}\SpecialCharTok{{-}}\NormalTok{ce}\SpecialCharTok{{-}}\NormalTok{le}\SpecialCharTok{{-}}\NormalTok{me FORTRAN\}}
\end{Highlighting}
\end{Shaded}

komutuyla ``Heceleme'' kelimesinin nereden bölüneceği, ``FORTRAN'', ``Fortran'' ya da
``fortran'' kelimelerinin bölünmeyeceği LaTeX'e söylenmiş olur.

\backmatter

  \bibliography{book.bib,packages.bib}

\printindex

\end{document}
